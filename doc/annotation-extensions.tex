\documentclass[10pt]{article}
\usepackage{pslatex}
\usepackage{fancyvrb}
\usepackage{fullpage}
\usepackage{graphicx}
\usepackage{relsize}
\usepackage{url}
\usepackage{xspace}
\usepackage[normalem]{ulem}
\usepackage{hevea}
%HEVEA \footerfalse    % Disable hevea advertisement in footer

\title{Possible extensions to Java annotations}
\author{Michael D. Ernst \\
{\ttfamily mernst@cs.washington.edu}
%HEVEA \\ \today
}

% At least 95% of every float page must be taken up by
% floats; there will be no page with more than 5% white space.
\def\topfraction{.95}
\def\dbltopfraction{\topfraction}
\def\floatpagefraction{\topfraction}     % default .5
\def\dblfloatpagefraction{\topfraction}  % default .5
\def\textfraction{.05}

% Left and right curly braces and backslash, in tt font
\newcommand{\ttlcb}{\texttt{\char "7B}}
\newcommand{\ttrcb}{\texttt{\char "7D}}
\newcommand{\ttbs}{\texttt{\char "5C}}

\begin{document}

\def\codesize{\smaller}
%HEVEA \def\codesize{\relax}
\newcommand{\code}[1]{\ifmmode{\mbox{\codesize\ttfamily{#1}}}\else{\codesize\ttfamily #1}\fi}
\def\<#1>{\code{#1}}
\newcommand{\myurl}[1]{{\codesize\url{#1}}}
%HEVEA \def\myurl{\url}

%NOT HEVEA \makeatletter
%NOT HEVEA \newcommand{\sout}[1]{\@print{<STRIKE>}{}#1\@print{</STRIKE>}}
%NOT HEVEA \makeatother

%HEVEA \def\strut{\relax}

%BEGIN LATEX
\RecustomVerbatimEnvironment{Verbatim}{Verbatim}{fontsize=\smaller}
%END LATEX

% Add line between figure and text
\makeatletter
\def\topfigrule{\kern3\p@ \hrule \kern -3.4\p@} % the \hrule is .4pt high
\def\botfigrule{\kern-3\p@ \hrule \kern 2.6\p@} % the \hrule is .4pt high
\def\dblfigrule{\kern3\p@ \hrule \kern -3.4\p@} % the \hrule is .4pt high
\makeatother

\hyphenation{Pecht-chan-ski}

\newcommand{\jlsURL}{http://java.sun.com/docs/books/jls/third_edition/html/}
\urldef{\jlsEightEightFourURL}\url{http://java.sun.com/docs/books/jls/third_edition/html/classes.html#8.8.4}
\urldef{\jlsNineSixURL}\url{http://java.sun.com/docs/books/jls/third_edition/html/interfaces.html#9.6}
\urldef{\jlsNineSixOneThreeURL}\url{http://java.sun.com/docs/books/jls/third_edition/html/interfaces.html#9.6.1.3}
\urldef{\jlsNineSevenURL}\url{http://java.sun.com/docs/books/jls/third_edition/html/interfaces.html#9.7}
\urldef{\jlsFifteenNine}\url{http://java.sun.com/docs/books/jls/third_edition/html/expressions.html#15.9}
\urldef{\jlsFifteenNineFiveOneURL}\url{http://java.sun.com/docs/books/jls/third_edition/html/expressions.html#15.9.5.1}
\urldef{\jlsFifteenTwelveFourFiveURL}\url{http://java.sun.com/docs/books/jls/third_edition/html/expressions.html#15.12.4.5}



\maketitle

This document is a companion to the Type Annotations Specification (JSR
308).  This document discusses some potential enhancements to annotations
that would be worthwhile, but were not included in JSR 308.
The JSR 308 webpage is \myurl{http://types.cs.washington.edu/jsr308/}.



%HEVEA This document is available in PDF format at \myurl{http://types.cs.washington.edu/jsr308/specification/java-annotation-design.pdf}.


\tableofcontents

%% Oracle folks suggest:
% cfor each proposal, state its impact on
%  * grammar
%  * class file
%  * reflection
% If there is only impact on the grammar, it can be considered a desugaring.



\section{Introduction\label{introduction}}

The Type Annotations (JSR 308) specification extends Java's annotation
syntax.  Java's annotations may be enhanced again in the future, both to
improve type annotations and for other purposes.  It is a goal of the Type
Annotations specification not to unnecessarily close off realistic future
avenues of extension.

This document (which was previously an appendix to the Type Annotations
specification) gives examples of further extensions to Java's annotation
system.
Inclusion here is not an endorsement; this list is intended to be a
resource for current and future language designers, explaining pros and
cons of possible changes.

Java has a massive worldwide user base, so the bar to modifying the Java
language is quite high.  Any proposed modification requires both compelling
use cases (for many users, not just a niche community) that are currently
impossible, and it requires a convincing demonstration (not just a claim)
that there are no negative consequences, such as undesirable interactions
with current language features or tools.  Where possible, this section
gives a brief discussion of benefits and potential problems as a starting
point.  (Problems that potentially apply to all extensions, such as
existing code that reflects over annotations encountering unexpected
structures, are not repeated for each extension.)


\section{Duplicate (multiple, repeated) annotations at a location\label{duplicate-annotations}}

% This is in desperate need of good motivation; right now it is completely
% lacking.

It may be desirable for some
% (but not all)
annotations to be
specified more than once at a single location.
Currently, this is impossible: ``It is a compile-time
error if a declaration is annotated with more than one annotation for a
given annotation type''~\cite[\S9.7]{GoslingJSB2005}.
(By contrast, C\# supports duplicate annotations on a given program element.)

As a related motivation, array-valued annotations can be clumsy to write:

\begin{Verbatim}
  @Resources({
      @Resource(name = "db1", type = DataSource.class)
      @Resource(name = "db2", type = DataSource.class)
  })
  public class MyClass { ... }
\end{Verbatim}

A cleaner syntax is desirable for both purposes:

\begin{Verbatim}
  @Resource(name = "db1", type = DataSource.class)
  @Resource(name = "db2", type = DataSource.class)
  public class MyClass { ... }
\end{Verbatim}

We note two possible approaches to this problem:  desugaring into arrays,
or adding a \<getAnnotations> method to two interfaces.
Among the two proposals, the second is cleaner, more elegant, and easier
for clients to use.  However, it requires either changing existing
interfaces or creating new ones.

For backward compatibility, in both approaches an annotation may be
repeated only if its definition is meta-annotated with \code{@Repeatable}:

\begin{Verbatim} 
  public @interface Repeatable {
    public boolean value() = true;
  }
\end{Verbatim}

\noindent
In the below, assume that the definitions of \<A> and \<B> are
meta-annotated with \<@Repeatable>.

% \noindent
% For annotations not so meta-annotated, the compiler throws an
% error/warning just as for Java SE 6.


It is already possible in Java SE 6 for a location to have both an
inherited and an explicit annotation of the same type.  The 
\code{@Inherited} meta-annotation specification states that a
\code{getAnnotation}[\code{s}] query returns only the
explicit annotation.  This requirement should perhaps be relaxed, even if
inheritance among annotations (Section~\ref{inheritance-among-annotations})
is not permitted.  For some annotation
processors, it is best to check that the explicit annotation is consistent
with the overridden, inherited one and possibly merge information in the
two annotations.  The current design makes that inconvenient to do.


\subsection{Desugar into wrapper annotations\label{duplicate-annotations-desugar}}

% Joe Darcy and Danny Coward have volunteered (in 2006 or 2007) to produce
% a proposal on this, but they have been foot-dragging since early 2007,
% and so I have given up on nagging and waiting.
Desugar duplicate annotations into
the current array syntax.  For instance, desugar

\begin{Verbatim}
  @A(1) @B @A(2) Object x;
\end{Verbatim}

\noindent
into

\begin{Verbatim}
  @AContainer({@A(1), @A(2)}) @B Object x;
\end{Verbatim}

This approach treats duplicate annotations as purely a syntactic
convenience; it does not change annotations in any deep way.  This approach
is compatible with certain existing J2EE annotation processors that are
already written to process both \code{@A} and \code{@AContainer}.

This proposal would need to specify what happens if only one \code{@A}
annotation is present.  Most likely, the desugaring
would still place \code{@A} into a singleton \code{@AContainer} annotation.
Otherwise, client code would become more complicated, since it would have
to deal with the possibility of either an \code{@A} annotation or an
\code{@AContainer} annotation.

This proposal would need to specify what happens if both \code{@A} and
\code{@AContainer} annotations are present.  Most likely, the desugaring
would insert \code{@A} into the existing \code{@AContainer} annotation,
either at the front or the back depending on whether the \<@A> or
\<@AContainer> annotation came first.
Otherwise, client code would become more complicated, since it would have
to deal with the possibility of both an \code{@A} annotation or an
\code{@AContainer} annotation, and with handling the relative order.

This proposal would need to specify what happens if the user queries for
annotation \code{@A}:  would that always throw an exception (probably the
best choice), or succeed if one \code{@A} annotation is present, or succeed
if an \code{@AContainer} annotation with a length-1 array is present, or
return the first \code{@A} annotation?

A problem with this proposal is that it requires defining an \code{@AContainer}
annotation for each annotation \code{@A}, or else annotation \code{@A}
cannot be duplicated.  It would be better not to burden programmers with
writing and maintaining this boilerplate code.

Another problem with this proposal is that it makes other possible
extensions to annotations, such as inheritance among annotations
(Section~\ref{inheritance-among-annotations}), more difficult and clumsy to
implement and use.  JSR 308 should aim to preserve, not to preclude, future
avenues of extension.

A final, rather minor, problem is that this proposal loses the ordering of
differently-named annotations (even if the ordering of same-named
annotations is preserved within the container).  For example, it cannot
distinguish these
declarations:
\begin{Verbatim}
  @A(1) @B @A(2) Object x;
  @A(1) @A(2) @B Object x;
\end{Verbatim}


%% More details from Joe Darcy, but I don't see that they add much to the
%% shorter description above.
% A new meta-annotation is introduced to allow an annotation type
% to declare what its containing type should be:
% 
% @interface Container {
%   Class<? extends Annotation> value();
% }
% 
% so that the declaration
% 
% @Container(MyRepeatedAnnotationContainer.class)
% @interface MyAnnotation()
% 
% would allow something like
% 
% @MyAnnotation(1) @MyAnnotation(2) @MyAnnotation(3)
% public class Foo();
% 
% to be treated as
% 
% @MyRepeatedAnnotationContainer(
%   {@MyAnnotation(1),
%    @MyAnnotation(2),
%    @MyAnnotation(3)})
% public class Foo();
% 
% (I'm not sure what the exact syntactic sugar should be; it might be
% beneficial to wrap the repeated annotations in "@{", "}" to emphasize
% the grouping or to restrict the repeated annotations to be sequentially
% applied.)
% 
% Declaring the container in the annotation type provides the compiler and
% the runtime the information needed to construct or lookup the containing
% type.  There could be a new method on AnnotatedElement like
% 
% <T extends Annotation> T[] getRepeatedAnnotation(Class<T> annotationClass)
% 
% or
% 
% <T extends Annotation> List<T> getRepeatedAnnotation(Class<T>
% annotationClass)
% 
% This method could potentially handle some of the tricky cases like
% returning the non-contained annotation if there is one, otherwise return
% the contained annotation, etc.
% 
% The intention is to allow the @Container meta-annotation to be
% retrofitted to existing wrapper annotations.


\subsection{Add new \<getAnnotations> method\label{duplicate-annotations-getAnnotations}}

Add a new \<getAnnotations> method, which returns multiple annotations,
to two interfaces: \code{java.lang.reflect.AnnotatedElement} and
\code{javax.lang.model.element.Element}.

\begin{enumerate}

\item
Create a new method

\begin{Verbatim}
  <T extends Annotation> T[] getAnnotations(Class<T> annotationClass)
\end{Verbatim}

% This method returns an array and not a List for consistency with existing
% methods getAnnotations and getDeclaredAnnotations.

\noindent
that returns all annotations for the specified annotation type that are
present on the receiver element.  As with \code{getDeclaredAnnotations},
the return value is never null but may have zero length.

\item
Slightly modify the specification of the existing method

\begin{Verbatim}
  <T extends Annotation> T getAnnotation(Class<T> annotationClass)
\end{Verbatim}

\noindent
When duplicate annotations exist, method \code{getAnnotation} could either
give the first one (convenient for backward compatibility) or throw an
exception (convenient to prevent erroneous processing).  The latter seems
like a better choice.

% Classes:
%   java.lang.Class.getAnnotation(java.lang.Class)
%   java.lang.Package.getAnnotation(java.lang.Class)
%   java.lang.reflect.AccessibleObject.getAnnotation(java.lang.Class)
%   java.lang.reflect.Constructor.getAnnotation(java.lang.Class)
%   java.lang.reflect.Field.getAnnotation(java.lang.Class)
%   java.lang.reflect.Method.getAnnotation(java.lang.Class)
% Interfaces:
%   java.lang.reflect.AnnotatedElement.getAnnotation(java.lang.Class)
%   javax.lang.model.element.Element.getAnnotation(java.lang.Class)

\end{enumerate}

The arrays returned by the methods \code{getAnnotations(Class)},  
\code{getAnnotations()}, and \code{getDeclaredAnnotations()} are required
to preserve the ordering of the annotations as they were declared in the
original \code{.java} source file.

No other changes would be necessary.  Existing code that uses
some other workaround (like special \code{@AContainer} annotations) would
continue to work.  Or, it could be converted to take advantage of this new
mechanism.

The problem with this proposal is that it requires adding a single method,
\<getAnnotations>, to two interfaces.  While the changes would not break
upward binary-compatibility, they would break source-compatibility:
existing implementations of \<Element> and \<AnnotatedElement> could not
be re-compiled under Java 8 without adding a \<getAnnotations> method.
Oracle's policy generally frowns upon such changes that break
source-compatibility, though Oracle sometimes makes them nonetheless.
% Such changes have to be approved by the CCC (conformance change committee).
For example, the
SQL interfaces have far more implementations than the compiler or
reflection APIs, but Java SE 6 added three methods to the
\<java.sql.Statement> interface and made it extend \<Wrapper>.
% This isn't even documented in the Java 6 Incompatibility Release notes
% (http://www.oracle.com/technetwork/java/javase/compatibility-137541.html#incompatibilities; was http://java.sun.com/javase/6/webnotes/compatibility.html#incompatibilities).

An alternative to changing the \<Element> (and \<AnnotatedElement>) interface is to create a new
\<Element2> interface that augments \<Element> with the \<getAnnotations>
method, and to mark \<Element> as deprecated.  Then, programmers should
rewrite their existing code to use the \<Element2> interface instead of the
\<Element> interface.  However, their code continues to compile until they
make the change.  A problem with this change is that it requires making a
copy (with names suffixed ``\<2>'') of the entire \<javax.lang.model>
package, which is ugly.

Another alternative is to change the \<Elements> utility interface.  The
new \<getAnnotations> methods logically belong in \<Element>, but they can
be added to the \<Elements> interface because it is explicitly documented
with ``Compatibility Note: Methods may be added to this interface in future
releases of the platform.''  Thus, adding static methods in \<Elements> is
a path toward extending functionality without changing core
interfaces/classes such as \<Element>.  A problem with this proposal is
that clients must remember to use the methods of \<Elements> when
functionality is not present in \<Element>.  A programmer browsing the
\<Element> interface might not think to check for static methods in
\<Elements>.


\subsection{JSR 175 design justification\label{duplicate-annotations-jsr175-justification}}

The JSR 175 Design FAQ~\cite{JSR175-PFD2} briefly addresses why this
feature was not originally included:

\begin{quote}
Why is it illegal [to] annotate a single element repeatedly with the same annotation type?

A similar effect is available by annotating once with an annotation type
whose sole element is of an array type. The resulting system is simpler and
cleaner: you don't have to worry about whether you're going to get back a
single instance of the annotation type or multiple instances when you query
an annotation. Also it's consistent with the usual restriction on
modifiers.
\end{quote}

The argument that ``the resulting system is simpler and cleaner'' is
perplexing.  A client must check for both the singular and plural versions
of the annotation and must handle the case when both the singular and the
plural versions are present.  By contrast to the stated claim, a system for
duplicate annotations with an interface that always returns a list (or
always returns an array) is what is needed to relieve the client of ``worry
about whether you're going to get back a single instance''.


\section{Locations for annotations\label{locations-for-annotations}}


\subsection{Annotations on statements\label{statement-annotations}}

Annotations on statements (or on some subset of statements, such as blocks
or loops) would be useful for a variety of purposes, including
atomicity/concurrency.  Supporting annotations on statements would require
defining both Java syntax and a convention for storing the information in
the class file.  See
\myurl{http://code.google.com/p/jsr308-langtools/wiki/AnnotationsOnStatements} for a
proposal that summarizes why statement annotations are desirable, and that
proposes a Java syntax, a classfile storage format, and how other tools
will accommodate them.  If you would like to help this feature become a
reality, then please pitch in!  You can join the
\code{jsr308-discuss@googlegroups.com} mailing list (via
\myurl{http://groups.google.com/group/jsr308-discuss}),
expand the partial design on the wiki, or work on the implementation.

The JSR 175 Design FAQ~\cite{JSR175-PFD2} briefly addresses why this
feature was not originally included:

\begin{quote}
  Why can't you annotate arbitrary program elements such as blocks
    and individual statements?

  This would greatly complicate the annotation syntax: We would have to
  sacrifice the simplicity of saying that annotations are simply modifiers,
  which can be used on declarations.
\end{quote}


\subsection{Expression annotations\label{expression-annotations}}

Annotating an expression
indicates some property of the computation, such as that it should
be performed atomically, that it acquires no locks, or that it should be
formatted specially by an IDE\@.  JSR 308 does not support expression
annotations, because we have not yet discovered compelling use cases for
them that cannot be equally well supported by statement annotations.  (A
minor inconvenience is that the use of statement annotations may require
the programmer to create a separate statement for the expression to be
annotated.)
Expression annotations are not type annotations and are different than
annotating a type cast, which indicates a property of a value (the result
of an expression).


\subsection{Implicit Java types in casts\label{implicit-java-types-in-casts}}

Arbitrary values can be annotated using an annotation on a cast:
\begin{Verbatim}
  (@Language("SQL") String) "select * from foo"
\end{Verbatim}

A possible shorthand would be to permit the Java type to be implicit:
\begin{Verbatim}
  (@Language("SQL")) "select * from foo"
\end{Verbatim}
This is not permitted, nor may a cast be omitted in a type test, as in
``\code{x instanceof @NonNull}''.  There are several reasons for this
decision:
\begin{enumerate}
\item Erasing the annotations should leave a valid Java program.
\item Stating the type reinforces that the annotation is a
type annotation rather than an expression annotation.
\item Especially in a type test, stating the type reinforces that the run-time
effect is to check and change the Java type.  In general, no run-time check
of the annotation is possible.
\item The benefit of omitting the type in the cast seems relatively minor.
\end{enumerate}

An even shorter shorthand would drop the parentheses:
\begin{Verbatim}
  @Language("SQL") "select * from foo"
\end{Verbatim}
In addition to the benefits and problems noted above, such an annotation is
syntactically ambiguous with an expression annotation.
Whether an annotation applies to expressions or to types is clear from the
annotation's documentation and its \code{@Target} meta-annotation,
similarly to how it is determined whether an annotation applies to a type
or to a declaration.
% (Section~\ref{disambiguating}).

% But a problem remains.
% If a within-brackets proposal for arrays
% % (Section~\ref{array-syntax})
% is
% chosen, then \code{Object[@X 2]}
% might mean that the array level has annotation \code{X} or that it's shorthand
% for \code{Object[(@X int) 2]}; similar confusion might occur when the array
% length is a more complex expression.

% Possible problem: @Readonly means something different when applied to an
% expression and to a type, though perhaps a different name would be better
% for the expression.


%       One advantage of the expression version is that casts in general
%       shouldn't appear in code (and tools may raise warnings about them).
%       But the same is true of annotation casts!
%       Compilers shouldn't complain about redundant casts that are
%       annotated (but annotation processors should).


\subsection{Only certain statements\label{locations-subset-of-statements}}

It would be possible to permit annotations only on blocks and/or loops, as
a restricted special case of statements.  This is less general than
permitting annotations on statements, and uses are more syntactically
cluttered (for instance, this requires a statement to be converted into a
block before it can be annotated).  Most declarations could not be
annotated as statements because enclosing the declaration in a block to
annotate it would change (and limit) the variable's scope.  This limitation
in flexibility does yield the advantage that there would be no syntactic
ambiguity between (say) statement annotations and declaration or type
annotations.

Similarly, permitting annotations on partial constructs (such as only the
body of a loop) appears both more complex, and no more useful, than
annotating complete constructs (such as a full statement).


\section{Changes to the annotation type\label{changes-to-the-annotation-type}}


\subsection{Inheritance among annotations (subclassing/subtyping annotations)\label{inheritance-among-annotations}}

% Multiple inheritance (i.e., from interfaces) when subtyping annotations:
% 
% According to the Java Language Specification section 9.4.1:
% 
% > It is possible for an interface to inherit several methods with  
% > override-equivalent signatures (§8.4.2). Such a situation does not  
% > in itself cause a compile-time error. The interface is considered to  
% > inherit all the methods. However, one of the inherited methods must  
% > must be return type substitutable for any other inherited method;  
% > otherwise, a compile-time error occurs (The throws clauses do not  
% > cause errors in this case.)
% 
% Therefore, we would make a similar rule regarding the properties/ 
% methods of annotation types.  Since all methods on annotation types  
% take no arguments, their signatures are trivially override- 
% equivalent.  Therefore, you can combine inherit multiple properties  
% with the same name so long as one of them has a "most specific" return  
% type that is substitutable for all the others.  I could not find the  
% precise definition of when a type is "substitutable" for another,  
% perhaps it refers to the return type substitution of 8.4.5?  I guess  
% that here I would simply to reproduce the behavior of interfaces,  
% which I believe must also resolve the return type of a multiply- 
% inherited method in such a scenario.
% 
% However, there is one additional wrinkle that occurs with annotations  
% that do not occur with interfaces: default values.  The simplest rule  
% here would be to say that at most one inherited method can supply a  
% default value.  Alternatively, any number can supply default values  
% but they must all have the same type and value.
% 
% The same rule would then apply to "overriding" members declared in sub- 
% annotation-types: in other words, you can redeclare a property from a  
% supertype, but the return type and default value must be compatible as  
% described above.


% \label{subclassing-annotations}\label{subtyping-annotations}

In Java SE 6, annotations cannot subclass one another, and an annotation is not allowed
to extend/implement any interfaces.  These restrictions make it difficult
to share behavior or to express similarities or relationships among
annotation types.  For example, it is inconvenient to define annotations
with choices in their structure, such as a discriminated union that uses
field names that act as explicit tags.  It is impossible to create
annotation processors or APIs that work with a specific set of annotations
(say, all those with a given set of element names).

Permitting inheritance among annotations would solve these problems.  (To
work around the problem, one could meta-annotate an annotation as a
``subannotation'' of another, and then the annotation processor could do
all the work to interpret the meta-annotation.  This is clumsy and
indirect.)

A potential objection is that multiple annotation processors might try to
claim the same annotation, one directly and one by claiming its supertype.
This can be accommodated.  In general, annotation processor designers
should, and can, avoid this situation by running one processor at a time
for a family of annotations that are related by inheritance.


%% This text is inaccurate; JLS 9.6 merely says that an member can have an
%% annotation type, and elsewhere inheritance among annotations is prohibited.
% Inheritance among annotations would slightly relax Java's current
% restriction on annotations as annotation members, namely that an annotation
% can have a member that is an annotation \emph{of a specific type}.

Without other changes, inheritance among annotations would not enable
recursive or self-referential annotations (see
Section~\ref{annotations-as-members}).

There are two general designs that permit inheritance among annotations:
permitting annotations to subclass other annotations, or permitting
annotations to extend/implement interfaces, or both.
We briefly elaborate on the two designs.

\paragraph{Subclassing annotations}

Permitting inheritance among annotations requires that the \code{final}
modifier on an annotation type declaration works just like for ordinary
classes:  it prevents creation of subtypes of the given annotation.  A
framework that defines annotation \code{@A} may not want to load an
untrusted subclass \code{@B} of \code{@A} into the framework.

The specification of the \code{getAnnotation(Class)} method must be
updated.  Suppose that annotation types \code{Sub1} and \code{Sub2} are
direct subtypes of \code{Super}, and \code{Super} has a \code{value} field
of type \code{int}.  The specification must indicate the behavior of
\code{getAnnotation(Super.class)} when called on locations with all
possible combinations of the three annotations, with all possible
combinations of arguments.  Here are a few examples:

\begin{itemize}
\item
 a location that is annotated with \code{@Sub1(1)} but not annotated with
 \code{@Super}
\item
 a location that is annotated with both \code{@Super(0)} and \code{@Sub1(1)}
\item
 a location that is annotated with both \code{@Sub1(0)} and
 \code{@Sub2(0)}, but not annotated with \code{@Super}
\item
 a location that is annotated with \code{@Sub1(1)} and \code{@Sub2(2)}, but
 not annotated with \code{@Super}
\end{itemize}

\noindent
Reflection APIs may also need to be updated.

As is clear from the examples above,
subclassing annotations raises the possibility of multiple effective
annotations applying at a location.
Therefore, subclassing annotations makes most sense if the specification
already permits duplicate annotations (see
Section~\ref{duplicate-annotations}).
In that case,
the new \code{getAnnotations(Class)} method returns all of the annotations.
(Then, the client can decide what to do with them, such as raise an
error, merge the annotations, or take some other action.)


The JSR 175 Design FAQ~\cite{JSR175-PFD2} briefly addresses why this
feature was not originally included:

\begin{quote}
Why don't you support annotation subtyping (where one annotation type
extends another)?

It complicates the annotation type system, and makes it much more difficult
to write ``Specific Tools''.

\ldots

``Specific Tools'' --- Programs that query known annotation types of
arbitrary external programs. Stub generators, for example, fall into this
category. These programs will read annotated classes without loading them
into the virtual machine, but will load annotation interfaces.
\end{quote}

% My best guess is that the "specific tools" would need to look not just
% for a specific annotation, but would have to check every annotation to
% see whether it's a subclass of the one the tool is written for.  And, the
% specific tool doesn't want to load untrusted annotations (or, for that
% matter, all annotations) just to check whether they are a subclass of the
% specific annotation.

% Support for the "final" modifier may answer much of this objection.


\paragraph{Extending/implementing interfaces}

%% Joe Darcy says:
% I would support allowing annotation type to extend interfaces, including
% when those interfaces happened to be annotation types, keeping the current
% restrictions on what methods can be declared in an annotation type.  There
% would be no deeper annotation subtyping relation between annotation types
% where one extended the other in terms of how they were processed, etc.


Permitting annotation types to extend/implement interfaces is less
powerful than subclassing among annotations.
For example, interfaces can't define default values, nor is it possible
to subclass an annotation to enhance functionality.
An advantage of this approach is that there is no possibility of multiple
annotations with a given supertype appearing at a location.

One use case is marker interfaces.
%
In another use case, the interface could define several elements/methods.
These could define a standard set of element names that are used by a set
of specific annotations.  APIs that use the interface could work with any
annotation from that set.

The currently permitted annotation member types (``primitive types, String,
Class and any invocation of Class, an enum type, an annotation type, or an
array of one of the preceding types''~\cite[\S9.6]{GoslingJSB2005}) should
be extended to include interfaces (or perhaps only interfaces that extend
\code{java.lang.annotation.Annotation}).


% Does this example belong here?
%   @Units(value={@Meters}, dividedBy={@Seconds}




\subsection{Potentially recursive annotations (annotations as members of annotations)\label{annotations-as-members}\label{annotations-as-arguments}\label{recursive-annotations}}

% Annotations as members of annotations
% 
% There is no need to change the classfile format.  Here is an email exchange
% where Joe Darcy says that a change is necessary, and Jonathan Aldrich
% explains that it is not.
% 
% Joe Darcy says:
% 
% One implication of being about to 
% return an arbitrary annotation is that the binary encoding of the 
% annotation data in the class files would need to change.  The actual 
% annotation objects returned at runtime are dynamic proxies implementing 
% the annotation as an interface.  For example, in something like
% 
% @Constraint(
%  trigger={"Child(ddl, this)", "Type(ddl, DropDownListCtrl.class)", 
% "Equals(select, true)"},
%  post={"!Selected(ddl)"}
% )
% 
% the fact that an array of strings needs to be returned for "trigger" is 
% gleaned from class definition of the Constraint annotation type.  If 
% that reasoning path is changed, this type information needs to be stored 
% somewhere else.  Such a change to the annotation encoding implies 
% downstream tools that consume classfiles and manipulate annotations need 
% to be updated too, etc.  So this change would actually be much more 
% pervasive throughout the Java ecosystem than, say, a repeating 
% annotation proposal which just desugars to a wrapper annotation with an 
% array-valued member, as has been proposed to the list before.  (As an 
% aside, changing the API to basically require instanceof checks is by 
% default a bit suspect.)
% 
% Jonathan Aldrich responds:
% 
% Joe raised a great question 
% about whether annotations taking other annotations as parameters would 
% require changing the class file format as well.  As far as I can tell, 
% from the specification of the Java 5 classfile format, it would not:
% 
% http://java.sun.com/docs/books/jvms/second_edition/ClassFileFormat-Java5.pdf
% 
% Specifically, annotations are stored in an "annotation" structure in the 
% classfile.  This structure contains a type index (the type of the 
% annotation) and a list of element_value_pairs.  The value in the 
% element_value_pair is an element_value, which is a tagged union, and if 
% the tag is "@" for annotation, then the union stores an "annotation" 
% structure.  That's the annotation structure above--which in fact 
% contains a type index, that in the case of "annotations taking arbitrary 
% annotations" tells us what type the nested annotation is.  Ironically, 
% in the current classfile format, this is wasted space, as it could be 
% inferred from the surrounding annotation type, but it nicely supports 
% this extension.  Thus, "annotations containing arbitrary annotations" 
% need change neither the syntax of the language nor the classfile format. 


In Java, an annotation can have a parameter (equivalently, a member) that is an
annotation (JLS
\ahref{\jlsNineSixURL}{\S9.6} and
\ahref{\jlsNineSevenURL}{\S9.7}).
JLS
\ahref{\jlsNineSevenURL}{\S9.7}
gives this example:

\begin{Verbatim}
  @Author(@Name(first = "Joe", last = "Hacker"))
\end{Verbatim}

However, an annotation type cannot have a member of its own type.
``It is a compile-time error if an annotation type \emph{T} contains an
element of type \emph{T}, either directly or
indirectly''~\cite[\S9.6]{GoslingJSB2005}.
(If inheritance among annotations is permitted (see
Section~\ref{inheritance-among-annotations}), then the natural
interpretation of this is that no element's declared type may be either the
annotation itself or any supertype thereof.)

As a particular example, it is not possible to define an annotation that
takes an \emph{arbitrary} annotation as a parameter, as in

\begin{Verbatim}
  @DefaultAnnotation(@AnyAnnotation)
\end{Verbatim}

These limitations reduce the expressiveness of annotations.  Here are some
examples:
\begin{itemize}
\item
  It is impossible to define annotations that take an arbitrary annotation
  as an argument.

  Two examples of such annotations are the \code{@DefaultAnnotation}
  example above, and an annotation that expresses that a method is
  polymorphic over annotations (as opposed to polymorphic over types, as
  generics do).

  Another example is annotation parameters that do not correspond to a type
  parameter.  Consider a container class that is designed to hold things of
  a specific type (say, it interacts with the elements in domain-specific
  ways), not of an arbitrary type.  The container has no type parameter,
  but it may still make sense to have annotations on the contained thing.
  Most examples of generics and arrays 
  % from Section~\ref{generics-and-arrays}
  also serve as examples here.  The
  argument applies not just to container classes but also to algorithms
  that process data of a specific type.

  These examples require the annotation declaration to be potentially
  recursive.  However, they do not require that any recursive annotation
  ever be created in practice.  They could be achieved even if creation of
  recursive annotation instances was prohibited.

\item
  % Need examples here; Jonathan Aldrich volunteered to provide them.
  It is impossible to define annotations with recursive structure.

  More generally, the kind of data that can be encoded in annotations is
  limited.  It is generally possible to store data which has a fixed
  structure, but it is much more difficult to encode more complex data,
  including hierarchical data.  As the success of \ahref{http://www.w3.org/XML/}{XML}, \ahref{http://www.yaml.org/}{YAML}, and other such
  technologies has shown, the ability to encode structured data is very
  powerful for a wide variety of uses, many of which we cannot anticipate
  today.  Today's annotations only fulfill a small piece of that potential.

  To express the base case, recursive annotations also require either
  inheritance among annotations
  (Section~\ref{inheritance-among-annotations}), or nullable members
  (Section~\ref{nullable-members}), or a requirement that recursion takes
  place through an array (an empty array would terminate the recursion).

\end{itemize}

% - What about the behavior of existing code that reflects over annotation 
% types and expects never to see recursively-typed members?


A more modest approach that makes annotations somewhat more expressive
would be to permit inheritance among annotations (see
Section~\ref{inheritance-among-annotations}) without permitting possibly
self-referential annotations.


The JSR 175 Design FAQ~\cite{JSR175-PFD2} briefly addresses why this
feature was not originally included:

\begin{quote}
  Why is it illegal for an annotation type to contain an element of the same type?

At first glance it would appear that you could never annotate with such a type, so the prohibition might seem unnecessary, but you can get yourself into trouble if you also provide a default:

\begin{Verbatim}
@interface Foo {
    Foo foo() default @Foo;
}

@Foo int i;
\end{Verbatim}

If this were legal, the resulting annotation would be self-referential (or
infinite), as well as useless.
\end{quote}


\subsection{Null as an annotation field value (nullable fields)\label{nullable-members}}

Currently, it is possible to choose/create a special value and to store
that value in an annotation field.
The proposal would permit use of \code{null} instead of an explicit
user-specified value.

The proposal makes some programming idioms slightly shorter.  For example,
specifying the base case of a recursive data structure will require simply
writing \code{null} instead of calling a constructor.  The proposal does
not eliminate any substantial complexity when processing a data structure,
but only converts a check against a given value into a check against null.

We note some possible objections to the proposal.

The proposal doesn't make anything possible that was not possible before.

The programmer-defined special value provides better documentation than
\code{null}, which might mean ``none'', ``uninitialized'', \code{null}
itself, etc.

The proposal is more error-prone.  It's much easier to forget checking
against null than to forget checking for an explicit value.

The proposal may make the standard idiom more verbose.  Currently only the
users of an annotation need to check for its special values.  With the
proposal, many tools that process annotations will have to check whether a
field's value is null lest they throw a null pointer exception.


\subsection{Positional arguments\label{positional-arguments}}

Annotation types cannot have positional arguments (except for the
\code{value} argument, when it is the only argument).  This limitation
makes writing annotations with multiple arguments more verbose than
necessary.

% Suggestion from Bruce Chapman
On a somewhat related topic, the ``SingleElementAnnotation'' form relies on
the field name \code{value}.  This name is not always appropriate;
designers who wish to permit the ``SingleElementAnnotation'' form are
forced to give a member a confusing name.  It would be nice to permit the
``SingleElementAnnotation'' form to initialize a different member than
\code{value} (say, via a \code{@ValueMember} annotation).


\section{Semantics of annotations\label{semantics}}

\subsection{Annotations for specific purposes\label{no-defined-annotations}}

JSR 308 does not define any annotations.
%  nor take any position on their semantics.
JSR 308 extends the Java and class file syntax to permit
annotations to be written in more places, and thus makes existing and
future annotations more useful to programmers.

By contrast, JSR 305 ``Annotations for Software Defect Detection'' aims to
define a small set of annotations.
Examples include type annotations such as non-nullness
(\code{@Nonnull}), signedness (\code{@Nonnegative}), tainting, and string
format; and also declaration annotations such as whether a method's return
value should always be checked by the caller.
A programmer who cares about code quality will use both annotations
defined in the JSR 305 ``standard library'', and also others that are
defined by third parties or by the programmer himself.
For more details about JSR 305, see
\myurl{http://jcp.org/en/jsr/detail?id=305} and
\myurl{http://groups.google.com/group/jsr-305/}.
Since JSR 305 appears to be abandoned, this section is now primarily of
historical interest.

Any type annotation, including those defined by JSR 305, is of limited use
without the JSR 308 syntax.  Without the JSR 308 annotation syntax, a
static checker may lose track of type qualifiers whenever a program uses
generic types (e.g., collection classes), whenever a method is invoked on
an object, whenever a cast is performed, whenever a class is subclassed,
etc.  From the point of code checking, using the old Java annotation syntax
is even worse than the type unsoundness of pre-generics Java, when there
was no compiler-enforced type correctness guarantee for collections
classes.  Therefore, use of JSR 305 without JSR 308 is much less effective.

% As of fall 2008,
The JSR 305 team does not plan any reference implementation.
This hinders both programmers who want to use the annotations,
and also people trying to interpret the meaning of the specification.  By
contrast, the JSR 308 reference implementation and the Checker Framework
for compile-time type-checking have been available since January 2007.
The Checker Framework contains checkers for many of the most useful
annotations proposed in JSR 305.




% For
% instance, it is JSR 305's role to decide whether Java programmers will
% write \code{@NonNull} or \code{@Nonnull} or \code{@NotNull}, and what it
% means when they write it.


\subsection{Annotation inheritance\label{annotation-inheritance}}

(This section is \emph{not} about whether the definition of an annotation
may inherit from other classes/interfaces.  Rather, it is about whether a
class/interface inherits annotations from its superclasses/interfaces.)

The annotation type
\ahref{http://download.oracle.com/javase/6/docs/api/java/lang/annotation/Inherited.html}{\code{java.lang.annotation.Inherited}}
(JLS \ahref{\jlsNineSixOneThreeURL}{\S9.6.1.3})
indicates that annotations on a class \code{C} corresponding to a given
annotation type are inherited by subclasses of \code{C}\@.  This implies
that annotations on interfaces are not inherited, nor are annotations on
members (methods, constructors, fields, etc.).

It might be useful to permit methods, fields, etc.\ to inherit annotations.

It might be useful to permit an annotation to be inherited from an
interface as well as from a superclass.

It might be useful to permit annotation inheritance to merge
information from the current and inherited annotations, instead of always
choosing the inherited one.

% These semantic issues are out of the scope of JSR 308 but may be
% taken up by JSR 305 (``Annotations for Software
% Defect Detection''~\cite{JSR305}).



% (An annotation
% processor can issue a warning if annotations are inconsistent between
% super- and sub-types.)

%   <li>A dynamic proxy class is a class (created by
%       <a href="http://download.oracle.com/javase/6/docs/api/java/lang/reflect/Proxy.html"><code>java.lang.reflect.Proxy</code></a>) that is created at run time and
%       that implements a list of interfaces (that are also given at run
%       time).  What annotations should appear on the methods of the proxy
%       class?  Arguably, whichever annotations were on the methods of the
%       given interfaces (when a method appears more than once, whatever
%       annotations are on the method whose interface appears first in the
%       list, which takes precedence).
% 
%       But whether an annotation should be transferred to a proxy
%       class may depend on the annotation itself; should there be a way to
%       indicate how it should be treated?  This feels related to the issue
%       of inheritance and defaulting of annotations; perhaps any tool that
%       processes the annotations should also look up the annotations on the
%       interface(s) and use them, so it does not matter whether the
%       generated class itself has the annotations or not.
% 
%       How does Java handle this currently?  Do we believe the
%       current behavior is incorrect and needs to be amended in JSR 308?
%   </li>

\subsection{Default annotations\label{default-annotations}}

Specifying a default for annotations can reduce code size and (when used
carefully and sparingly) increase code readability.  For instance,
% Figure~\ref{fig:example} uses
a \code{@DefaultQualifier(NonNull.class)} can avoid the clutter of
% 5
multiple
\code{@NonNull} annotations.  It would be nicer to have a general mechanism,
such as
\begin{Verbatim}
  @DefaultAnnotation(NonNull.class, locations={ElementType.LOCAL_VARIABLE})
\end{Verbatim}
Defaults for annotations are a semantic issue that is
out of the scope of JSR 308.  It could be taken up by JSR 305 (``Annotations
for Software Defect Detection''~\cite{JSR305}).

The defaulting syntax must also be able to specify the arguments to the
default annotation (in the above example, the arguments to \code{@NonNull}).

A better syntax would use an annotation, not a string or class literal, as the
argument to \code{@DefaultAnnotation}, as in
\begin{Verbatim}
  @DefaultAnnotation(@MyAnnotation(arg="foo"))
\end{Verbatim}
In Java, it is not possible to define an annotation that takes an
arbitrary annotation as a parameter; see
Section~\ref{annotations-as-arguments}.


An issue for JSR 260 (Javadoc) and JSR 305 (Annotation semantics) is how
inherited and defaulted annotations are handled in Javadoc:  whether they
are written out in full, or in some abbreviated form.  Just as too many
annotations may clutter source code, similar clutter-reduction ideas may
need to be applied to Javadoc.


\section{Type abbreviations and typedefs\label{typedef}}

An annotated type may be long and hard to read; compare \code{Map<String,
  Object>} to \code{@NonNull Map<@NonNull String, @NonNull Object>}.  Class
inheritance annotations and subclassing provides a partial solution;
consider
% , as noted
% %BEGIN LATEX
% on page~\pageref{class-inheritance-annotations}
% %END LATEX
% in Section~\ref{class-inheritance-annotations} with
the following example:

\begin{Verbatim}
  final class MyStringMap extends
    @Readonly Map<@NonNull String, @NonEmpty List<@NonNull @Readonly String>> {}
\end{Verbatim}

This approach limits reusability:  if a method is declared to take a
\code{MyStringMap} parameter, then a \code{Map} (even of the right type,
including annotations) cannot be passed to it.  (By contrast, a
\code{MyStringMap} can always be used where a \code{Map} of the appropriate
type is expected.)  Goetz~\cite{Goetz2006} recommends exploiting Java's
type inference to avoid some (but not all) instances of the long type name.

In summary, a built-in typedef mechanism might achieve both code
readability and reusability.


\section{Class file syntax\label{class-file-syntax}}

Changes to the class file syntax are out of the scope of JSR 308, which,
for backward compatibility, currently does not change the way that existing
annotations are stored in the class file.
Class file syntax changes require modification of
compilers, JVMs, javap, and other class file tools.
% (see Sections~\ref{class-file-tools} and~\ref{other-tools}).

% However, some changes to the class file syntax may have significant
% benefits.  


\subsection{Type annotations for stack elements}

The Java class file stores the types of stack elements, at entry to each
basic block in the control flow graph (which may differ from the Java
blocks in the program).  As a result, byte code verification need not
perform type inference for local variables and stack elements.  The class
file format should eventually similarly store the annotations on those
types, to aid in annotation verification of class files.

We will defer this change until after a maintenance review fixes some other
issues in StackMap.  The fixes will make this enhancement much easier to
implement.


\subsection{Reducing class file size via use of the constant pool\label{constant-pool}}

Annotations could be stored in the constant pool, and use constant pool
references from the annotation points.  That would reduce class file size,
especially if an annotation is used in many places in the same class, as is
more likely once JSR 308 support is in place.
% with the annotations enabled by JSR 308 and those proposed in
% JSR 305.

\subsection{Optional/default fields in annotations\label{optional-default-fields}}

In order to reduce the size of the class file, some fields may be omitted
from the \<.class> file, in which case any access of them returns their
default value.


\section{Access to method bodies in annotation processing API\label{out-of-scope-annotation-processing-api}}

A type-checking compiler plug-in (annotation processor) must process
annotations (including those in method bodies), and it also must check
each use of a variable/method whose declared type is annotated.  For
example, if a variable \code{x} is declared as \code{@NonNull Object x;},
then every assignment to \code{x} must be checked, because any assignment
\code{x = null;} would be illegal.

The JSR 269 annotation processing API
% specifies that the \code{process}
% method is invoked on class, field, and method annotations.  It
does not
process annotations on local variables, as it is not designed to access
method bodies.  This limitation makes JSR 269 insufficient for creating a
type-checking compiler plug-in.

An annotation and source code processing API for JSR 308 annotations could
take advantage of JSR 198's Java Model.  The Java Model defines a parsed
view into the contents of a source file that is intended for construction
of IDE extensions that are portable across multiple IDEs --- precisely the
situation with compiler plug-ins.  JSR 308 may be shipped without defining
this API, but defining this API may be desirable in the future (say, in a
later version of Java), particularly after more experience is gained with
type annotation processors.

In the absence of such an API, a type checker can be written using
compiler-specific APIs.  This is how the Checker Framework
(\myurl{http://types.cs.washington.edu/checker-framework/}) works.


\bibliographystyle{alpha}
\bibliography{bibstring-unabbrev,types,ernst,invariants,generals,alias,concurrency}

\end{document}

% LocalWords:  fontsize getAnnotations getAnnotation Desugar desugar EE wiki
% LocalWords:  AContainer getDeclaredAnnotations AnnotatedElement classfile
% LocalWords:  instanceof subannotation DefaultAnnotation YAML ValueMember
% LocalWords:  SingleElementAnnotation Nonnull signedness Nonnegative typedefs
% LocalWords:  DefaultQualifier MyStringMap javap StackMap ernst
