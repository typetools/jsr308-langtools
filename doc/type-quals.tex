\documentclass[10pt]{article}
\usepackage{fancyvrb}
\usepackage{fullpage}
\usepackage{relsize}
\usepackage{url}
\usepackage{xspace}

%HEVEA \footerfalse    % Disable hevea advertisement in footer

\begin{document}

% \newcommand{\code}[1]{{\smaller \tt #1}}
\newcommand{\code}[1]{\ifmmode{\mbox{\smaller\ttfamily{#1}}}\else{\smaller\ttfamily #1}\fi}
%% Hevea version omits "\smaller"
%HEVEA \renewcommand{\code}[1]{\ifmmode{\mbox{\ttfamily{#1}}}\else{\ttfamily #1}\fi}
%BEGIN LATEX
\RecustomVerbatimEnvironment{Verbatim}{Verbatim}{fontsize=\smaller}
%END LATEX

% Add line between figure and text
\makeatletter
\def\topfigrule{\kern3\p@ \hrule \kern -3.4\p@} % the \hrule is .4pt high
\def\botfigrule{\kern-3\p@ \hrule \kern 2.6\p@} % the \hrule is .4pt high
\def\dblfigrule{\kern3\p@ \hrule \kern -3.4\p@} % the \hrule is .4pt high
\makeatother

\bibliographystyle{alpha}

\title{Custom type qualifiers via annotations on Java types}
\author{Matthew M. Papi and Michael D. Ernst \\
{\ttfamily mernst@cs.washington.edu} \\
\today
}
\maketitle


A separate document, ``Type annotations (JSR 308)'', presents a proposal for
extending the syntax of Java (and the class file format) to permit
annotations on any use of a type in Java.  By contrast, the current Java
standard only permits annotations on declarations, which is less general
and restricts how annotations can be utilized.

The ``Type annotations'' document specifies the syntax of Java
annotations, but not their semantics.  The semantics of each annotation is
defined by its author, who also creates plug-ins to check and enforce those
semantics.  Those plug-ins can operate at compile time and load time ---
making guarantees about run-time behavior requires load-time checking, but
compile-time checking is useful as an early check and is more convenient
for programmers.

This document explores implementation strategies for plug-ins that check
annotations.  These strategies are equally valid for Java's original
annotations and for the new extended annotations.  Both compile-time and
load-time processing are outside the scope of the annotations proposal, but
are presented here as an example of how annotations on types might be used.
The processors described here do not require any further changes to Java
beyond those described in the ``Type annotations'' document.



\section{Compile-Time Processing}
\label{sec:compile-time}

A compiler plug-in that enforces annotation semantics (for type qualifiers
or any other use) can be written using JSR-269 (the pluggable annotation
processing API) and the Tree API, which were defined for just such a
purpose.  The Tree API allows an application to obtain and traverse (but
not modify) an instance of a program's abstract syntax tree.  JSR-269
provides a set of classes for working with annotations, as well as an
interface to the compiler so that annotation processing code can be run,
and errors issued, at compile-time.  These interfaces must be slightly
extended to accommodate the new locations for annotations.
%  described in section~\ref{sec:syntax}.

A plug-in developer can create a
class that extends JSR-269's \code{AbstractProcessor} and provide an
implementation of its \code{process} method. During compilation, the
compiler calls \code{process} when it encounters annotations, passing
these annotations as arguments. The \code{process} method's
implementation can use the Tree API to obtain and analyze the program
elements surrounding the annotation. If the plug-in discovers a
violation of a type qualifier's semantics, the \code{process} method
can use the JSR-269 \code{Messager} class to report the violation as
an error or warning.

A framework for writing compiler plug-ins is outside the scope of this
proposal.  After obtaining some experience with compiler plug-ins, we or
others may propose support for writing them, perhaps
in the spirit of \cite{ChinMM2005}.

In keeping with Java's separate compilation model, the Tree API gives
access to the AST for all \code{.java} files that are being compiled; a
plug-in has access only to signatures for code that is read from \code{.class}
files.  A whole-program analysis might need to use a separate
infrastructure, and is outside the scope of this proposal.


\section{Load-Time Processing}
\label{sec:load-time}

Whenever compile-time checking is used for a given annotation, then
load-time checking is required as well.  This is exactly how Java already
works.  The byte-code verifier is the key type-checker for Java, which
enforces the guarantees of the type system and prevents certain run-time
errors.  By contrast, the Java compiler (e.g., \code{javac}) is a
programmer convenience that provides early warnings but cannot make
guarantees on its own.

Load-time checking is necessary for two general reasons:  unchecked
bytecodes and changes in environment.
%
The first reason is that bytecodes can be produced from any source,
including being directly constructed, being produced by a buggy compiler,
or being produced by a malicious party.  To guarantee type safety, the JVM
must check the bytecodes.  Similarly, the compiler reads and trusts the
annotations on signatures of called methods; a type safety guarantee
requires the entire program to be checked at load time.
%
The second reason is that the environment --- such as called methods ---
may change between the time that a class is compiled and the time that it
is loaded/run, for instance if a library it calls is changed and
recompiled.  Again, the only way to guarantee type safety is to check the
entire program at once, at load time.

A type-checking plug-in that enforces annotation (e.g., type qualifier) semantics at load time can be
invoked from the \code{premain} method, and can use a third-party bytecode
library for convenient analysis of the bytecodes of loaded classes. In
Java, the \code{premain} method can be used to perform bytecode
analysis (and instrumentation) just before a program's \code{main}
method is invoked.

A plug-in developer can create a class that declares a \code{premain}
method.  The virtual machine invokes \code{premain}, passing an
instance of \code{java.lang.instrument.Instrumentation} as an
argument. This instance allows the developer to obtain an array of
loaded classes, which the bytecode library can convert to its own
representation. This representation can then be used to traverse the
bytecodes of each class's methods and perform verification of a type
qualifier's semantics --- all before the virtual machine invokes the
program's \code{main} method.

As with compile-time processing, 
a framework for writing load-time plug-ins is outside the scope of this
proposal.  After obtaining some experience with JVM plug-ins, we or
others may propose support for writing them, perhaps
in the spirit of \cite{Fong2004}.


\bibliography{bibstring-unabbrev,types,ernst,invariants,generals,alias}

\end{document}

% LocalWords:  fontsize pt NonNull NonNullDefault getNeighbors const Readonly
% LocalWords:  VM toolchain JLS checkcast ops op pc
% LocalWords:  LocalVariableTable AbstractProcessor Messager premain javac YY
% LocalWords:  YYYY localizable unaliased typestate readwrite mpapi annotatable
% LocalWords:  ASM Bracha instanceof java Jens Palsberg Hevea
