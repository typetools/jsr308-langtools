An advantage of prefix is that it maps nicely to English, in which
adjectives (annotations) precede nouns (types).


\subsection{Syntax of array annotations}
\label{sec:array-syntax}

As discussed in Section~\ref{sec:type-annotation-use-cases}, it is
desirable to be able to independently annotate both the element type and
each distinct level of a nested array.
Forbidding annotations on arbitrary levels of an array would simplify the
annotation system, though it would reduce expressiveness.
The syntax of array types is rather different than the syntax of other
Java types, so the annotation syntax must also be different.
(Arrays are not very commonly used in Java, so perhaps the syntax need not
be perfect, so long as it is usable and expressive.)

This section presents several proposals for array syntax.

For the array syntax, there are two choices to make.  First, where should 
array annotations appear?
\begin{itemize}
  \item within the brackets (\code{[]}) of the array syntax:
    \code{@NonNull Document[@Readonly]}
  \item outside the brackets in prefix notation (before the brackets):
    \code{@NonNull Document @Readonly []}
  \item outside the brackets in postfix notation (after the brackets):
    \code{@NonNull Document[] @Readonly} \\ or, if postfix syntax is adopted
    for all type annotations: \code{Document @NonNull [] @Readonly}
\end{itemize}
\noindent
Second, should an annotation on a set of brackets refer to the array or the
elements?  Taking the within-the-brackets syntax as an example, would
\code{@NonNull Document[@Readonly]} mean that the array is \code{@Readonly}
and contains \code{@NonNull} elements, or that the array is \code{@NonNull} and
contains \code{@Readonly} elements?
(For the fully postfix syntax, the second question is moot:  the only sensible
choice is for the annotation on the brackets to refer to the array, not the
elements.)

Here are some (mutually incompatible) principles that an ideal syntax would satisfy.

\newcommand{\arrayAddLevelsPrinciple}{P1}
\newcommand{\multipleVariablesPrinciple}{P2}
\newcommand{\wholeTypePrinciple}{P3}
\begin{description}
  \item[\arrayAddLevelsPrinciple]
    Adding array levels should not change the meaning of existing
    annotations.  For example, it would be confusing to have a syntax in
    which
\begin{Verbatim}
  @A List<@B Object>        // @A refers to List
  @A List<@B Object>[@C]    // @A refers to array, @C refers to List
\end{Verbatim}
Another way of stating this principle is that a textual subpart of a declaration
should describe a type that is part of the declared type.  Stating a
subpart of the given type should not require shuffling around the
annotations.

  \item[\multipleVariablesPrinciple]
    When two variables appear in a variable declaration, the annotations
    should mean the same thing for both variables.
    In Java, arrays can be declared with brackets after the type, after
    the identifier, or both, as in \code{String[] my2dArray[];}.  For
    example, \code{arr1} should have the same annotations
    as the elements of \code{arr2}:
\begin{Verbatim}
  @A T[@B] arr1, arr2[@C];
\end{Verbatim}
\noindent
Likewise, the \code{T}s should have the same annotations for \code{v3} and \code{arr4}:
\begin{Verbatim}
  @A T v3, arr4[@B][@C];
\end{Verbatim}
\noindent
And, these three declarations should mean the same thing:
\begin{Verbatim}
  @A T[@B] arr5[@C];
  @A T[@B][@C] arr6;
  @A T arr7[@B][@C];
\end{Verbatim}

  \item[\wholeTypePrinciple]
    Type annotations before a declaration refer to the full type, just as
    variable annotations (which occur in the same position --- at the very
    beginning of the declaration) refer to the entire variable.  This is
    also consistent with annotations on generics (though the syntax of
    generics and arrays is quite different in other ways), where
    \code{@NonNull List<String>} is a non-null \code{List} of possibly-null
    \code{String}s.  This principle is inconsistent with principles
    \arrayAddLevelsPrinciple{} and
    \multipleVariablesPrinciple{}, unless type annotations are
    forbidden before a declaration.

\end{description}

If an annotation on brackets refers to the array, then the syntax violates
principle \wholeTypePrinciple{}.  If an annotation on brackets refers
to the elements, then the syntax violates principles
\arrayAddLevelsPrinciple{} and \multipleVariablesPrinciple{}.


Here are several proposals for the syntax of such array annotations.

The examples below use the following variables:

\begin{description}
%BEGIN LATEX
\itemsep 0pt \parskip 0pt
%END LATEX
\item{\code{array\_of\_rodocs}}
  a mutable one-dimensional array of immutable \code{Document}s
\item{\code{roarray\_of\_docs}}
  an immutable one-dimensional array of mutable \code{Document}s
\item{\code{array\_of\_array\_of\_rodocs}}
  a mutable array, whose elements are mutable
  one-dimensional arrays of immutable \code{Document}s
\item{\code{array\_of\_roarray\_of\_docs}}
  an immutable array, whose elements are mutable
  one-dimensional arrays of mutable \code{Document}s
\item{\code{roarray\_of\_array\_of\_docs}}
  a mutable array, whose elements are immutable
  one-dimensional arrays of mutable \code{Document}s
\end{description}


\begin{enumerate}
  \item{Within brackets, refer to the array being accessed}

An annotation before the entire array type binds to the member type that it
abuts;
\code{@Readonly Document[][]} can be interpreted as
\code{(@Readonly Document)[][]}.

An annotation within brackets refers to the \emph{array} that is accessed
using those brackets.

The type of elements of \code{@A Object[@B][@C]} is \code{@A Object[@C]}.

The example variables would be declared as follows:

\begin{Verbatim}
  @Readonly Document[] array_of_rodocs;
  Document[@Readonly] roarray_of_docs;
  @Readonly Document[][] array_of_array_of_rodocs = new Document[2][12];
  Document[@Readonly][] array_of_roarray_of_docs = new Document[@Readonly 2][12];
  Document[][@Readonly] roarray_of_array_of_docs = new Document[2][@Readonly 12];
\end{Verbatim}

  \item{Within brackets, refer to the elements being accessed}

An annotation before the entire array type refers to the (reference to the)
top-level array
itself; \code{@Readonly Document[][] docs4} indicates that the array is
non-modifiable (not that the \code{Document}s in it are non-modifiable).

An annotation within brackets applies to the \emph{elements} that are
accessed using those brackets.

The type of elements of \code{@A Object[@B][@C]} is \code{@B Object[@C]}.

The example variables would be declared as follows:

\begin{Verbatim}
  Document[@Readonly] array_of_rodocs;
  @Readonly Document[] roarray_of_docs;
  Document[][@Readonly] array_of_array_of_rodocs = new Document[2][@Readonly 12];
  Document[@Readonly][] array_of_roarray_of_docs = new Document[@Readonly 2][12];
  @Readonly Document[][] roarray_of_array_of_docs = new Document[2][12];
\end{Verbatim}

  \item{Outside brackets, prefix}

The type of elements of \code{@A Object @B [] @C []} is \code{@B Object @C []}.

The example variables would be declared as follows:

\begin{Verbatim}
  Document @Readonly [] array_of_rodocs;
  @Readonly Document[] roarray_of_docs;
  @Readonly Document[][] array_of_array_of_rodocs = new Document[2][12];
  Document[] @Readonly [] array_of_roarray_of_docs = new Document[2] @Readonly [12];
  Document @Readonly [][] roarray_of_array_of_docs = new Document @Readonly [2][12];
\end{Verbatim}

  \item{Outside brackets, postfix}

The type of elements of \code{@A Object[] @B [] @C} is \code{@B Object[] @C}.

In Java, array types are constructed using postfix syntax, so postfix
annotation syntax for them is appealing.

Possible disadvantage:  Prefix notation may be more natural to Java
programmers, as it is used in other places in the Java syntax.

The example variables would be declared as follows:

\begin{Verbatim}
  Document[] @Readonly array_of_rodocs;
  @Readonly Document[] roarray_of_docs;
  Document[][] @Readonly array_of_array_of_rodocs = new Document[2][12] @Readonly;
  Document[] @Readonly [] array_of_roarray_of_docs = new Document[2] @Readonly [12];
  @Readonly Document[][] roarray_of_array_of_docs = new Document[2][12];
\end{Verbatim}

or, in a fully postfix syntax:

\begin{Verbatim}
  Document @Readonly [] array_of_rodocs;
  Document[] @Readonly roarray_of_docs;
  Document @Readonly [][] array_of_array_of_rodocs = new Document[2][12] @Readonly;
  Document[] @Readonly [] array_of_roarray_of_docs = new Document[2] @Readonly [12];
  Document[][] @Readonly roarray_of_array_of_docs = new Document[2][12];
\end{Verbatim}


%BEGIN LATEX
\newcounter{saveenumi}
\setcounter{saveenumi}{\theenumi}
%END LATEX
\end{enumerate}

The within-the-brackets syntax has problems with ambiguity, when an
explicit size is provided in a \code{new} array construction expression.
In this example the annotated element could be the array or the type \code{Y}:
\begin{Verbatim}
  new X[@ReadOnly Y.class.getMethods().length]
\end{Verbatim}
And in this example, the annotated element is the array, but the annotation could
be the marker annotation \code{@ReadOnly} with a parenthesized expression \code{(2)} or
could be the annotation \code{@ReadOnly(2)}.
\begin{Verbatim}
  new X[2][@ReadOnly (2) ]
\end{Verbatim}


It is also possible to imagine array annotations that do not require new
locations for the annotations.  The advantage of this is that there is no
new syntax.  A disadvantage is that the array level annotations are
syntactically separated from the array levels themselves, so the meaning
may not be as clear.

\begin{enumerate}
%% Problem: the numbering starts over in HTML.
%BEGIN LATEX
\setcounter{enumi}{\thesaveenumi}
%END LATEX
\item Use an array-valued annotation that gives the annotations for the
  different dimensions.  It could give the dimensions explicitly:
\begin{Verbatim}
   // dimension 1 and 2 of the array are annotated
   @ArrayAnnots({
     @ArrayAnnot(i=1, value={Readonly.class}),
     @ArrayAnnot(i=2, value={Readonly.class})
   })
   Object[][][] arr;
\end{Verbatim}
or use the order in which the annotations are given.
\begin{Verbatim}
   @ArrayAnnots({
     @ArrayAnnot({Readonly.class}),
     @ArrayAnnot({Readonly.class})
   })
   Object[][] arr2;

   @ArrayAnnots({
     @Readonly,
     @Readonly
   })
   Object[][] arr2;
\end{Verbatim}
The latter syntax is less convenient when not every level of the array is
being annotated, or when multiple annotations are put on an array.
(This document should give examples of those situations.)

\item
Use an annotation that lists the dimensions that have a property:
\begin{Verbatim}
   // In each case, the elements in the array are readonly
   // dimension 0 has no annotation
   // dimensions 1 and 2 are also readonly

   @ReadonlyDims({1,2}) @Readonly Object[][][] roa;
   @Dims({1,2}, @Readonly) @Readonly Object[][][] roa;
\end{Verbatim}
One advantage of this syntax over the one that gives an array of
annotations is that each annotation is given independently, so it will be
easier for tools to insert, delete, or conditionally display a given
annotation.  However, the array of annotations more closely mirrors the
syntax of the array declaration itself.

\end{enumerate}


%% This is the full example from Werner Dietl
% // additional annotation for arrays
% @interface ArrayAnnots {
%   ArrayAnnot[] value();
% }
% 
% // identify some data with one array dimension
% @interface ArrayAnnot {
%   // which dimension of the array
%   int i() default 0;
% 
%   // some annotation for that dimension
% 
%   // it would be nice if we could allow any
%   // annotation type, but this did not compile  
%   // java.lang.annotation.Annotation[] value();
% 
%   // if we already have an annotation supertype
%   // we could use that
%   // MyAnnotation[] value();
% 
%   // but the most general is to allow class literals
%   java.lang.Class[] value();
% }
% 
% // annotate the dimensions that should be readonly
% @interface ReadonlyDims {
%   int[] value();
% }
% 
% @interface Readonly {}
% 
% public class Array {
% 
%   // dimension 1 and 2 of the array are annotated
%   @ArrayAnnots({
%     @ArrayAnnot(i=1, value={Readonly.class}),
%     @ArrayAnnot(i=2, value={Readonly.class})
%   })
%   Object[][][] arr;
% 
%   // no dimension given so the order is used
%   @ArrayAnnots({
%     @ArrayAnnot({Readonly.class}),
%     @ArrayAnnot({Readonly.class})
%   })
%   Object[][] arr2;
% 
%   // the elements in the array are readonly
%   // dimension 0 has no annotation
%   // dimensions 1 and 2 are also readonly
%   @ReadonlyDims({1,2}) @Readonly Object[][][] roa;
% }



%%% Local Variables: 
%%% mode: latex
%%% TeX-master: t
%%% End: 
