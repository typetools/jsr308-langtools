\documentclass[10pt]{article}
\usepackage{pslatex}
\usepackage{fancyvrb}
\usepackage{fullpage}
\usepackage{graphicx}
\usepackage{relsize}
\usepackage{url}
\usepackage{xspace}
\usepackage[normalem]{ulem}
\usepackage{hevea}
%HEVEA \footerfalse    % Disable hevea advertisement in footer

\title{Type Annotations Specification (JSR 308)}
\author{Michael D. Ernst \\
{\ttfamily mernst@cs.washington.edu}
%HEVEA \\ \today
}

% At least 95% of every float page must be taken up by
% floats; there will be no page with more than 5% white space.
\def\topfraction{.95}
\def\dbltopfraction{\topfraction}
\def\floatpagefraction{\topfraction}     % default .5
\def\dblfloatpagefraction{\topfraction}  % default .5
\def\textfraction{.05}

% Left and right curly braces and backslash, in tt font
\newcommand{\ttlcb}{\texttt{\char "7B}}
\newcommand{\ttrcb}{\texttt{\char "7D}}
\newcommand{\ttbs}{\texttt{\char "5C}}

\begin{document}

\def\codesize{\smaller}
%HEVEA \def\codesize{\relax}
\newcommand{\code}[1]{\ifmmode{\mbox{\codesize\ttfamily{#1}}}\else{\codesize\ttfamily #1}\fi}
\def\<#1>{\code{#1}}
% Text size is \small
% \small and \footnotesize are the same size!!  And seem to be the same as
% the default...  Why?
\newcommand{\codesmall}[1]{\ifmmode{\mbox{\small\ttfamily{#1}}}\else{\small\ttfamily #1}\fi}
\newcommand{\codefootnotesize}[1]{\ifmmode{\mbox{\footnotesize\ttfamily{#1}}}\else{\footnotesize\ttfamily #1}\fi}
\newcommand{\myurl}[1]{{\codesize\url{#1}}}
%HEVEA \def\myurl{\url}

%NOT HEVEA \makeatletter
%NOT HEVEA \newcommand{\sout}[1]{\@print{<STRIKE>}{}#1\@print{</STRIKE>}}
%NOT HEVEA \makeatother

%HEVEA \def\strut{\relax}

%BEGIN LATEX
\RecustomVerbatimEnvironment{Verbatim}{Verbatim}{fontsize=\smaller}
%END LATEX

% Add line between figure and text
\makeatletter
\def\topfigrule{\kern3\p@ \hrule \kern -3.4\p@} % the \hrule is .4pt high
\def\botfigrule{\kern-3\p@ \hrule \kern 2.6\p@} % the \hrule is .4pt high
\def\dblfigrule{\kern3\p@ \hrule \kern -3.4\p@} % the \hrule is .4pt high
\makeatother

\hyphenation{Pecht-chan-ski}

\urldef{\jlsEightEightFour}\url{http://docs.oracle.com/javase/specs/jls/se7/html/jls-8.html#jls-8.8.4}
\urldef{\jlsFifteenEightThree}\url{http://docs.oracle.com/javase/specs/jls/se7/html/jls-15.html#jls-15.8.3}
\urldef{\jlsFifteenEightFour}\url{http://docs.oracle.com/javase/specs/jls/se7/html/jls-15.html#jls-15.8.4}
\urldef{\jlsFifteenNine}\url{http://docs.oracle.com/javase/specs/jls/se7/html/jls-15.html#jls-15.9}



\maketitle

The JSR 308 webpage is \myurl{http://types.cs.washington.edu/jsr308/}.
It contains the
\ahref{http://types.cs.washington.edu/jsr308/specification/java-annotation-design.html}{latest version}
% latest version
of this document, along
with other information such as a FAQ, the reference implementation, and
sample annotation processors.

%HEVEA This document is available in PDF format at \myurl{http://types.cs.washington.edu/jsr308/specification/java-annotation-design.pdf}.


\tableofcontents

\section{Introduction\label{intro}}

JSR 308 extends Java's annotation system~\cite{JSR175} so that
annotations may appear on any use of a type.
(By contrast, Java~SE~7 permits annotations only on
declarations; JSR 308 is
backward-compatible and continues to permit those annotations.)
Such a generalization removes
% removed "arbitrary" in response to Brian Goetz comment.
limitations of Java's annotation
system, and it enables new uses of annotations.

This document specifies the \emph{syntax} of extended Java annotations, but
it makes no commitment as to their \emph{semantics}.  As with Java's
existing annotations~\cite{JSR175}, the semantics is dependent on annotation processors
(compiler plug-ins), and not every annotation is necessarily sensible in
every location where it is syntactically permitted to appear.
This proposal is compatible with existing annotations,
such as those specified in JSR 250, ``Common Annotations for the Java
Platform''~\cite{JSR250}, and proposed annotations, such as those to be
specified in (the now-defunct) JSR 305, ``Annotations for Software Defect
Detection''~\cite{JSR305}.
% (For a comparison of JSR 305 and JSR 308, see Section~\ref{semantics}%
% %BEGIN LATEX
% , page~\pageref{semantics}%
% %END LATEX
% .)

This proposal does not change the compile-time, load-time, or run-time
semantics of Java.  It does not change the abilities of Java annotation
processors as defined in JSR 269~\cite{JSR269}.
% or of instrumentation agents~\cite{JSR163}.
The proposal merely makes annotations more general --- and thus more useful
for their current purposes, and also usable for new purposes that are
compatible with the original vision for annotations~\cite{JSR175}.

This document has two parts:  a 
% short
normative part and a 
% longer
non-normative part.
The normative part specifies the changes to
the Java language syntax (Section~\ref{syntax}) and
the class file format (Section~\ref{class-file}).
%
The non-normative part
motivates annotations on types by presenting one
possible use, type qualifiers
(Appendix~\ref{type-qualifiers}), along with related work.
It lists tools that
must be updated to accommodate the Java and class file
modifications (Appendix~\ref{mods-discussion}), along with some
requirements on those tool changes.

A JSR, or Java Specification Request, is a proposed specification for some
aspect of the Java platform --- the Java language, virtual machine,
libraries, etc.  For more details, see the Java Community Process FAQ at
\myurl{http://jcp.org/en/introduction/faq}.


A FAQ (Frequently Asked Questions) document complements this specification;
see \myurl{http://types.cs.washington.edu/jsr308/jsr308-faq.html}.



\section{Java language syntax extensions\label{syntax}}

\newcommand{\preverbnegspace}{\vspace{-5pt}}

\subsection{Source locations for annotations on types\label{type-annotation-locations}}

In Java SE 7, annotations can be written only on method formal parameters
and the declarations of packages, classes, methods, fields, and local variables.
JSR 308 extends Java to allow annotations on any use of a type, and
on type parameter declarations.  JSR 308 does not extend Java to allow
annotations on type names that are not type uses, such as the type names
that appear in \<import> statements, class literals, static member accesses,
and annotation uses (see Section~\ref{type-names}).
% (whether explicit or implicit, as in the case of method receivers)
JSR 308 uses a simple prefix syntax for type annotations, as illustrated in
the below examples, with a special case
for receiver types (item \#\ref{type-annotation-locations-receiver} below).
In addition to supporting type annotations, JSR 308 also makes one
extension to declaration annotations (item
\#\ref{type-annotation-locations-type-parameter-declaration} below).
% that is necessitated by
% non-orthogonality in the Java grammar.

\begin{enumerate}
\item
  A type annotation appears before the type's simple name, as in \code{@NonNull String}
  or \code{java.lang.@NonNull String}.
%
  Here are examples:
\begin{itemize}
\item for generic type arguments to parameterized classes:
\preverbnegspace
\begin{Verbatim}
  Map<@NonNull String, @NonEmpty List<@Readonly Document>> files;
\end{Verbatim}

\item for generic type arguments in a generic method or constructor invocation:
\preverbnegspace
\begin{Verbatim}
  o.<@NonNull String>m("...");
\end{Verbatim}

\item for type parameter bounds, including wildcard bounds:
\preverbnegspace
\begin{Verbatim}
  class Folder<F extends @Existing File> { ... }
  Collection<? super @Existing File>
\end{Verbatim}

\item for class inheritance:
\preverbnegspace
\begin{Verbatim}
  class UnmodifiableList<T> implements @Readonly List<@Readonly T> { ... }
\end{Verbatim}

\item for \code{throws} clauses:
\preverbnegspace
\begin{Verbatim}
  void monitorTemperature() throws @Critical TemperatureException { ... }
\end{Verbatim}

\item for constructor invocation results (that is, for object creation):

\preverbnegspace
\begin{Verbatim}
  new @Interned MyObject()
  new @NonEmpty @Readonly List<String>(myNonEmptyStringSet)
  myVar.new @Tainted NestedClass()
\end{Verbatim}
\preverbnegspace

For generic constructors (JLS \ahref{\jlsEightEightFour}{\S8.8.4}), the
annotation follows the explicit type arguments (JLS
\ahref{\jlsFifteenNine}{\S15.9}):

\preverbnegspace
\begin{Verbatim}
  new <String> @Interned MyObject()
\end{Verbatim}

\item for nested types:
\preverbnegspace
\begin{Verbatim}
  Map.@NonNull Entry
\end{Verbatim}

\item for casts:
\preverbnegspace
\begin{Verbatim}
  myString = (@NonNull String) myObject;
\end{Verbatim}
\preverbnegspace
It is not permitted to omit the Java type, as in
\code{myString = (@NonNull) myObject;}.
% ; see Sections~\ref{type-annotation-use-cases}.
% and~\ref{implicit-java-types-in-casts}.

\item for type tests:
\preverbnegspace
\begin{Verbatim}
  boolean isNonNull = myString instanceof @NonNull String;
\end{Verbatim}
\preverbnegspace
It is not permitted to omit the Java type, as in
\code{myString instanceof @NonNull}.
% ; see Sections~\ref{type-annotation-use-cases}.
% and~\ref{implicit-java-types-in-casts}.

\item for method and constructor references, including their receiver,
  receiver type arguments, and type arguments to the method or constructor
  itself:
\preverbnegspace
\begin{Verbatim}
  @Vernal Date::getDay
  List<@English String>::size
  Arrays::<@NonNegative Integer>sort
\end{Verbatim}


  %% We don't permit this any more:
  % If the nested type is a static type, then both the nested and
  % the outer type can be annotated individually, as in \code{@ReadOnly
  %   Outer.@ReadWrite StaticNested}.
% %
%   Here is an example:
% \begin{itemize}
% \item for static class member access:
% \preverbnegspace
% \begin{Verbatim}
%   Map.@NonNull Entry mapEntry;
% \end{Verbatim}
% \end{itemize}

\end{itemize}

% Don't mention that casts and type tests have no run-time effect; that is
% not appropriate here, where we are focusing on the syntax rather than
% introducing the semantics.


% This is separate from regular type arguments, because there is a separate
% grammar rule for it in the JLS.
\item
  An annotation on a wildcard type argument appears before the wildcard,
  as in \code{MyClass<@Immutable ?\ extends Comparable<MyClass>>}.

%   Here are examples:
% \begin{itemize}
% \item for wildcard type arguments:
% \preverbnegspace
% \begin{Verbatim}
%   Collection<? super @Existing File>
% \end{Verbatim}
% \end{itemize}

% \item
%   An annotation on a disjunctive type (as in a multi-\<catch> clause) or a
%   conjunctive type (as in an \<extends> clause) appears
%   at the beginning of the disjunct/conjunct, before an extra \<|> or \<&>
%   character.  It is treated as syntactic sugar for
%   writing the type annotation on each disjunct/conjunct.
%   % In addition, of course, to any explicitly-present annotations on the
%   % disjuncts.
% 
%   Here are examples:
% \begin{itemize}
% \item TODO
% \end{itemize}

\item
  The annotation on a given array level
  prefixes the brackets that introduce that level.  To declare
  a non-empty array of English-language strings, write \code{@English String
    @NonEmpty []}.
  The varargs syntax ``\code{...}'' is treated analogously to array brackets
  and may also be prefixed by an annotation.
%
  Here are examples:
% \begin{itemize}
% \item for arrays:
\preverbnegspace
\begin{Verbatim}
  @Readonly Document [][] docs1 = new @Readonly Document [2][12]; // array of arrays of read-only documents
  Document @Readonly [][] docs2 = new Document @Readonly [2][12]; // read-only array of arrays of documents
  Document[] @Readonly [] docs3 = new Document[2] @Readonly [12]; // array of read-only arrays of documents
\end{Verbatim}

\preverbnegspace
This syntax permits independent annotations for each distinct level of
array, and for the elements.
% ; see Section~\ref{array-syntax} for alternative syntaxes.
% \end{itemize}

\item
A type annotation is permitted in front of a constructor
declaration, where declaration annotations are already permitted.  In that
location, a type annotation is treated as applying to the constructed
object (which is different than the receiver, if any, of the constructor).
Generic type parameter annotations are not
possible on the constructor result.
%
  Here is an example:
% \begin{itemize}
% \item for constructor declaration return types:
\preverbnegspace
% TODO:  Use a more realistic example.
\begin{Verbatim}
  class Invocation {
    @Immutable Invocation() { ... }
    ...
  }
\end{Verbatim}
\preverbnegspace

(Note that the result of a constructor is different from the receiver.  The
receiver only exists for inner class constructors, as in
\code{theReceiver.new InnerClass()}.  The receiver is the containing
object, and in the body of the constructor it is referred to as
\code{\emph{Supertype}.this}.  In the constructor body, the result is referred to as
\code{this}.  In any non-constructor, the receiver (if any) is referred to
as \code{this}.)
% \end{itemize}

Outer class annotations for a constructor result must be identical to
those on the receiver, so they can be inferred from the annotations on the
receiver.

\item
  \label{type-annotation-locations-receiver}%
  It is permitted to explicitly declare the method receiver as the first
  formal parameter.
  (Each non-static method has an implicit formal parameter, \code{this}, which is
  called the \emph{receiver}.)
  Then, no special
  syntax is needed for expressing a receiver type annotation:  it is simply an
  annotation on the type of the \code{this} formal parameter.

  Only the first formal parameter may be
  named \code{this}, and such a formal parameter is permitted only on an instance
  method.  It is forbidden on a static method or a lambda expression.  The type
  of the \code{this} formal parameter must be the same as the class that contains
  the method and may include type arguments if that class has any.
  A receiver \code{this} formal parameter is also permitted on a
  constructor of an inner class, in which case its type is that of the
  class that contains the inner class.

  %% Examples for constructors:

  % public class OuterWithInnerStatic {
  %   class Inner { }
  %   static class Nested {
  %     class Inner2 { }
  %   }
  % }

  % class OtherClass {
  %   void m(OuterWithInnerStatic o, OuterWithInnerStatic.Nested n) {
  %     o.new Inner();
  %     n.new Inner2();
  %   }
  % }

  In a method in an inner type,
  the receiver type can be
  written as (for example) either \code{Inner} or as \code{Outer.Inner},
  and in the latter case annotations on both parts are possible, as in
  \code{@ReadOnly Outer.@Mutable Inner}.
  In a constructor in an inner type, the receiver
  has a name such as \code{Outer.this}.

  The optional receiver parameter has no effect on execution --- it only
  serves as a place to write annotations on the receiver.  The compiler
  generates the same bytecodes, and reflection returns the same results
  regarding number of method formal parameters, whether or not the optional
  receiver parameter is present.


  As an example, here are the standard definitions for \code{toString} and \code{equals}:
\preverbnegspace
\begin{Verbatim}
  class MyClass {
    ...
    public String toString() { ... }
    public boolean equals(Object other) { ... }
  }
\end{Verbatim}

\noindent
It is equivalent to write instead

\preverbnegspace
\begin{Verbatim}
  class MyClass {
    ...
    public String toString(MyClass this) { ... }
    public boolean equals(MyClass this, Object other) { ... }
  }
\end{Verbatim}

\noindent
The only purpose of writing the receiver explicitly is to make it
possible to annotate the receiver's type:

\preverbnegspace
\begin{Verbatim}
  class MyClass {
    ...
    public String toString(@Readonly MyClass this) { ... }
    public boolean equals(@Readonly MyClass this, @Readonly Object other) { ... }
  }
\end{Verbatim}

  An anonymous class has no name for its innermost class, so it is not
  possible to annotate the receiver of its constructor or methods.
  If you need to do so, re-write the
  anonymous class into a named class and then use a receiver parameter.

  A method in an inner class (or a constructor in an
  inner class) has multiple receivers:  one for
  each containing class.  For a method, the receivers are named \<this>,
  \<Outer.this>, etc.  Only the innermost receiver may be written as a
  formal parameter.  This restriction does not limit expressiveness, because
  the annotations on the other receivers can be inferred
  from the annotations on the fully-qualified type of the innermost
  receiver, just as for outer class annotations on a constructor result.

  For example, consider method \<innerMethod>:

\preverbnegspace
\begin{Verbatim}
class Outer {
  class Middle {
    class Inner {
      void innerMethod(@A Outer.@B Middle.@C Inner this) { ... }
    }
  }
}
\end{Verbatim}

\noindent
It has three receivers (JLS \ahref{\jlsFifteenNine}{\S15.9}).
One is referred to as \<this> and has type \<@C Inner>.
One is referred to as \<Middle.this> and has type \<@B Middle>.
One is referred to as \<Outer.this> and has type \<@A Outer>.

A constructor has no receiver, unless the constructor is in an inner class.
Within an inner class constructor, the receiver has a name such as
\<Outer.this>, and \<this> refers to the result.  (By contrast, within a
method \<this> refers to the receiver; JLS \ahref{\jlsFifteenEightThree}
notes the two distinct uses for \<this>.)
An example of an inner constructor with both a result and a receiver
annotation is:

\preverbnegspace
\begin{Verbatim}
class Outer {
  class Inner {
    @Result Inner(@Receiver Outer Outer.this, boolean b) { }
  }
}
\end{Verbatim}


It is not permitted to write a declaration annotation on the
receiver (\code{this}) formal parameter declaration.
% (for one thing, there is nowhere to store it in the class file)
More specifically, it is a compile-time error if an annotation is
written on the \code{this} formal parameter but is not applicable to type
uses (is not meta-annotated with
\code{@Target(\ttlcb ElementType.TYPE\_USE, ...\ttrcb)}).


\item
\label{type-annotation-locations-type-parameter-declaration}%
It is permitted to write an annotation on a type parameter
declaration.  This appears before the declared name, as in
\code{class MyClass<@Immutable T> \ttlcb\ ...\ \ttrcb} or
\code{interface WonderfulList<@Reified E> \ttlcb\ ...\ \ttrcb}.
% Niko Matsakis gave another example in email of 
% Wed, 25 Feb 2009 11:57:37 +0100
This is a declaration
annotation, not a type annotation, but its omission in JSR
175~\cite{JSR175} was an oversight, and JSR 308 is taking the opportunity to correct it.

\end{enumerate}


% Section~\ref{type-annotation-syntax-examples} contains examples of the annotation syntax.


% \subsubsection{Examples of annotation syntax\label{type-annotation-syntax-examples}}
% 
% This section gives examples of the type annotation syntax.
% % This list is not necessarily exhaustive (but if you notice something
% % missing, let us know so that we can add it).
% % Section~\ref{type-annotation-use-cases} motivates annotating these
% % locations by giving the meaning of annotations that need to be applied to
% % these locations.
% 
% % Bill Pugh suggests using @Foo and @Bar here, so avoid giving people the
% % impression that we are proposing specific annotations.  But I think that
% % makes the discussion unnecessarily abstract and thus harder to follow.
% 
% % This list should perhaps be given in the same order as figure 1 (and
% % maybe make that explicit in both places).


\subsubsection{Implicit type uses are not annotatable\label{implicit-type-uses}}

There are places that Java implicitly uses a type but does not express it
in the source code.  For example, consider an array type such as
\<String[]>.  This array takes an \<int> as an index, and the array can be
thought of as mapping \<int>s to \<String>s.  However, there is no way to
write a type annotation on the \<int> type of the index.  Analogously to
arrays, there is no convenient way to annotate the \<int> argument type
that indexes a given \<List>.  These restrictions make it more difficult,
and syntactically clumsy, to write an annotation processor that ensures
proper use of array or list indices.


\subsubsection{Not all type names are annotatable\label{type-names}}

The Java language uses type names in three different ways:  in type
definitions/declarations; in type uses; and in other contexts that are not
a type declaration or use.  JSR 308 permits annotations on type uses (and
also on type parameter declarations).  JSR 308 does not support annotations
on type names that syntactically look like, but are not, type uses.
In the JLS grammar, type uses have a non-terminal name ending in
\emph{Type}.
% Say something about other uses of type names?  I would need to see the
% new grammar first.

Here are examples of such type names that are not annotatable.

\paragraph{Annotation uses}
An annotation use cannot itself be annotated.  (An annotation \emph{declaration}
can be annotated by a so-called meta-annotation.)
For instance, in

\preverbnegspace
\begin{Verbatim}
  @MyAnnotation Object x;
\end{Verbatim}
\preverbnegspace

\noindent
there is no way to annotate the use of \<MyAnnotation>.

\paragraph{Class literals}
It is not permitted to annotate the type name in a class literal, as in

\preverbnegspace
\begin{Verbatim}
  @Even int.class           // illegal!
  int @NonEmpty [].class    // illegal!
\end{Verbatim}
\preverbnegspace

\noindent
This type name refers to a class, not a type.  The expression evaluates to a
\<Class>, which does not reflect the type of the annotation, so there is no
point in being able to write the annotation, which would have no effect.

\paragraph{Import statement}
It is not permitted to annotate the type name in an \<import> statement.

\preverbnegspace
\begin{Verbatim}
  import java.util.@NotAllowed Date;      // illegal!
  import @IllegalSyntax java.util.Date;   // illegal syntax
\end{Verbatim}
\preverbnegspace

\noindent
This use of a type name is not a type use and is more properly viewed as
a scoping mechanism.

\paragraph{Static member accesses}

Static member accesses are preceded by a type name, but that type name may
not be annotated:

\preverbnegspace
\begin{Verbatim}
  @Illegal Outer.StaticNestedClass      // illegal!
  @Illegal Outer.staticField            // illegal!
\end{Verbatim}
\preverbnegspace

The type name in front of a static member access is a scoping mechanism,
not a use of a type --- there's nothing of type \<Outer> in the above
examples.  Furthermore, since there is only one instance of any static
member, the static member cannot be affected by an annotation on the name of
the class being used to access it.  Affecting the type of that single
thing, depending on the annotation on the class name being used to access
it, feels unnatural.

A static member access may itself be a type use, in which case the used
type may be annotated by annotating the last component of the static member
access, which is the simple name of the type being used.

\preverbnegspace
\begin{Verbatim}
  Outer.@Legal StaticNestedClass x = ...;     // legal
\end{Verbatim}
\preverbnegspace

\paragraph{Super references}

In a field access expression such as \<MyType.super.fieldname> or a method
reference expression like \<MyType.super::methodname>, \<MyType> is a
scoping mechanism rather than a use of a type.


\subsection{Java language grammar changes\label{grammar}}

This section
gives changes to the \ahref{http://docs.oracle.com/javase/specs/jls/se7/html/jls-18.html}{grammar} of the Java
language~\cite[ch.~18]{GoslingJSBB2012}, organized in the same way as
the rules of Section~\ref{type-annotation-locations}.
Additions are \underline{underlined}.
% , and deletions \sout{crossed out}.

Infelicities in the Java grammar make this section somewhat long.
Some improvements are
possible (for instance, by slightly refactoring the Java grammar), but this
version attempts to minimize changes to existing grammar productions.


% The original grammar appears at
% http://docs.oracle.com/javase/specs/jls/se7/html/jls-18.html
% (older: http://java.sun.com/docs/books/jls/third_edition/html/syntax.html)

% The BNF for the spec does not have to be the same as the BNF for the
% implementation.  An example is that the BNF does not address the ">>" issue
% in generics, even though that's important for an implementation.


\newcommand{\term}[1]{\code{#1}}         % terminal
\newcommand{\nt}[1]{\emph{#1}}           % non-terminal
\newcommand{\opt}[1]{\emph{[}#1\emph{]}} % optional (zero or one)
\newcommand{\multi}[1]{\emph{\{}#1\emph{\}}} % multiple (any number)
\newcommand{\alt}[1]{\emph{(}#1\emph{)}} % alternatives
\newcommand{\altor}[1]{\ensuremath{\mathit{|}}} % alternatives separator
\newcommand{\new}[1]{\underline{#1}}     % additions

\newcommand{\basictypename}{\alt{ \term{byte} \altor{} \term{short} \altor{} \term{char} \altor{} \term{int} \altor{} \term{long} \altor{} \term{float} \altor{} \term{double} \altor{} \term{boolean} }}

\begin{enumerate}
\item
Any \nt{Type} may be prefixed by \opt{\nt{Annotations}}:

\begin{tabbing}
\qquad \= \kill
\nt{Type}: \\
\qquad \new{\opt{\nt{Annotations}} \nt{UnannType}} \\
 \\
\new{\nt{UnannType}}: \\
\qquad    \basictypename{} \multi{\new{\opt{\nt{Annotations}}} \term{[]}} \\
\qquad    \nt{\new{Unann}ReferenceType} \multi{\new{\opt{\nt{Annotations}}} \term{[]}} \\
 \\
\nt{MethodOrFieldDecl}: \\
\qquad       \nt{\new{Unann}Type} \nt{Identifier} \nt{MethodOrFieldRest} \\
 \\
\nt{InterfaceMethodOrFieldDecl}: \\
\qquad       \nt{\new{Unann}Type} \nt{Identifier} \nt{InterfaceMethodOrFieldRest} \\
\\
\nt{Resource}: \\
\qquad        \multi{\nt{VariableModifier}} \nt{\new{Unann}ReferenceType} \nt{VariableDeclaratorId} \term{=} \nt{Expression} \\
\\
\nt{ForVarControl}: \\
\qquad       \multi{\nt{VariableModifier}} \nt{\new{Unann}Type} \nt{VariableDeclaratorId} \nt{ForVarControlRest} \\
\\
\new{\nt{QualifiedTypeList}}: \\
\qquad        \new{\opt{\nt{Annotations}}} \nt{UnannReferenceType} \multi{ \term{,}   \new{\opt{\nt{Annotations}}} \nt{UnannReferenceType} }
\end{tabbing}

% \nt{QualifiedTypeList} was \nt{QualifiedIdentifierList} in the Java SE 7
% Edition of the JLS\@.

% Optionally, two uses of
%   ``\opt{\nt{Annotations}} \nt{Type}''
% can be simplified to
%   ``\nt{Type}''.
% (in ForVarControl, FormalParameterDecls).
% 
% OLD:
%     ForVarControl
%             [final] [Annotations] Type Identifier ForVarControlRest
% NEW:
%     ForVarControl
%             [final] Type Identifier ForVarControlRest
% 
% OLD:
%     FormalParameterDecls:
%             [final] [Annotations] Type FormalParameterDeclsRest]
% NEW:
%     FormalParameterDecls:
%             [final] Type FormalParameterDeclsRest]

% \item
Annotations are permitted on the simple name in a static nested class.

\begin{tabbing}
\qquad \= \kill
\nt{ReferenceType}: \\
\qquad \new{\opt{\nt{Annotations}} \nt{UnannReferenceType}} \\
 \\
\new{\nt{UnannReferenceType}}: \\
\qquad            \nt{Identifier} \opt{\nt{TypeArguments}} \multi{ . \new{\opt{\nt{Annotations}}} \nt{Identifier} \opt{\nt{TypeArguments}}}
\end{tabbing}

\item
Annotations may appear on the wildcard in any type argument.
A wildcard is the declaration of an anonymous (unnamed) type parameter.

\begin{tabbing}
\qquad \= \kill
\nt{TypeArgument}: \\
\qquad            \new{\opt{\nt{Annotations}}} \term{?}\ \opt{\nt{(} \term{extends} \altor{} \term{super} \nt{)} \nt{Type}}
\end{tabbing}


% \item
% Annotations may appear at the beginning of a disjunctive type, before a
% vertical bar, or at the beginning of a conjunctive type, before an ampersand:
% 
% \todo{give the grammar rules: catch, extends, others?}


\item
To permit annotations on levels of an array (in declarations, not
constructors), change ``\multi{\term{[]}}''  to
``\multi{\new{\opt{\nt{Annotations}}} \term{[]}}''.
(This was abstracted out as
``\nt{BracketsOpt}'' in the 2nd edition of the JLS~\cite{GoslingJSB2000}.)

\begin{tabbing}
\qquad \= \kill
\nt{VariableDeclaratorRest}: \\
\qquad        \multi{\new{\opt{\nt{Annotations}}} \term{[]}} \opt{\term{=} \nt{VariableInitializer}} \\
 \\
\nt{VariableDeclaratorId}: \\
\qquad       \nt{Identifier} \multi{\new{\opt{\nt{Annotations}}} \term{[]}} \\
 \\
\nt{MethodDeclaratorRest}: \\
\qquad            \nt{FormalParameters} \multi{\new{\opt{\nt{Annotations}}} \term{[]}} \opt{\term{throws} \nt{QualifiedTypeList}} \alt{ \nt{Block} \altor{} \term{;} } \\
 \\
\nt{InterfaceMethodDeclaratorRest}: \\
\qquad            \nt{FormalParameters} \multi{\new{\opt{\nt{Annotations}}} \term{[]}} \opt{\term{throws} \nt{QualifiedTypeList}}   \term{;} \\
 \\
\nt{ConstantDeclaratorRest}: \\
\qquad        \multi{\new{\opt{\nt{Annotations}}} \term{[]}} \opt{\term{=} \nt{VariableInitializer}}
\end{tabbing}

Annotations may also appear on varargs (\code{...}):

\begin{tabbing}
\qquad \= \kill
\nt{FormalParameterDecls}: \\
\qquad       \multi{\nt{VariableModifier}} \nt{\new{Unann}Type} \nt{FormalParameterDeclsRest} \\
 \\
\nt{FormalParameterDeclsRest}: \\
\qquad       \nt{VariableDeclaratorId} \opt{\term{,} \nt{FormalParameterDecls}} \\
\qquad        \new{\opt{\nt{Annotations}}} \term{...} \nt{VariableDeclaratorId}
\end{tabbing}


\item
No grammar changes are required to support type annotations on constructor
result types.


\item
The receiver may be expressed explicitly.
Then, annotations appear on the receiver type in the normal way.
If the receiver is not expressed explicitly, then its type cannot be annotated.
% The receiver \<this> is implicitly \<final>.

\begin{tabbing}
\qquad \= \kill
\nt{FormalParameters}: \\
\qquad        \term{(} \opt{\nt{FormalParameter\new{OrReceiver}Decls}} \term{)} \\
\\
\new{\nt{FormalParameterOrReceiverDecls}}: \\
\qquad        \nt{Type} \opt{\nt{Identifier} \term{.}} \term{this} \opt{\term{,} \nt{FormalParameterDecls}} \\
\qquad        \nt{FormalParameterDecls}
\end{tabbing}


\item
An annotation may appear on the type parameter
declaration in a class or method declaration.

\begin{tabbing}
\qquad \= \kill
\nt{TypeParameter}: \\
\qquad            \new{\opt{\nt{Annotations}}} \nt{Identifier} \opt{\term{extends} \nt{Bound}}
\end{tabbing}

\end{enumerate}




\subsubsection{Syntax of array annotations\label{array-syntax}}

As discussed in Section~\ref{type-annotation-use-cases}, it is
desirable to be able to independently annotate both the base type and
each distinct level of a nested array.
Forbidding annotations on arbitrary levels of an array would simplify the
syntax, but it would reduce expressiveness to an unacceptable degree.
The syntax of array annotations follows the same general prefix rule as
other annotations --- it looks slightly different because the syntax
of array types is different than the syntax of other Java types.
(Arrays are less commonly used than generics, so even if you don't like the
array syntax, it need not bother you in most cases.)

Most programmers read the Java type \code{String[][]} as
``array of arrays of Strings''.  Analogously, the Java expression
\code{new String[2][5]} is ``new length-2 array of length-5 array of Strings''.
After \code{a = new String[2][5]}, \code{a} is an array with
2 elements, and \code{a[1]} is a 5-element array.

In other words, the order in which a programmer reads an array type or
expression is left-to-right for the
brackets, \emph{then} left-to-right for the base type.

\begin{Verbatim}
  type:                       String             []            []

  order of reading:  2------------->  1 ------------------------>
\end{Verbatim}

% Even Josh Bloch got this wrong when I gave it to him as a Java Puzzler of
% my own (he thought you would read from right to left), so it's
% understandable if you didn't realize that this is how Java arrays have
% always worked.


To more fully describe the 2x5 array, a programmer could use the
type ``length-2 array of length-5 array of Strings'':

\begin{Verbatim}
  type:                       String  @Length(2) [] @Length(5) []

  order of reading:  2------------->  1 ------------------------>
\end{Verbatim}

The prefix notation is natural, because the type is read in exactly
the same order as any Java array type.  As another example, to express
``non-null array of length-10 arrays of English Strings'' a programmer
would write

\begin{Verbatim}
  type:              @English String  @NonNull [] @Length(10) []

  order of reading:  2------------->  1 ----------------------->
\end{Verbatim}


% The prefix notation doesn't really work for people who read the original
% declaration by starting with the base type and then working
% \emph{right-to-left}, as ``String array array''.


An important property of this syntax is that, in two declarations that
differ only in the number of array levels, the annotations mean the same
thing.  For example, \code{var1} has
the same annotations as the elements of \code{arr2}:

\preverbnegspace
\begin{Verbatim}
  @NonNull String var1;
  @NonNull String[] arr2;
\end{Verbatim}

\noindent
because in each case \code{@NonNull} refers to the String, not the array.
This consistency is especially important since the two variables may appear in a single
declaration:

\preverbnegspace
\begin{Verbatim}
  @NonNull String var1, arr2[];
\end{Verbatim}

%% Probably don't bring this up.  It should be possible to infer it anyway.
% In this (deprecated but legal) syntax, in each component you read
% left-to-right, like this:
% 
%     @English String @NonNull [] [] []  arr1,   arr2 @Length(10) [] [] [];
%     3-------------> 2--------------->               1------------------>


%     For example, \code{arr1} should have the same annotations
%     as the elements of \code{arr2}:
% \preverbnegspace
% \begin{Verbatim}
%   @A T[@B] arr1, arr2[@C];
% \end{Verbatim}

%     By contrast, it would be confusing to have a syntax in which
% \preverbnegspace
% \begin{Verbatim}
%   @A List<@B Object>          // @A refers to List
%   @A List<@B Object> @C []    // @A refers to array, @C refers to List
% \end{Verbatim}

A potential criticism is that a type annotation at the very beginning of a
declaration does not refer to the full type, even though declaration
annotations (which also occur at the beginning of
the declaration) do refer to the entire variable.  As an example, in
\code{@NonNull String[] arr2;} the variable \code{arr2} is not non-null.
This is actually a criticism of the fact that in a Java declaration such as
\code{String[] arr2;}, the top-level type constructor does not appear on
the far left.
An annotation on the whole type (the array) should
appear on the syntax that indicates the array --- that is, on the brackets.


Other array syntaxes can be imagined, but they are less consistent with
Java syntax and therefore harder to read and write.
Examples include
making annotations at the beginning of the type refer to the whole type,
using a postfix syntax rather than a prefix syntax, and postfix syntax
within angle brackets as for generics.

% The IGJ type system~\cite{ZibinPAAKE2007} has been implemented using both a
% generics-like syntax and also the JSR 308 annotation syntax.  In a case
% study, a programmer preferred the JSR 308 syntax to the generics-like
% syntax~\cite{ZibinPAAKE2007}.


%% For a discussion of the two major alternatives for array syntax, see my
%% email of July 13, 2010 titled "Consistency of array type and expression
%% annotations" (and subsequent and previous email).





\subsection{Target meta-annotations for type annotations\label{target-meta-annotation}}

Java uses the \code{@Target} meta-annotation as a machine-checked way of
expressing where an annotation is intended to appear.  The
\code{ElementType} enum classifies the places an annotation may appear in a
Java program.  JSR 308 adds two new constants to the \code{ElementType}
enum.

\begin{itemize}
\item
\code{ElementType.TYPE\_PARAMETER} stands for a type parameter --- that is,
the declaration of a type variable.  Examples are in generic class
declarations \code{class MyClass<T> \ttlcb...\ttrcb}, generic method
declarations \code{<T> foo(...)\ \ttlcb...\ttrcb}, and wildcards
\code{List<?>}, which declare an anonymous type variable.

\item
\code{ElementType.TYPE\_USE} stands for all uses of types, plus two
special cases.

\begin{enumerate}
\item 
A type annotation (one meta-annotated with
\code{@Target(ElementType.TYPE\_USE)}) is permitted to be written anywhere 
\code{@Target(ElementType.TYPE)} or \code{@Target(ElementType.TYPE\_PARAMETER)} would permit --- that is, on
a class, interface, or enum declaration, or on a type parameter declaration.  Strictly speaking,
these are declaration sites, not uses of a type.  However, it is convenient
to write a type annotation at a type declaration, as shorthand for applying
it at all uses.  For example, \code{@Interned class MyClass \ttlcb\ ...\ \ttrcb}
could indicate that all uses of \code{MyClass} are interned, even though
for other classes some instances may be interned and other instances not
interned.

\item
A type annotation may appear before a constructor, in which case it
represents the object that the constructor is creating (which is not the
same as the receiver of the constructor).
\end{enumerate}

\end{itemize}

\code{ElementType.TYPE\_PARAMETER} and \code{ElementType.TYPE\_USE} are
distinct from the existing \code{ElementType.TYPE} enum element
of Java SE 7, which indicates that an annotation may appear on a type
declaration (a class, interface, or enum declaration).  The program elements
denoted by \code{ElementType.TYPE\_PARAMETER},
\code{ElementType.TYPE\_USE}, and \code{ElementType.TYPE} are disjoint
(except for the special cases about class and type parameter declarations,
noted above).


As an example, in this declaration:

% Does not need to be "@Target({ElementType.TYPE_USE})" because when the
% annotation argument is an array, it is permitted to supply just one
% element to indicate a singleton array.
\preverbnegspace
\begin{Verbatim}
  @Target(ElementType.TYPE_USE)
  public @interface NonNull { ... }
\end{Verbatim}

\noindent
the \code{@Target(ElementType.TYPE\_USE)} meta-annotation indicates that
\code{@NonNull} may appear on any use of a type.

If an annotation is not meta-annotated with \<@Target> (which would be poor
style!), then the compiler treats the annotation as if it is
meta-annotated with all of the \code{ElementType} enum constants
that appear in Java 7: \<ANNOTATION\_TYPE>, \<CONSTRUCTOR>, \<FIELD>,
\<LOCAL\_VARIABLE>, \<METHOD>, \<PACKAGE>, \<PARAMETER>, and \<TYPE>.
% , but not \<TYPE_PARAMETER>, \<TYPE_USE>, or any other constants that may
% be added to \<ElementType> in the future.

% If an annotation appears at a source location where it could be interpreted
% as applying to either a declaration or a type, as in \<@A int x;>, then
% the compiler applies the annotation to every target that is consistent with
% its meta-annotation.  The order of annotations is not used to disambiguate.
As in Java SE 7, the compiler issues an error if a programmer places an
annotation in a location not permitted by its \<@Target> meta-annotation.  (The
compiler issues the error even if no annotation processor is being run.)
% The Target Javadocs state
%   If such a meta-annotation is present, the compiler will enforce the
%   specified usage restriction.
% (See http://download.oracle.com/javase/6/docs/api/java/lang/annotation/Target.html .)

A programmer could write a \code{@Target} meta-annotation indicating that an
annotation is \emph{both} a type annotation and a declaration annotation.
We have not found an example where such a meta-annotation is desirable;
although it is legal, it is considered bad style.
For example, suppose that a programmer wrote 

\preverbnegspace
\begin{Verbatim}
  @Target({ ElementType.TYPE_USE, ElementType.METHOD })
  public @interface SillyAnnotation { ... }
\end{Verbatim}

\noindent
If \<@SillyAnnotation> were written before a non-\<void> method, then
the annotation would apply to both the return type and the
method declaration.  The annotation appears twice in the AST
during annotation processing, and it appears twice in the classfile.
% In other words, a Java 7 and a Java 8 compiler would
% produce classfiles with different numbers of annotations.
A tool that reads a classfile and writes Java-like output, such as Javadoc
or a decompiler, must take care not to write an annotation twice in the
decompiled code, as in ``\code{@SillyAnnotation @SillyAnnotation int m()}\ldots''.
% Is this rule a good idea??  Or should we use this rule:
%   It is legal to write \<@A void foo() \ttlcb\ ...\ \ttrcb>, where \<@A> is
%   meta-annotated with \<@Target(\ttlcb ElementType.TYPE\_USE, ElementType.METHOD\ttrcb)>
%    or \<@A> has no \<@Target> meta-annotation; in either case, the \<@A>
%   annotation is applied only to the method declaration.  The code construct
%   would have been illegal if \<@A> were a type annotation.
If you write \<@SillyAnnotation> on a \<void>-returning method, then
it applies only to the declaration, because it is not written in a type
annotation position.



% \subsubsection{Discussion:  Disambiguating type and declaration annotations\label{disambiguating}}
% 
% An annotation written before a method declaration annotates either the return type,
% or the method declaration.  There is never any ambiguity regarding the
% programmer intention:  in that location, a type annotation annotates the
% return type, and a declaration annotation annotates the method itself.  The
% \code{@Target} meta-annotation indicates whether an annotation is a type
% annotation.  Field declarations are treated similarly.
% 
% 
% % This reuse of the annotation location means that there is no need for new
% % syntax for annotations on types in declarations.
% 
% Suppose that we have these annotation declarations:
% \begin{Verbatim}
%   @Target(ElementType.TYPE_USE)
%   @interface NonNegative { }
% 
%   @Target(ElementType.METHOD)
%   @interface Override { }
% 
%   @Target(ElementType.FIELD)
%   @interface GuardedBy { ... }
% \end{Verbatim}
% Then, in
% \begin{Verbatim}
%   @Override
%   @NonNegative int getHeight() { ... }
% \end{Verbatim}
% \code{@Override} applies to the method and \code{@NonNegative} applies
% to the return type.
% Furthermore, in these two field declarations
% \begin{Verbatim}
%   @NonNegative int balance;
%   @GuardedBy("myLock") long lastAccessedTime;
% \end{Verbatim}
% the annotation \code{@NonNegative} applies to the field type \code{int},
% not to the whole variable declaration nor to the variable \code{balance}
% itself.  The annotation \code{@GuardedBy("myLock")} applies to the
% field \code{lastAccessedTime}.
% 
% Here are a few facts that follow from the specification.  For brevity, we
% use ``type annotation'' as shorthand for ``an annotation that is
% meta-annotated with \code{@Target(ElementType.TYPE\_USE)}''.
% \begin{itemize}
% \item
%   A type annotation need not also be meta-annotated with the targets
%   \code{ElementType.METHOD} or
%   \code{ElementType.FIELD} in order to be applied to a method
%   return type or a field type --- and it generally should \emph{not}
%   contain such a \code{@Target} meta-annotation.
% \item
%   A type annotation may not appear before a void method (a method with no
%   return type).  This also follows from the fact that \code{void} is not a
%   type.
% \item
%   It is legal to write \<@A void foo() \ttlcb\ ...\ \ttrcb>, where \<@A> is
%   meta-annotated with \<@Target(\ttlcb ElementType.TYPE\_USE, ElementType.METHOD\ttrcb)>
%    or \<@A> has no \<@Target> meta-annotation; in either case, the \<@A>
%   annotation is applied only to the method declaration.  The code construct
%   would have been illegal if \<@A> were a type annotation.
% \item
%   A type annotation may appear before a constructor, in which case it
%   represents the object that the constructor is creating (which is not the
%   same as the receiver of the constructor).
% \item
%   As with any other non-static method, a type annotation may
%   appear on the receiver of an inner class constructor.
% \item
%   A type annotation may not appear on a package declaration or an import
%   statement, which do
%   not contain a use of a type.
% \end{itemize}
% 
% 
% A programmer could write a \code{@Target} meta-annotation indicating that an
% annotation is \emph{both} a type annotation and a declaration annotation.  In
% such a case, the annotation would apply to both the return type and the
% method declaration, and it would exist twice in the class file.  We have
% not found an example where that is desirable; although it is legal, it is
% considered bad style.
% 
% 
% When an annotation has no \code{@Target} meta-annotation (which is bad
% style!), it is treated as if it applies to all locations.
% %
% For example, if the \code{@Foo} annotation definition lacks a
% \code{@Target} meta-annotation, then in this code:
% 
% \preverbnegspace
% \begin{Verbatim}
%   @Foo int m() { return 0; }
% \end{Verbatim}
% 
% \noindent
% the \code{@Foo} annotation applies to both the \code{m} method, and the
% \code{int} return type.  (If it is not intended as a type annotation, it will
% be ignored by any type annotation processor, so it does no harm on the
% \code{int} return type.)  The \code{@Foo} annotation appears twice in the AST
% during annotation processing, and it appears twice in the classfile.
% % In other words, a Java 7 and a Java 8 compiler would
% % produce classfiles with different numbers of annotations.
% A tool that reads a classfile and writes Java-like output, such as Javadoc
% or a decompiler, must take care not to write an annotation twice in the
% decompiled code, as in ``\code{@Foo @Foo int m()}\ldots''.
% This requirement does
% not add extra work when constructing the tool, since the tool already has to handle
% the case where the \<@Target> annotation explicitly included both
% \<ElementType.METHOD> and \<ElementType.TYPE\_USE>.
% 
% 
% In summary:  for certain syntactic locations, which target (Java construct)
% is being annotated depends on the annotation.
% There is no ambiguity for the compiler:
% the compiler applies the annotation to every
% target that is consistent with its meta-annotation (see
% Section~\ref{target-meta-annotation}).
% In practice, programmers
% have no difficulty in understanding where a given annotation applies.
% 
% 
% 
% % The Target Javadocs state
% %   If such a meta-annotation is present, the compiler will enforce the
% %   specified usage restriction.
% % (See http://download.oracle.com/javase/7/docs/api/java/lang/annotation/Target.html .)
% 
% 
% 
% 
% 
% % Explain the difference between
% %  * meta-annotating with TYPE.
% %  * not annotating.
% %  * annotating with everything (except ANNOTATION).
% % Explain backward-compatibility with anything that's not meta-annotated.
% 
% 
% % We want to avoid text like this, in the FindBugs manual:
% % "When this annotation is applied to a method it applies to the method return value."
% % The problem is that it's not machine-checked.
% % (That text is for CheckForNull, but such text does not appear for other
% % annotations such as NonNull and Nullable.)
% 
% 
% 
% %% IMPLEMENTATION NOTE:
% % As noted in Section~\ref{target-meta-annotation},
% % the order of annotations is not used to disambiguate;
% % \code{@NonNull @Override Dimension getSize()} has the same
% % meaning but is poor style.  (An annotation processor could warn about this
% % order, just as tools warn about modifiers --- such as \code{final static}
% % --- that are not written in the order recommended by Oracle.)
% % % The recommended order is:
% % %   public protected private abstract static final transient volatile synchronized native strictfp interface
% 
% 
% %% ALTERNATIVE DESIGN
% % Different syntaxes are possible, in which declaration and type annotations
% % appear in different locations in the grammar.  Such a syntax is warranted
% % in two circumstances.  (1) If programmers find it confusing to determine
% % which construct an annotation applies to, and the alternate syntax is
% % clearer overall.  (2) If there exist annotations that can apply equally
% % well to both methods and return value types (or to both types and variable
% % declarations, etc.), and that have different semantics in the two
% % situations.  Both circumstances are poor style.  (The only example of an
% % annotation that can go on both locations that we have found is an
% % annotation that is itself actually a collection of other annotations,
% % such as "@MultipleAnnotations" --- which is not even possible to write
% % currently, since an annotation cannot take an arbitrary other annotation
% % as an argument.)  Another argument for the alternate syntax is
% % that the grammar should be maximally flexible to permit unforeseen future
% % uses of such annotations.  This must be traded off against readability and
% % usability for important known common cases.
% 
% 
% 
% % The person who defines the annotation (and writes
% % its annotation processor) decides whether an annotation that appears before
% % the return value applies to the method declaration or to the return type.
% 
% 
% 
% % It is also convenient to use annotations to describe the effect of invoking
% % a method, such as whether it may block, allocate memory, etc.  Such
% % annotations could be placed on either the return value (as is current practice)
% % or the receiver; this semantic decision is up to the person who defines the
% % annotation.



\section{Class file format extensions\label{class-file}}

\newcommand{\RuntimeVisibleAnnotations}{\code{Runtime\-Visible\-Annotations}\xspace}
\newcommand{\RuntimeInvisibleAnnotations}{\code{Runtime\-Invisible\-Annotations}\xspace}
\newcommand{\RuntimeInOrVisibleAnnotations}{\code{Runtime}\-[\code{In}]\code{visible\-Annotations}\xspace}
\newcommand{\RuntimeVisibleParameterAnnotations}{\code{Runtime\-Visible\-Parameter\-Annotations}\xspace}
\newcommand{\RuntimeInvisibleParameterAnnotations}{\code{Runtime\-Invisible\-Parameter\-Annotations}\xspace}
\newcommand{\RuntimeInOrVisibleParameterAnnotations}{\code{Runtime}\-[\code{In}]\code{visible\-Parameter\-Annotations}\xspace}
\newcommand{\RuntimeInOrVisibleAnnotationsOrParameterAnnotations}{\code{Runtime}\-[\code{In}]\code{visible}[\code{Parameter}]\code{Annotations}\xspace}
\newcommand{\RuntimeVisibleTypeAnnotations}{\code{Runtime\-Visible\-Type\-Annotations}\xspace}
\newcommand{\RuntimeInvisibleTypeAnnotations}{\code{Runtime\-Invisible\-Type\-Annotations}\xspace}
\newcommand{\RuntimeInOrVisibleTypeAnnotations}{\code{Runtime}\-[\code{In}]\code{visible\-Type\-Annotations}\xspace}
\newcommand{\RuntimeInOrVisibleTypeParameterAnnotations}{\code{Runtime}\-[\code{In}]\code{visible\-Type\-Parameter\-Annotations}\xspace}

% Don't use this in section headings, as that throws Hevea off.
\newcommand{\extendedannotation}{\code{type\-\_anno\-ta\-tion}\xspace}


This section defines how to store type annotations in a Java class file.
It also defines how to store local variable annotations, which are
permitted in Java SE 7 source code but are discarded by the compiler.

The class file stores only annotations that are explicitly written in the
Java source file.  Annotations that are inferred are not stored in the
class file.  One location where Java infers types is the ``diamond'' used
in class instance creation expressions (JLS \ahref{\jlsFifteenNine}{\S15.9}).
Lambda leads to other examples.  The inference of type annotations can
depend on the annotation processor,
% example: flow-sensitive type refinement can affect effective types
and it would be undesirable for the
annotations in the classfile to depend on whether a given annotation
processor is being run.


\paragraph{Why store type annotations in the class file?}

The class file format represents the type of every variable and expression
in a Java class, including all temporaries and values stored on the stack.
(Sometimes the representation is explicit, such as via the
\code{StackMapTable} attribute, and sometimes it is implicit.)
Since JSR 308 permits annotations to be added to a type, the class file
format should be updated to continue to represent the full, annotated type
of each expression.


More pragmatically, Java annotations must be stored in the class file for two reasons.

First, annotated \emph{signatures} (public members) must be available to
tools that read class files.  For example, a type-checking compiler
plug-in~\cite{JSR269,PapiACPE2008} needs to read annotations when compiling
a client of the class file.  The Checker Framework
(\myurl{http://types.cs.washington.edu/checker-framework/}) is
one way to create such plug-ins.
% (an annotation processor)
% , as described in Section~\ref{compile-time})

Second, annotated method \emph{bodies} must be present to permit checking
the class file against the annotations.  This is necessary to give
confidence in an entire program, since its parts (class files) may
originate from any source.  Otherwise, it would be necessary to simply
trust annotated classes of unknown provenance~\cite{BurdyHP2007}.
% so load-time checking (by a plug-in to the bytecode verifier)
% % as described in Section~\ref{load-time})
% is required to ensure
% that the guarantees of the source language are enforced at run time.
% (A third \emph{non}-goal is providing reflective access within method
% bodies.)
% ; the key use is class file tools

% There are a wide variety of class file tools that might want the
% annotations.  One specific reason for class file retention is the ability
% to pass information to code generation, a JIT, etc., when a compiler is
% compiling to Java as a target language.


\paragraph{How Java SE 7 stores annotations in the class file}

% Class files already store annotations in the form of
% ``attributes''.

In Java SE 7, an annotation is stored in the class file in an
\emph{attribute}~\cite{JSR175,LindholmYBB2012}.  An attribute 
associates data with a program element (a
method's bytecodes, for instance, are stored in a \code{Code}
attribute of the method). The \RuntimeVisibleParameterAnnotations attribute
stores formal parameter annotations that are accessible at runtime using
reflection, and the
\RuntimeInvisibleParameterAnnotations attribute stores formal parameter annotations
that are not accessible at runtime.
\RuntimeVisibleAnnotations and 
\RuntimeInvisibleAnnotations are analogous, but for annotations on fields,
methods, and classes.

These attributes contain arrays of
\code{annotation} structure elements, which in turn contain arrays of
\code{element\_value} pairs.  The \code{element\_value} pairs store
the names and values of an annotation's arguments.

Annotations on a field are stored as attributes of the field's 
\code{field\_info} structure~\cite[\S 4.6]{LindholmYBB2012}.
Annotations on a method are stored as attributes of the method's
\code{method\_info} structure~\cite[\S 4.7]{LindholmYBB2012}.
Annotations on a class are stored as attributes of the class's
\code{attributes} structure~\cite[\S 4.2]{LindholmYBB2012}.

Generic type information is stored in a different way in the class file, in
a signature attribute.  Its details are not germane to the current
discussion.


\paragraph{Changes in JSR 308}
JSR 308 introduces two new attributes:
\RuntimeVisibleTypeAnnotations and
\RuntimeInvisibleTypeAnnotations. These attributes are
structurally identical to the \RuntimeVisibleAnnotations and
\RuntimeInvisibleAnnotations attributes described above with one
exception: rather than an array of \code{annotation} elements,
\RuntimeVisibleTypeAnnotations and
\RuntimeInvisibleTypeAnnotations contain an array of
\extendedannotation elements, which are described in
Section~\ref{class-file:ext}.

\preverbnegspace
\begin{Verbatim}
Runtime[In]VisibleTypeAnnotations_attribute {
   u2 attribute_name_index;  // "Runtime[In]VisibleTypeAnnotation"
   u4 attribute_length;
   u2 num_annotations;
   type_annotation annotations[num_annotations];
}
\end{Verbatim}

% There's no point in separating the type annotations into two attributes,
% one for class/field/method and one for formal parameters, so we don't do
% so.

A type annotation is stored in a \RuntimeInOrVisibleTypeAnnotations attribute
on the smallest enclosing class, field, method, or \<Code> structure.
Type annotations in the body of an initializer appear with the code that
performs the initialization, not on the field that is being initialized.
Type annotations in the body of instance initializer appear on on all
initial constructors, and type annotations in the body of a class
initializer appear on the \<clinit> symbol.


A type annotation written on a declaration's type (e.g., on a field type or
on a method return type)
appears in the \RuntimeInOrVisibleTypeAnnotations attribute of that declaration.
JSR 308 does not add a \RuntimeInOrVisibleTypeParameterAnnotations
attribute; annotations on type parameters are stored with the method or class.

% TODO: No RuntimeInvisibleParameterTypeAnnotations (similar to the
% RuntimeInvisibleParameterAnnotations) are introduced.


In the \<annotations> array, annotations that target instructions must be
sorted in increasing order of bytecode offset.  Other annotations can be in
any order, and may be interleaved with instruction annotations.
Annotations that target instructions are those in
Figure~\ref{tbl:target_types} with \<target\_type> $\ge$ 0x40; see below
for details.

No changes are made to the StackMapTable that stores the types of elements
on the stack.
% Doing so would eliminates the need for pluggable type-checking on byte code
% to perform type qualifier inference for local variables and stack elements,
% much as byte code verification does not need to perform type inference for
% local variables and stack elements.
% StackMapTable is the attribute that defines these.  They are differential
% and rather tricky, but we'll just copy that format.  There is no public
% API, just a defined classfile format.
No changes are made to the Exception table.
Local variable type annotations are stored in the class file, see
Section~\ref{class-file:ext:ri:localvar}.

The VM does not have to validate all the constraints in
Section~\ref{class-file:ext:target_info}, such as that offsets point to the
right type of instruction.  The VM does not do format checking (JVMS 4.8)
on attributes used solely for reflection.

The JSR 308 changes apply to class file version 52 and higher.


\paragraph{Backward compatibility}
For backward compatibility, JSR 308 uses new attributes for storing
the type annotations.  In other words, JSR 308 merely reserves the
names of a few new attributes and specifies their layout.
JVMs ignore unknown
attributes.
JSR 308 does
not alter the way that existing annotations on classes, methods, method formal parameters,
and fields are stored in the class file.
% , except those on parameterized types or arrays.
JSR 308 mandates no changes to the processing of existing annotation locations;
in the absence of other changes to the class file format,
class files generated from programs that use no new
annotations will be identical to those generated by a standard Java SE 7
% (that is, pre-extended-annotations)
compiler.
Furthermore, the bytecode array will be identical between two programs that
differ only in their annotations.
Attributes have no effect on the bytecode array, because they exist outside
it; however, they can represent properties of it by referring to the
bytecode (including referring to specific instructions, or bytecode offsets).


\subsection{The \code{type\_annotation} structure\label{class-file:ext}}

The \extendedannotation structure has the following format:

\preverbnegspace
\begin{Verbatim}
type_annotation {
    // New fields in JSR 308:
    u1 target_type;    // the type of the targeted program element, see Section 3.2
    union {
        type_parameter_target;
        supertype_target;
        type_parameter_bound_target;
        empty_target;
        method_formal_parameter_target;
        throws_target;
        localvar_target;
        catch_target;
        offset_target;
        type_argument_target;
        method_reference_target;
    } target_info;     // identifies the targeted program element, see Section 3.3
    type_path target_path; // identifies targeted type in a compound type (array, generic, etc.), see Section 3.4
    // Original fields from "annotation" structure:
    u2 type_index;
    u2 num_element_value_pairs;
    {
        u2 element_name_index;
        element_value value;
    } element_value_pairs[num_element_value_pairs];
}
\end{Verbatim}

% TODO: These are no longer used; is that what I want?
%%        field_target;
%%        typeindex_target;


We first briefly recap
the three fields of \code{annotation}~\cite[\S 4.8.15]{LindholmYBB2012}.

\begin{itemize}
\item \code{type\_index} is an index into the constant pool indicating the
  annotation type for this annotation.
\item \code{num\_element\_value\_pairs} is a count of the
  \code{element\_value\_pairs} that follow.
\item Each \code{element\_value\_pairs} table entry represents a single
  element-value pair in the annotation (in the source code, these are the
  arguments to the annotation):  \code{element\_name\_index} is a
  constant pool entry for the name of the annotation type element, and
  \code{value} is the corresponding value~\cite[\S 4.8.15.1]{LindholmYBB2012}.
\end{itemize}

Compared to \code{annotation},
the \extendedannotation structure contains two additional fields.
These fields implement a discriminated (tagged) union type:
field \code{target\_type} is the tag (see
Section~\ref{class-file:ext:target_type}), and its value determines the
size and
contents of \code{target\_info} (see Section~\ref{class-file:ext:target_info}).
Section~\ref{class-file:ext:type_path} describes how the classfile
indicates a part of a compound type.


\subsection{The \code{target\_type} field:  the type of annotated element\label{class-file:ext:target_type}\label{target_type}}

The \code{target\_type} field denotes the type of program element that
the annotation targets, such as whether the annotation is on a field, a
method receiver, a cast, or some other location.
Figure~\ref{tbl:target_types} gives the value of \code{target\_type} for
every possible annotation location.


\begin{figure}[thp!]
\begin{center}
\begin{tabular}{|l|l|c|c|}
\multicolumn{4}{l}{Targets for type parameter declarations (\<ElementType.TYPE\_PARAMETER>):} \\
\hline
{\bf Annotation target} & {\bf \code{TargetType} enum constant} & {\bf \code{target\_type} value} & {\bf \code{target\_info} definition} \\ \hline
%% The canonical location for this information is
%%   langtools/src/share/classes/com/sun/tools/javac/comp/TargetType.java
%% so whenever that is updated, this table should be as well.
class type parameter & \code{CLASS\_TYPE\_PARAMETER}
& \code{0x00} & \S\ref{class-file:ext:ri:mtp} \\
method type parameter & \code{METHOD\_TYPE\_PARAMETER}
& \code{0x01} & \S\ref{class-file:ext:ri:mtp} \\
\hline
\multicolumn{4}{c}{~} \\
\multicolumn{4}{l}{Targets for type uses that may be externally visible in
  classes and members (\<ElementType.TYPE\_USE>):} \\
\hline
{\bf Annotation target} & {\bf \code{TargetType} enum constant} & {\bf \code{target\_type} value} & {\bf \code{target\_info} definition} \\ \hline
class \code{extends}/\code{implements} & \code{CLASS\_EXTENDS}
& \code{0x10} & \S\ref{class-file:ext:ri:extends} \\
class type parameter bound & \code{CLASS\_TYPE\_PARAMETER\_BOUND}
& \code{0x11} & \S\ref{class-file:ext:ri:tpbound} \\
method type parameter bound & \code{METHOD\_TYPE\_PARAMETER\_BOUND}
& \code{0x12} & \S\ref{class-file:ext:ri:tpbound} \\
field type & \code{FIELD}
& \code{0x13} & \S\ref{class-file:ext:ri:return} \\
method return type & \code{METHOD\_RETURN}
& \code{0x14} & \S\ref{class-file:ext:ri:return} \\
method receiver type & \code{METHOD\_RECEIVER}
& \code{0x15} & \S\ref{class-file:ext:ri:receiver} \\
method formal parameter type & \code{METHOD\_FORMAL\_PARAMETER}
& \code{0x16} & \S\ref{class-file:ext:ri:formal-parameters} \\
exception type in \code{throws} & \code{THROWS}
& \code{0x17} & \S\ref{class-file:ext:ri:throws} \\
\hline
\multicolumn{4}{c}{~} \\
\multicolumn{4}{l}{Targets for type uses that occur only within code blocks (\<ElementType.TYPE\_USE>):} \\
\hline
{\bf Annotation target} & {\bf \code{TargetType} enum constant} & {\bf \code{target\_type} value} & {\bf \code{target\_info} definition} \\ \hline
local variable type & \code{LOCAL\_VARIABLE}
& \code{0x40} & \S\ref{class-file:ext:ri:localvar} \\
resource variable type & \code{RESOURCE\_VARIABLE}
& \code{0x41} & \S\ref{class-file:ext:ri:localvar} \\
exception parameter type & \code{EXCEPTION\_PARAMETER}
& \code{0x42} & \S\ref{class-file:ext:ri:catch} \\
type test (\code{instanceof}) & \code{INSTANCEOF}
& \code{0x43} & \S\ref{class-file:ext:ri:instanceof} \\
object creation (\code{new}) & \code{NEW}
& \code{0x44} & \S\ref{class-file:ext:ri:new} \\
constructor reference receiver & \code{CONSTRUCTOR\_REFERENCE}
& \code{0x45} & \S\ref{class-file:ext:ri:cref-receiver} \\
method reference receiver & \code{METHOD\_REFERENCE}
& \code{0x46} & \S\ref{class-file:ext:ri:mref-receiver} \\
cast & \code{CAST}
& \code{0x47} & \S\ref{class-file:ext:ri:cast} \\
type argument in constructor call & \code{CONSTRUCTOR\_INVOCATION\_TYPE\_ARGUMENT}
& \code{0x48} & \S\ref{class-file:ext:ri:con-typearg} \\
type argument in method call & \code{METHOD\_INVOCATION\_TYPE\_ARGUMENT}
& \code{0x49} & \S\ref{class-file:ext:ri:con-typearg} \\
type argument in constructor reference & \code{CONSTRUCTOR\_REFERENCE\_TYPE\_ARGUMENT}
& \code{0x4A} & \S\ref{class-file:ext:ri:cref-typearg} \\
type argument in method reference & \code{METHOD\_REFERENCE\_TYPE\_ARGUMENT}
& \code{0x4B} & \S\ref{class-file:ext:ri:mref-typearg} \\
\hline
\end{tabular}
\end{center}
%BEGIN LATEX
\vspace{-10pt}
%END LATEX
\caption{\label{tbl:target_types}
  Values of \code{target\_type} for each possible target of JSR 308's new
  annotations.  All of these appear in \extendedannotation attributes
  in the class file.
  % By contrast, declaration annotations other than type parameter
  % declarations appear in in \code{annotation} attributes in the class file.
  % 
  The symbolic constants appear in the enums
  \code{com.sun.tools.javac.code.TargetType} and
  \code{com.sun.tools.classfile.TypeAnnotation.TargetType}.
}
\end{figure}


\subsection{The \code{target\_info} field:  identifying a program element\label{class-file:ext:target_info}}

\code{target\_info} is a structure that contains enough information to
uniquely identify the target of a given annotation.  A different
\code{target\_type} may require a different set of fields, so the structure
of the \code{target\_info} is determined by the value of
\code{target\_type}.

% , and in some cases the size of \code{target\_info} is zero ...

All indexes count from zero.

See Section~\ref{class-file:ext:type_path} for indicating a specific part
of a compound type:  a parameterized, wildcard, array, or nested type.


\newcommand{\targetinfocontents}[2]{
  When the annotation's target is #1, \code{target\_info} contains one
  \code{#2}:
%, which has the following structure:
}

\newcommand{\targetinfocontentspartial}[1]{
  When the annotation's target is #1, \code{target\_info}}

\newcommand{\attributelocation}[2]{
A \RuntimeInOrVisibleTypeAnnotations attribute #1 appears in the attributes table of a #2.}






\subsubsection{Type parameters\label{class-file:ext:ri:mtp}}

\targetinfocontents{a type parameter of a class or method}{type\_parameter\_target}

\preverbnegspace
\begin{Verbatim}
    type_parameter_target {
        u1 type_parameter_index;
    };
\end{Verbatim}

\code{type\_parameter\_index} specifies the 0-based index of the type parameter.

A \RuntimeInOrVisibleTypeAnnotations attribute containing a
\code{type\_parameter\_target} appears in the attributes table of a
\code{method\_info} structure if it targets a method type parameter,
otherwise, in the attributes table of a \code{ClassFile} structure if it
targets a class declaration type parameter.


\subsubsection{Class supertypes:  \code{extends} and \code{implements} clauses\label{class-file:ext:ri:extends}}

\targetinfocontents{a type in an \code{extends} or
\code{implements} clause}{\code{supertype\_target}}

\preverbnegspace
\begin{Verbatim}
    supertype_target {
        u2 supertype_index;
    };
\end{Verbatim}

\code{supertype\_index} is \code{-1} (\code{65535}) if the annotation is on the
superclass type.  Otherwise, \code{supertype\_index} specifies the 0-based index of the
targeted type in the
\code{interfaces} array field of the \code{ClassFile} structure;
simply the value $i$ is used if the annotation is on the $i$th
superinterface type.

\attributelocation{containing a
\code{supertype\_target}}{\code{ClassFile} structure}


\subsubsection{Type parameter bounds\label{class-file:ext:ri:tpbound}}

\targetinfocontents{a bound of a type parameter of a class or
method}{type\_parameter\_bound\_target}

\preverbnegspace
\begin{Verbatim}
    type_parameter_bound_target {
        u1 type_parameter_index;
        u1 bound_index;
    };
\end{Verbatim}

\code{type\_parameter\_index} specifies the index of the type parameter, while
\code{bound\_index} specifies the index of the bound.  Indexes start at 0.
Bound index 0 is always a class, not interface, type.
If the programmer-supplied upper bound of the type variable is an
interface, it is treated as the second bound, and the implicit first bound
is \code{java.lang.Object}.

Consider the following
example:

\preverbnegspace
\begin{Verbatim}
  <T extends @A Object & @B Comparable, U extends @C Cloneable>
\end{Verbatim}

\noindent
Here
\code{@A} has \code{type\_parameter\_index} 0 and \code{bound\_index} 0,
\code{@B} has \code{type\_parameter\_index} 0 and \code{bound\_index} 1, and
\code{@C} has \code{type\_parameter\_index} 1 and \code{bound\_index} 1.

A \RuntimeInOrVisibleTypeAnnotations attribute containing a
\code{type\_parameter\_bound\_target} appears in the attributes table of a
\code{method\_info} structure if it targets a method type parameter bound,
otherwise, in the attributes table of a \code{ClassFile} structure if it
targets a class declaration type parameter bound.


\subsubsection{Method return type, receiver, and fields\label{class-file:ext:ri:return}\label{class-file:ext:ri:receiver}}

% TODO: Specify methodorfield_target
When the type annotation's target is a method return type, a constructor
result (which is stored in the same place, using the \code{METHOD\_RETURN}
enum of \code{TargetType}), a receiver type (for an instance method or for an inner class
constructor), or a field, \code{target\_info} is empty.
More formally, it contains \code{empty\_target}:

\preverbnegspace
\begin{Verbatim}
    empty_target {
    };
\end{Verbatim}

% TODO: Make the following paragraph consistent with the rest
\attributelocation{targeting a field type}{\code{field\_info} structure}
\attributelocation{targeting a method return type, constructor result
  type, or receiver}{\code{method\_info} structure}


\subsubsection{Method formal parameters\label{class-file:ext:ri:formal-parameters}}

\targetinfocontentspartial{a method formal parameter type} contains one \code{method\_formal\_parameter\_target}, which
indicates which of the method's formal parameters is being annotated:

\preverbnegspace
\begin{Verbatim}
    method_formal_parameter_target {
        u1 method_formal_parameter_index;
    };
\end{Verbatim}

The index indicates the $i$th formal parameter type of the method
descriptor of the enclosing \code{method\_info} structure.

\attributelocation{containing a \code{method\_formal\_parameter\_target}}{\code{method\_info} structure}


\subsubsection{\code{throws} clauses\label{class-file:ext:ri:throws}}

\targetinfocontents{a type in a \code{throws} clause}{throws\_target}

\preverbnegspace
\begin{Verbatim}
    throws_target {
        u2 throws_type_index;
    };
\end{Verbatim}

\code{throws\_type\_index} specifies the index of the exception type in the
clause in \code{exception\_index\_table} of \code{Exceptions\_attribute};
simply the value $i$ denotes an annotation on the $i$th exception
type.

\attributelocation{containing a \code{throws\_target}}{\code{method\_info} structure}


\subsubsection{Local variables and resource variables\label{class-file:ext:ri:localvar}}

\targetinfocontents{a local variable or resource variable type}{localvar\_target}

\preverbnegspace
\begin{Verbatim}
    localvar_target {
        u2 table_length;
        {
            u2 start_pc;
            u2 length;
            u2 index;
        } table[table_length];
    };
\end{Verbatim}

The \code{table\_length} field specifies the number of entries in the
\code{table} array; multiple entries are necessary because a compiler is
permitted to break a single variable into multiple live ranges with different
local variable indices.
The \code{start\_pc} and \code{length} fields specify the variable's
live range in the bytecodes of the local variable's containing method
(from offset \code{start\_pc}, inclusive, to offset \code{start\_pc +
  length}, exclusive).  The
\code{index} field stores the local variable's index in that method.
These fields are similar to those of the optional
\code{LocalVariableTable} attribute~\cite[\S 4.8.12]{LindholmYBB2012}.

Storing local variable type annotations in the class file raises certain
challenges.  For example, live ranges are not isomorphic to local
variables.  Note that a local variable with no live range might not appear in
the class file; that is OK, because it is irrelevant to the program.

\attributelocation{containing a \code{localvar\_target}}{\code{Code} attribute}


\subsubsection{Exception parameters (\<catch> clauses)\label{class-file:ext:ri:catch}}

\targetinfocontentspartial{an exception parameter (the exception type
in a \<catch> statement)} indicates an offset in the exception table (which
appears in the \<exception\_table> slot of the \<Code> attribute).

\preverbnegspace
\begin{Verbatim}
    catch_target {
        u2 exception_table_index;
    };
\end{Verbatim}

\attributelocation{containing a \code{catch\_target}}{\code{Code} attribute}


\subsubsection{Type tests, object creation, and method/constructor references\label{class-file:ext:ri:instanceof}\label{class-file:ext:ri:new}\label{class-file:ext:ri:mref-receiver}\label{class-file:ext:ri:cref-receiver}}

\targetinfocontents{an \code{instanceof}
expression, a \code{new} expression, or a method or constructor reference receiver}{offset\_target}

\preverbnegspace
\begin{Verbatim}
    offset_target {
        u2 offset;
    };
\end{Verbatim}

The \code{offset} field denotes the offset (i.e., within the bytecodes
of the containing method) of the \code{instanceof} bytecode emitted for the type tests,
the \code{new} bytecode emitted for the object creation expression, or the
instruction that implements a method/constructor reference.
These annotations are attached to a single bytecode, not a bytecode range
(or ranges):  the annotation provides information about the type of a
single value, not about the behavior of a code block.


\attributelocation{containing an \code{offset\_target}}{\code{Code} attribute}

% The Java compiler must emit a \code{checkcast} bytecode for any annotated
% cast, so that there is a target for the \code{offset} field.  Ordinarily, a
% Java compiler is permitted to omit \code{checkcast} bytecodes for casts
% that are guaranteed to be no-ops.  For example, a cast from \code{String}
% to \code{@NonNull String} may be a no-op for the underlying Java type
% system (which sees a cast from
% \code{String} to \code{String}), but it may be significant to the type
% system for the qualified types
% % ; for example, it may be needed for load-time type checking.
% Retaining the \code{checkcast} bytecodes does not change
% behavior, and is unlikely to affect performance.  A JIT will eliminate the
% cast, leading to no runtime performance penalty, and the impact on class
% file size is minimal, especially when compared to the size of the
% annotations themselves.  No \code{checkcast} bytecode need be emitted for
% un-annotated type casts.


\subsubsection{Casts and type arguments to constructor/method invocation/references\label{class-file:ext:ri:cast}\label{class-file:ext:ri:con-typearg}\label{class-file:ext:ri:mref-typearg}\label{class-file:ext:ri:cref-typearg}}

\targetinfocontents{a cast or a type argument in a constructor/method
  call/reference}{type\_argument\_target}

\preverbnegspace
\begin{Verbatim}
    type_argument_target {
        u2 offset;
        u1 type_argument_index;
    };
\end{Verbatim}

The \code{offset} field denotes the offset (i.e., within the bytecodes
of the containing method) of the \code{new} bytecode emitted for
constructor call, or the
% \code{invoke\{dynamic|interface|special|static|virtual\}} bytecode emitted for
instruction that implements a
method invocation or a method reference.  These annotations
are attached to a single bytecode, not a bytecode range.

\code{type\_argument\_index} specifies the index of the type argument in the
expression.


For annotated casts, the attribute may be attached to a
\code{checkcast} bytecode, or to any other bytecode.  The rationale for
this is that the Java compiler is permitted to omit \code{checkcast}
bytecodes for casts that are guaranteed to be no-ops.  For example, a
cast from \code{String} to \code{@NonNull String} may be a no-op for the
underlying Java type system (which sees a cast from \code{String} to
\code{String}).  If the compiler omits the \code{checkcast} bytecode, the
\code{@NonNull} attribute would be attached to the (last) bytecode that
creates the target expression instead.  This approach permits code
generation for existing compilers to be unaffected.

If the compiler eliminates an annotated cast, it is required to retain the
annotations on the cast in the class file.  When a cast is removed, the
compiler may need to adjust (the locations of) the annotations, to account
for the relationship between the expression's type and the casted-to type.
Consider:

\preverbnegspace
\begin{Verbatim}
  class C<S, T> { ... }
  class D<A, B> extends C<B, A> { ... }
  ...
  ... (C<@A1 X, @A2 Y>) myD ...
\end{Verbatim}

\noindent
The compiler may leave out the upcast, but in that case it must record that
\<@A1> is attached to the second type argument of \<D>, even though it was
originally attached to the first type argument of \<C>.


\attributelocation{containing a \code{type\_argument\_target}}{\code{Code} attribute}



TODO: \code{type\_argument\_index} specifies the index of a type argument in the list
       of explicit non-wildcard type arguments passed to the
       constructor/method invocation.
       e.g. \code{x.<@Foo String>m()}
       e.g. \code{new <@Foo String> @Bar A()}



\subsection{The \code{type\_path} structure:  Identifying part of a compound type\label{class-file:ext:type_path}}

\newcommand{\tpath}[2]{\codefootnotesize{path\_length: #1; path: [#2]}}
\newcommand{\tarray}{\codefootnotesize{ARRAY(0, 0)}}
\newcommand{\tinner}{\codefootnotesize{INNER\_TYPE(1, 0)}}
\newcommand{\twildcard}{\codefootnotesize{WILDCARD(2, 0)}}
\newcommand{\ttypearg}[1]{\codefootnotesize{TYPE\_ARGUMENT(3, #1)}}

% See test/tools/javac/annotations/typeAnnotations/referenceinfos/FromSpecification.java

\begin{figure}[tp!]
\begin{center}

\code{@A Map<@B ? extends @C String, @D List<@E Object>>}
\\
\code{@I String @F [] @G [] @H []}
\\
\code{@M O1.@L O2.@K O3.@J NestedStatic}

%BEGIN LATEX
~
%END LATEX

% \resizebox{\columnwidth}{!}{\includegraphics{figures/type-tree}}

\begin{tabular}{|c|l|}
\hline
\bf{Anno.} & \bf{\code{type\_path}} \\ \hline
\code{@A}       & \tpath{0}{} \\
\code{@B}       & \tpath{1}{\ttypearg{0}} \\
\code{@C}       & \tpath{2}{\ttypearg{0}, \twildcard}  \\
\code{@D}       & \tpath{1}{\ttypearg{1}}  \\
\code{@E}       & \tpath{2}{\ttypearg{1}, \ttypearg{0}}  \\

\code{@F}       & \tpath{0}{} \\
\code{@G}       & \tpath{1}{\tarray} \\
\code{@H}       & \tpath{2}{\tarray, \tarray} \\
\code{@I}       & \tpath{3}{\tarray, \tarray, \tarray} \\

\code{@J}       & \tpath{3}{\tinner, \tinner, \tinner} \\
\code{@K}       & \tpath{2}{\tinner, \tinner} \\
\code{@L}       & \tpath{1}{\tinner} \\
\code{@M}       & \tpath{0}{} \\
\hline
\end{tabular}

%BEGIN LATEX
~ \\ ~
%END LATEX

\code{@A Map<@B Comparable<@F Object @C [] @D [] @E []>, @G List<@H Document>>}

%BEGIN LATEX
~
%END LATEX

% % The table is not top-aligned if I increase the size.  I don't know why.
% \resizebox{.4\columnwidth}{!}{\includegraphics{figures/type-tree3}}
% ~~
\begin{tabular}{|c|l|}
\hline
\bf{Anno.} & \bf{\code{type\_path}} \\ \hline
\code{@A}       & \tpath{0}{} \\
\code{@B}       & \tpath{1}{\ttypearg{0}} \\
\code{@C}       & \tpath{2}{\ttypearg{0}, \ttypearg{0}} \\
\code{@D}       & \tpath{3}{\ttypearg{0}, \ttypearg{0}, \tarray} \\
\code{@E}       & \tpath{4}{\ttypearg{0}, \ttypearg{0}, \tarray, \tarray} \\
\code{@F}       & \tpath{5}{\ttypearg{0}, \ttypearg{0}, \tarray, \tarray, \tarray} \\
\code{@G}       & \tpath{1}{\ttypearg{1}} \\
\code{@H}       & \tpath{2}{\ttypearg{1}, \ttypearg{0}} \\
\hline
\end{tabular}

%BEGIN LATEX
~ \\ ~
%END LATEX

\code{@H O1.@E O2<@F S, @G T>.@D O3.@A Nested<@B U, @C V>}

%BEGIN LATEX
~
%END LATEX

% \resizebox{.4\columnwidth}{!}{\includegraphics{figures/type-tree2}}
% ~~
\begin{tabular}{|c|l|}
\hline
\bf{Anno.} & \bf{\code{type\_path}} \\ \hline
\code{@A}       & \tpath{3}{\tinner, \tinner, \tinner} \\
\code{@B}       & \tpath{4}{\tinner, \tinner, \tinner, \ttypearg{0}} \\
\code{@C}       & \tpath{4}{\tinner, \tinner, \tinner, \ttypearg{1}} \\
\code{@D}       & \tpath{2}{\tinner, \tinner} \\
\code{@E}       & \tpath{1}{\tinner} \\
\code{@F}       & \tpath{2}{\tinner, \ttypearg{0}} \\
\code{@G}       & \tpath{2}{\tinner, \ttypearg{1}} \\
\code{@H}       & \tpath{0}{} \\
\hline
\end{tabular}

\end{center}
\caption{\label{tbl:locations}
  Example values of the \code{target\_path}
  field, which is of type \code{type\_path}.
  \newline
  The top of the figure shows examples of each of parameterized types,
  array types, and nested types.  The middle shows the interaction between
  parameterized types and array types.  The bottom shows the interaction
  between parameterized types and nested types.
  \newline
  For clarity, we use a meaningful name for each \code{type\_path\_entry},
  in addition to the raw bytes.
  % TODO: re-define the \t macros to change this.
  }
\end{figure}

The \code{type\_path} structure identifies a part of a compound type.
In a compound type (a parameterized, wildcard, array, or nested type),
there are multiple places that an annotation may appear.
For example, consider the difference among
\code{@X Map<String, Object>},
\code{Map<@X String, Object>}, and
\code{Map<String, @X Object>};
or the difference among
\code{Foo<@X ? extends T>} and
\code{Foo<? extends @X T>};
or the difference among 
\code{@X String [] []},
\code{String @X [] []}, and
\code{String [] @X []};
or the difference among 
\code{@X Outer.Inner} and
\code{Outer.@X Inner}.
The \code{type\_path} structure distinguishes among these locations.

\begin{Verbatim}
struct type_path {
  u1              path_length;
  type_path_entry path[path_length];
}

struct type_path_entry {
  u1 type_path_kind;
    // 0: annotation is deeper in this array type
    // 1: annotation is deeper in this nested type
    // 2: annotation is on the bound of this wildcard type arg
    // 3: annotation is on the i'th type arg of this parameterized type
  u1 type_argument_index;
    // 0: ignore me
    // non-0: the 1st, 2nd, etc type arg of this parameterized type
}
\end{Verbatim}

% TODO: what about having BOTH lower_bound and upper_bound for wildcard_path??

When \code{type\_path\_kind} is 0, 1, or 2, then
\code{type\_argument\_index} must be 0.

If the compound type is viewed as a tree, then 
the \code{type\_path} structure represents a path in that tree.
% , from the top down or the ``outside in''.

Figure~\ref{tbl:locations} shows some examples.

The \code{type\_path} that is stored in the class file is with respect to the full type, not the
source code representation.  In the source code, a type can be abbreviated
or partially written out, such as a nested type written without the outer
type or a raw type written without the type arguments.

In the class file, all annotations on the full type appear.  For example,
the classfile would record the type in this \<new> expression as
\<@A Outer.@B Inner>:

\preverbnegspace
\begin{Verbatim}
  @A Outer f1 = ...
  ... f1.new @B Inner() ...
\end{Verbatim}


%%%%%%%%%%%%%%%%%%%%%%%%%%%%%%%%%%%%%%%%%%%%%%%%%%%%%%%%%%%%%%%%%%%%%%%%%%%
%%% Non-normative text follows
%%%

\label{page:non-normative-start}

\appendix

\section{Example use of type annotations:  Type qualifiers\label{type-qualifiers}}

One example use of annotation on types is to create custom type qualifiers
for Java,
such as \code{@NonNull}, \code{@ReadOnly}, \code{@Interned}, or \code{@Tainted}.
Type qualifiers are modifiers on a type; a declaration that uses a
qualified type provides extra
information about the declared variable.
A designer can define new type
qualifiers using Java annotations, and can provide compiler plug-ins to check
their semantics (for instance, by issuing lint-like warnings during
compilation).
A programmer can then use these type qualifiers
throughout a program to obtain additional guarantees at compile time
about the program.

The type system defined by the type qualifiers does not change Java
semantics, nor is it used by the Java compiler or run-time system.  Rather,
it is used by the checking tool, which can be viewed as performing
type-checking on this richer type system.  (The qualified type is usually
treated as a subtype or a supertype of the unqualified type.)
As an example, a variable of type \code{Boolean} has one of the values
\code{null}, \code{TRUE}, or \code{FALSE} (more precisely, it is null or it
refers to a value that is equal to \code{TRUE} or to \code{FALSE}\@).  A
programmer can depend on this, because the Java compiler guarantees it.
Likewise, a compiler plug-in can guarantee that a variable of type
\code{@NonNull Boolean} has one of the values \code{TRUE} or \code{FALSE}
(but not \code{null}), and a programmer can depend on this.  Note that a
type qualifier such as \code{@NonNull} refers to a type, not a variable,
though JSR 308 could be used to write annotations on variables as well.

Type qualifiers can help prevent errors and make possible a variety of
program analyses.  Since they are user-defined, developers can create
and use the type qualifiers that are most appropriate for their
software.

A system for custom type qualifiers requires
extensions to Java's annotation system, described in this document; the
existing Java SE 7 annotations are inadequate.
Similarly to type qualifiers, other pluggable type
systems~\cite{Bracha2004} and similar lint-like checkers also require these
extensions to Java's annotation system.

Our key goal is to create a type qualifier system that is compatible with
the Java language, VM, and toolchain.
Previous proposals
for Java type qualifiers are incompatible with the existing Java language
and tools, are too inexpressive, or both.
The use of annotations for custom type qualifiers has a number of
benefits over new Java keywords or special comments. First, Java
already implements annotations, and Java SE 7 features a framework
for compile-time annotation processing. This allows JSR 308 to
build upon existing stable mechanisms and integrate with the Java
toolchain, and it promotes the maintainability and simplicity of the
modifications. Second, since annotations do not affect the runtime
semantics of a program, applications written with custom type
qualifiers are backward-compatible with the vanilla JDK\@.
No modifications to the virtual machine are necessary.

Four compiler plug-ins that perform type qualifier type-checking, all built
using JSR 308, are distributed at the JSR 308 webpage,
\myurl{http://types.cs.washington.edu/jsr308/}.  The four checkers, respectively,
help to prevent and detect null pointer errors (via a \code{@NonNull}
annotation), equality-checking errors (via a \code{@Interned} annotation),
mutation errors (via the Javari~\cite{BirkaE2004,TschantzE2005} type
system), and mutation errors (via the IGJ~\cite{ZibinPAAKE2007} type
system).  A paper~\cite{PapiACPE2008} discusses experience
in which these plug-ins exposed bugs in real programs.


\subsection{Examples of type qualifiers\label{type-qualifier-examples}}

The ability to place annotations on arbitrary occurrences of a type
improves the expressiveness of annotations, which has many benefits for
Java programmers.  Here we mention just one use that is enabled by extended
annotations, namely the creation of type qualifiers.
(Figure~\ref{fig:example} gives an example of the use of type qualifiers.)

\begin{figure*}[t]
\begin{center}

%% Hevea handles Verbatim, but not its commandchars or numbers options.
%% I tried "%BEGIN IMAGE" here, but it didn't work.
%BEGIN LATEX
\begin{quote}                   % avoid line numbers in left column
% \newcommand{\NonNull}[1]{@NonNull }
\newcommand{\NonNull}[1]{\relax}
\begin{Verbatim}[commandchars=\\\[\],numbers=left,numbersep=6pt]
class DAG {

    Set<\NonNull Edge> edges;                   \label[edgesNonNull]

    // ...

    \NonNull List<\NonNull Vertex> getNeighbors(@Readonly DAG this, @Interned @Readonly Vertex v) { \label[getNeighbors]\label[InternedAnnotation]
        \NonNull List<\NonNull Vertex> neighbors = new LinkedList<\NonNull Vertex>();
        for (\NonNull Edge e : edges)           \label[eNonNull]
            if (e.from() == v)                  \label[useE1]\label[objectequality]
                neighbors.add(e.to());          \label[useE2]
        return neighbors;
    }
}
\end{Verbatim}
\end{quote}
\vspace{-\baselineskip}
%END LATEX

\begin{rawhtml}
<DIV CLASS="center"><PRE CLASS="verbatim"> 1  class DAG {
 2
 3      Set&lt;Edge&gt; edges;          <a name="edgesNonNull"></a>
 4
 5      // ...
 6
 7      List&lt;Vertex&gt; getNeighbors(@Readonly DAG this, @Interned @Readonly Vertex v) { <a name="getNeighbors"></a><a name="InternedAnnotation"></a>
 8          List&lt;Vertex&gt; neighbors = new LinkedList&lt;Vertex&gt;();
 9          for (Edge e : edges)                <a name="eNonNull"></a>
10              if (e.from() == v)              <a name="useE1"></a><a name="objectequality"></a>
11                  neighbors.add(e.to());      <a name="useE2"></a>
12          return neighbors;
13      }
14  }
</PRE></DIV>
\end{rawhtml}

\end{center}

\caption{
The \code{DAG} class, which represents a directed acyclic graph,
illustrates how type qualifiers might be written by a
programmer and checked by a type-checking plug-in in order to detect or
prevent errors.  Typical code uses less than 1 type annotation per 50
lines~\cite{PapiACPE2008}, but this example was chosen to illustrate places
where annotations do appear.
% Most programs would use only a subset of these annotations.
\newline
\textbf{(1) Nullness:}
The \<@NonNull>annotation is the default, so 
no reference in the \code{DAG}
class may be null unless otherwise annotated.  It is equivalent to
writing line~\ref{edgesNonNull} as ``\code{@NonNull Set<@NonNull Edge> edges;}'',
for example.
% The @NonNull annotation on line~\ref{eNonNull}
This guarantees that the uses
of \code{edges} on line~\ref{eNonNull}, and of \code{e} on lines~\ref{useE1} and~\ref{useE2}, cannot cause
a null pointer exception.
% That annotation is correct because each
% element of \code{edges} is non-null, as indicated on
% line~\ref{edgesNonNull}.
Similarly, the return type of \code{getNeighbors()} (line~\ref{getNeighbors})
is \code{@NonNull}, which
enables its clients to depend on the fact that it will always return a
\code{List}, even if \code{v} has no neighbors.
\newline
\textbf{(2) Immutability:}
The two \code{@Readonly} annotations on method \code{getNeighbors}
(line~\ref{getNeighbors}) guarantee to clients that the method does not
modify, respectively, its \code{Vertex} argument or its \code{DAG} receiver
(including its \code{edges} set or any edge in that set).
The lack of a \code{@Readonly} annotation on the return value indicates
that clients are free to modify the returned \code{List}.
\newline
\textbf{(3) Interning:}
The \code{@Interned} annotation on line~\ref{InternedAnnotation} (along
with an \code{@Interned} annotation on the return type in the declaration
of \code{Edge.from()}, not shown) indicates that
the use of object equality (\code{==}) on line~\ref{objectequality} is
valid.  In the absence of such annotations, use of the
\code{equals} method is preferred to \code{==}.
% \end{itemize}
}

\label{fig:example}
\end{figure*}


As an example of how JSR 308 might be used, consider a
\code{@NonNull} type qualifier that signifies that a variable should
never be assigned \code{null}~\cite{Detlefs96,Evans96,Detlefs-etal98,FahndrichL2003,ChinMM2005}.
% The code in Figure~\ref{fig:example}
% demonstrates \code{@NonNull} in a sample method.
A programmer can
annotate any use of a type with the \code{@NonNull} annotation.
A compiler plug-in would check that a \code{@NonNull} variable is never
assigned a possibly-\code{null} value, thus enforcing the \code{@NonNull}
type system.

%% Commented out to avoid complaints from Oracle.
% A pair
% of plug-ins would check that \code{@NonNull} variables are never
% assigned a possibly-\code{null} value: one plug-in for the compiler, which
% would traverse a program's abstract syntax tree and use the compiler's
% API to report any errors it finds, and another plug-in for the virtual
% machine that would analyze annotation information stored in the class
% file when the class is loaded. The compile-time checker is needed during development
% to alert a programmer of violations of a type qualifier's semantics. The
% load-time checker is equally important, so
% that the guarantees provided by the type qualifier are extended to the
% user whenever the program is run.



\code{@Readonly} and \code{@Immutable} are other examples of useful type
qualifiers~\cite{ZibinPAAKE2007,BirkaE2004,TschantzE2005,GreenfieldboyceF2005,KnieselT2001,SkoglundW2001,PoratBKM2000}.
Similar to C's \code{const}, an object's internal state may not be modified
through references that are declared \code{@Readonly}. A type qualifier
designer would create a compiler plug-in (an annotation processor) to check
the semantics of \code{@Readonly}.  For instance, a method may only be
called on a \code{@Readonly} object if the method was declared with a
\code{@Readonly} receiver.
(Each non-static method has an implicit formal parameter, \code{this}, which is
called the \emph{receiver}.)
\code{@Readonly}'s immutability guarantee can
help developers avoid accidental modifications, which are often manifested
as run-time errors.
An immutability annotation can also improve performance.
The Access Intents mechanism of WebSphere Application Server already
incorporates such functionality:
a programmer can indicate that a particular method (or all methods) on
an Enterprise JavaBean is readonly.

Additional examples of useful type qualifiers abound.  We mention just a
few others.
% Java uses \code{final} to indicate a reference may not be assigned to (this
% is orthogonal to the notion of immutability of the referred-to object).
C uses the \code{const}, \code{volatile}, and \code{restrict} type qualifiers.
%
Type qualifiers \code{YY} for two-digit year strings and \code{YYYY} for
four-digit year strings helped to detect, then verify the absence of, Y2K
errors~\cite{ElsmanFA99}.
%
Expressing units of measurement (e.g., SI units such as meter, kilogram,
second) can prevent errors in which a program mixes incompatible
quantities; units such as dollars can prevent other errors.
%
Range constraints, also known as ranged types, can indicate that a
particular \code{int} has a value between 0 and 10; these are often
desirable in realtime code and in other applications, and are supported in
languages such as Ada and Pascal.
%
Type qualifiers can indicate data that originated from an untrustworthy
source~\cite{PalsbergO95,VolpanoS97}; examples for C include \code{user} vs.\ \code{kernel} indicating user-space and
kernel-space pointers in order to prevent attacks on operating systems~\cite{JohnsonW2004}, and
\code{tainted} for strings that originated in user input and that should
not be used as a format string~\cite{ShankarTFW2001}.
%
A \code{localizable} qualifier can indicate where translation of
user-visible messages should be performed.  Annotations can
indicate other
properties of its contents, such as the format or encoding of a string
(e.g., XML, SQL, human language, etc.).
%  (multibyte, UTF, etc.)
% Add a citation to the Microsoft multibyte/unibyte work; I can't find one.
%
\code{Local} and \code{remote} qualifiers can indicate whether particular
resources are available on the same machine or must be retrieved over the
network.
%
An \code{interned} qualifier can indicate which objects have been converted
to canonical form and thus may be compared via reference equality.
Type qualifiers such as \code{unique}
and \code{unaliased} can express properties about pointers and
aliases~\cite{Evans96,ChinMM2005}; other qualifiers can detect and prevent
deadlock in concurrent programs~\cite{foster:pldi02,aiken:pldi03}.
A \code{ThreadSafe} qualifier~\cite{Goetz2006} could indicate that a given
field should contain a thread-safe implementation of a given interface;
this is more flexible than annotating the interface itself to require that
\emph{all} implementations must be thread-safe.
%
Annotations can identify performance characteristics or goals; for
example, some collections should not be iterated over, and others should
not be used for random access.
%
Annotations (both type qualifiers and others) can specify cut points in
aspect-oriented programming (AOP\@)~\cite{EichbergM2004}.
Flow-sensitive type qualifiers~\cite{foster:pldi02} can express typestate
properties such as whether a file is in the open, read, write, readwrite,
or closed state, and can guarantee that a file is opened for reading before
it is read, etc.  The Vault language's type guards and capability states are
similar~\cite{DelineF2001}.


\subsection{Example tools that do pluggable type-checking for type qualifiers\label{type-qualifier-tools}}

The Checker Framework
(\myurl{http://types.cs.washington.edu/checker-framework/})
gives a way to create pluggable type-checkers.  A pluggable type-checker
verifies absence of certain bugs (and also verifies correct usage of type
qualifiers).  The Checker Framework is distributed with a set of example
type-checkers.  The Checker Framework is built on the Type Annotations (JSR
308) syntax, though it also permits annotations to be written in comments
for compatibility with previous versions of Java.


\subsection{Uses for annotations on types\label{type-annotation-use-cases}}

This section gives examples of annotations that a programmer may wish to
place on a type.
Each of these uses is either impossible or extremely inconvenient in the
absence of the new locations for annotations proposed in this document.
For brevity, we do not give examples of uses for every type annotation.
The specific annotation names used in this
section, such as \code{@NonNull}, are examples only; this document does not
define any annotations, merely specifying where they can appear in Java
code.


% This paragraph is out of place; where should it go?
It is worthwhile to permit annotations on all uses of types (even those for
which no immediate use is apparent) for consistency, expressiveness, and
support of unforeseen future uses.
%
An annotation need not utilize every
possible annotation location.  For example, a system that fully specifies
type qualifiers in signatures but infers them for implementations~\cite{GreenfieldboyceF2005}
may not need annotations on casts, object creation, local variables, or
certain other locations.  Other
systems may forbid top-level (non-type-argument, non-array) annotations
on object creation (\code{new}) expressions, such as \code{new @Interned
  Object()}.
% However, the annotation system proposed here is expressive
% enough to handle arbitrary type qualifiers.


\paragraph{Generics and arrays\label{generics-and-arrays}}
% Annotations on generic type arguments and arrays are necessary so the
% programmer can fully specify types.
Generic collection classes are
declared one level at a time, so it is easy to annotate each level
individually.

It is desirable that the syntax for arrays be equally expressive.
% , for
% uniformity with generics, and for the same reasons that annotations on
% generic type arguments are useful.
Here are examples of uses for annotations on array levels:
\begin{itemize}
\item
  The Titanium~\cite{YelickSPMLKHGGCA1998} dialect of Java requires the
  ability to place the
  \code{local} annotation (indicating that a memory reference in a parallel
  system refers to data on the same processor) on various levels of an
  array, not just at the top level.
\item
  In a dependent type system~\cite{PfenningF:deptlp,Xi1998,Xi-Pfenning99},
  one wishes to specify the dimensions of an
  array type, such as \code{Object @Length(3) [] @Length(10) []} for a
  3$\times$10 array.
\item
  An immutability type system, as discussed in Section~\ref{type-qualifier-examples},
  needs to be able to specify which levels of
  an array may be modified.  Consider specifying a procedure that inverts a
  matrix in place.  The procedure formal parameter type should guarantee that the procedure
  does not change the shape of the array (does not replace any of the rows
  with another row of a different length), but must permit changing
  elements of the inner arrays.  In other words, the top-level array is
  immutable, the inner arrays are mutable, and their elements
  are immutable.
\item
  An ownership domain system~\cite{AbiAntounA2006:OOPSLA} uses array
  annotations to indicate properties of arrays, similarly to type
  parameters.
\item
  \newcommand{\bs}{\char"5C}
  The ability to specify the nullness of the array and its elements
  separately is so important that JML~\cite{LeavensBR2006:JML} includes
  special syntax \code{\bs nonnullelements(a)} for a possibly-null array
  \code{a} with non-null elements.

  A simple example is a method that accepts a list of files to search.
  \code{null} may be used to indicate that there were no files to search,
  but when a list is provided, then the \code{File}s themselves must be
  non-null.  Using JSR 308, a programmer would declare the formal parameter as
  \code{@NonNull File @Nullable [] filesToSearch} --- more concisely,
  depending on the default nullness, as either \code{File @Nullable []
    filesToSearch} or \code{@NonNull File [] filesToSearch}

%% Examples from Eclipse's Parser (org.eclipse.jdt.internal.compiler.parser):
%     protected ASTNode[] astStack = new ASTNode[AstStackIncrement];
%     protected Expression[] expressionStack = new
%     Expression[ExpressionStackIncrement];
%     protected ASTNode[] genericsStack = new ASTNode[GenericsStackIncrement];
%     protected char[][] identifierStack;
%     this.identifierStack = new char[30][];
% These are all created with a certain size with all elements as null, but 
% the references to the stacks themselves are never null.


%% Examples, from a simple grep for "array.*or null" in the HTML Javadoc.
%% (There must be many more than the ones matching that simple search!)
%% I omitted examples places where the array elements may be null, and also
%% where they are of non-null type, such as primitive arrays.
% java.security.CodeSource.getCertificates
% java.security.CodeSource.getCodeSigners
% java.nio.charset.Charset constructor
% javax.imageio.spi.ImageReaderWriterSpi constructor
% javax.imageio.spi.ImageReaderWriterSpi.getFileSuffixes
% javax.imageio.spi.ImageReaderWriterSpi.getMIMETypes
% javax.imageio.spi.ImageReaderWriterSpi.getExtraStreamMetadataFormatNames
% javax.imageio.spi.ImageReaderWriterSpi.getExtraImageMetadataFormatNames
% javax.net.ssl.SSLParameters.getCipherSuites
% javax.net.ssl.SSLParameters.setCipherSuites
% javax.net.ssl.SSLParameters.getProtocols
% javax.net.ssl.SSLParameters.setProtocols
% javax.net.ssl.SSLParameters constructors
% javax.net.ssl.X509KeyManager.getClientAliases
% javax.net.ssl.X509KeyManager.chooseClientAlias
% javax.net.ssl.X509KeyManager.getServerAliases
% javax.net.ssl.X509KeyManager.chooseServerAlias
% javax.net.ssl.X509KeyManager.getCertificateChain
% javax.imageio.metadata.IIOMetadataFormatImpl.getChildNames
% javax.imageio.metadata.IIOMetadataFormat.getChildNames

  The opposite example, of a non-null array with nullable elements, is
  typical of fields in which, when an array element is no longer
  relevant, it is set to null to permit garbage collection.

%% I don't see how this point belongs in this list.
\item
  In a type system for preventing null pointer errors, using a default of
  non-null, and explicitly annotating references that may be null, results
  in the fewest annotations and least user
  burden~\cite{FahndrichL2003,ChalinJ2007,PapiACPE2008}.  Array elements
  can often be null (both due to initialization, and for other reasons),
  necessitating annotations on them.
\end{itemize}


\paragraph{Receivers\label{receivers}}
A type qualifier on a formal parameter is a contract regarding what the
method may (or may not) do with that formal parameter.  Since the method receiver
(\code{this}) is an implicit formal parameter, programmers should be able
to express type qualifiers on it, for consistency and expressiveness.  An
annotation on the receiver is a contract regarding what the method may
(or may not) do with its receiver.


% (In Java's syntax, the receiver's type is implicit rather than explicitly
% written in the source code of the method.)

For example, consider the following method:

\preverbnegspace
\begin{Verbatim}
  package javax.xml.bind;
  class Marshaller {
    void marshal(@Readonly Marshaller this,
                 @Readonly Object jaxbElement,
                 @Mutable Writer writer) {
      ...
    }
  }
\end{Verbatim}

\noindent
The annotations indicate that \code{marshal} modifies its second formal parameter
but does not modify its first formal parameter nor its receiver.

The syntax also permits expressing constraints on the
generic type parameters of the receiver.  Here are some examples:

\preverbnegspace
\begin{Verbatim}
  class Collection<E> {
    // The elements of the result have the same annotation as the elements
    // of the receiver.  (In fact, they are the same elements.)
    public @PolyNull Object[] toArray(Collection<@PolyNull E> this) { ... }
  }
  interface List<T> {
    // The size() method changes neither the receiver nor any of the elements.
    public int size(@Readonly List<@Readonly T> this) { ... }
  }
  class MyMap<T,U> {
    // The map's values must be non-null, but the keys may be arbitrary.
    public void requiresNonNullValues(MyMap<T, @NonNull U> this) { ... }
  }
\end{Verbatim}

A method in an inner class has two \<this> formal parameters:  that for the inner
class, and that for the enclosing instance (which is \<this> in the currently-executing method in the outer class).  It
is desirable to specify the types for both.  
One use case for this is a read-only type system, where you want to
make the distinction between the outer and inner object.
This can be specified as

\preverbnegspace
\begin{Verbatim}
  void m(@ReadOnly Outer.@ReadWrite Inner this) {}
\end{Verbatim}


A receiver annotation is different than a class annotation, a method
annotation, or a return value annotation:
\begin{itemize}
\item
  There may be different receiver annotations on different methods that
  cannot be factored out into the containing class.

\item
  Stating that a method does not modify its receiver is different than
  saying the method has no side effects at all, so it is not appropriate as
  a method annotation (such as JML's \code{pure}
  annotation~\cite{LeavensBR2006:JML}).

\item
  A receiver annotation is also distinct from a return value
  annotation:  a method might modify its receiver but return an immutable
  object, or might not modify its receiver but return a mutable object.
\end{itemize}

% As another example, this receiver annotation
% 
% \preverbnegspace
% \begin{Verbatim}
%   Dimension getSize(@Readonly MyClass this) { ... }
% \end{Verbatim}
% 
% indicates that \code{getSize} does not modify its receiver.

\noindent
Since a receiver annotation is distinct from other
annotations, new syntax is required for the receiver annotation.

%% This is about the C++ receiver syntax, which Java used to use.
% The syntax
% is adopted from C++, just as the overall syntax of Java was.
% This syntax is cleaner than creating a parallel annotation, such as
% \code{@ReadonlyReceiver}, for each type annotation, and is also cleaner
% than changing the definition of annotations to permit writing the similar
% \code{@Receiver(@Readonly)}.

% In Joe Darcy's words, this is a syntactic problem, so a syntactic
% solution like a new location for receiver annotations is cleaner than a
% non-syntactic solution.

As with Java's annotations on formal parameters, annotations on the
receiver do not affect the Java signature,
compile-time resolution of overloading, or run-time resolution of
overriding.  The Java type of every receiver in a class is the same --- but
their annotations, and thus their qualified type in a type qualifier
framework, may differ.

Some people question the need for receiver annotations.  In
case studies~\cite{PapiACPE2008}, every type system required some
receiver annotations.  Even the Nullness type system required them to
express whether the receiver was fully initialized (only in a
fully-initialized object can fields be guaranteed to be non-null).
So, the real question is how to express receiver annotations, not
whether they should exist.


\paragraph{Casts}
There are two distinct reasons to annotate the type in a type cast:  to
fully specify the casted type (including annotations that are retained
without change), or to indicate
an application-specific invariant that is beyond the reasoning capability
of the Java type system.
Because a user can apply a type cast to any expression, a user can
annotate the type of any expression.  (This is different than annotating
the expression itself, which is not legal.)
%; see Section~\ref{expression-annotations}.)

\begin{enumerate}
\item
Annotations on type casts permit the type in a type cast to be fully
specified, including any appropriate annotations.  In this case, the
annotation on the cast is the same as the annotation on the type of the
operand expression.  The annotations are preserved, not changed, by the
cast, and the annotation serves as a reminder of the type of the cast
expression.  For example, in

\preverbnegspace
\begin{Verbatim}
  @Readonly Object x;
  ... (@Readonly Date) x ...
\end{Verbatim}

\noindent
the cast preserves the annotation part of the type and changes only the
Java type.  If a cast could not be annotated, then a cast would remove the
annotation:

\preverbnegspace
\begin{Verbatim}
  @Readonly Object x;
  ... (Date) x ...       // annotation processor issues warning due to casting away @Readonly
\end{Verbatim}

This cast changes the annotation; it uses \code{x} as a
non-\code{@Readonly} object, which changes its type and would require a
run-time mechanism to enforce type safety.


An annotation processor could permit the unannotated cast syntax but
implicitly add the annotation, treating the cast type as \code{@Readonly
  Date}.  This has the advantage of brevity, but the disadvantage of being
less explicit and of interfering somewhat with the second use of cast
annotations.  Experience will indicate which design is better in practice.


% In some cases, it is safe to change the annotation part of the type when
% that is guaranteed to be safe --- for instance, casting from
% non-\code{@Readonly} to @Readonly --- but since a non-\code{@Readonly}
% expression can always be used in a @Readonly context, and overloading
% cannot be performed based on annotations alone, such a cast is never
% strictly necessary.


\item
A second use for annotations on type casts is --- like ordinary Java casts
--- to provide the compiler with information that is beyond the ability of
its typing rules.  Such properties are often called ``application
invariants'', since they are facts guaranteed by the logic of the
application program.

As a trivial example, the following cast changes the annotation but is
guaranteed to be safe at run time:

\preverbnegspace
\begin{Verbatim}
  final Object x = new Object();
  ... (@NonNull Object) x ...
\end{Verbatim}

An annotation processing tool could trust such type casts, perhaps issuing
a warning to remind users to verify their safety by hand or in some other
manner.  An alternative approach would be to check the type cast
dynamically, as Java casts are, but we do not endorse such an approach,
because annotations are not intended to change the run-time behavior of a
Java program and because there is not generally a run-time representation
of the annotations.

%   (For example, annotations, like return values, are not part
% of the Java signature, so they do not affect compile-time resolution of
% overloading nor run-time resolution of overriding.)

\end{enumerate}


\paragraph{Type tests}
Annotations on type tests (\code{instanceof}) allow the programmer to
specify the full type, as in the first justification for annotations on
type casts, above.  However, the annotation is not tested at run time ---
the JVM only checks the base Java type.
In the
% preferred
implementation, there is no run-time representation of the annotations on
an object's type, so dynamic type test cannot determine whether an
annotation is present.  This abides by the intention of the Java annotation
designers, that annotations should not change the run-time behavior of a
Java program.

Annotation of the type test permits the idiom

\ifhevea
\begin{Verbatim}
  if (x instanceof MyType) {
    ... (MyType) x ...
  }
\end{Verbatim}
\else
\preverbnegspace
\begin{Verbatim}[commandchars=\\\[\]]
  if (x instanceof \emph[MyType]) {
    ... (\emph[MyType]) x ...
  }
\end{Verbatim}
\fi

\noindent
to be used with the same annotated type \emph{\code{T}} in both
occurrences.  By contrast, using different types in the type test and the
type cast might be confusing.

To prevent confusion caused by incompatible annotations, an annotation
processor could require the annotation parts of the operand and the type to
be the same:

\preverbnegspace
\begin{Verbatim}
  @Readonly Object x;
  if (x instanceof Date) { ... }            // error: incompatible annotations
  if (x instanceof @Readonly Date) { ... }  // OK
  Object y;
  if (y instanceof Date) { ... }            // OK
  if (y instanceof @NonNull Date) { ... }   // error: incompatible annotations
\end{Verbatim}

(As with type casts, an annotation processor could implicitly add a
missing annotation; this would be more concise but less explicit, and
experience will dictate which is better for users.)

As a consequence of the fact that the annotation is not checked at run
time, in the following

\preverbnegspace
\begin{Verbatim}
  if (x instanceof @A1 T) { ... }
  else if (x instanceof @A2 T) { ... }
\end{Verbatim}
the second conditional is always dead code.  An annotation processor may
warn that one or both of the \code{instanceof} tests is a compile-time type
error.

% javac does not warn about dead code in
%   if (x instanceof Object) { ... }
%   else if (x instanceof Object) { ... }


A non-null qualifier is a special case because it is possible to check at
run time whether a given value can have a non-null type.  A type-checker for
a non-null type system could take advantage of this fact, for instance to
perform flow-sensitive type analysis in the presence of a \code{x != null}
test, but JSR 308 makes no special allowance for it.


\paragraph{Object creation}
Java's \code{new} operator indicates the type of the object being created.
As with other Java syntax, programmers should be able to indicate the full
type, even if in some cases (part of) the type can be inferred.
In some cases, the annotation cannot be inferred; for instance, it is
impossible to tell whether a particular object is intended to be mutated
later in the program or not, and thus whether it should have a
\code{@Mutable} or \code{@Immutable} annotation.
Annotations on object creation expressions could also be statically verified
(at compile time) to be compatible with the annotations on the constructor.

% Such annotations could optionally disambiguate among multiple
% constructors --- for example, a \code{@Readonly} version might choose a
% code path that performs no side effects.

\paragraph{Type bounds}
Annotations on type parameter bounds (\code{extends})
and wildcard bounds (\code{extends} and \code{super})
allow the programmer to fully constrain generic types.  Creation of
objects with constrained generic types could be statically verified to
comply with the annotated bounds.

\paragraph{Inheritance}
Annotations on class inheritance (\code{extends} and
\code{implements}) are necessary to allow a programmer to fully
specify a supertype. It would otherwise be impossible to extend the
annotated version of a particular type $t$ (which is often a valid
subtype or supertype of $t$) without using an anonymous class.

\noindent
\label{class-inheritance-annotations}
These annotations also provide a convenient way to alias otherwise
cumbersome types. For instance, a programmer might declare

\preverbnegspace
\begin{Verbatim}
  final class MyStringMap extends
    @Readonly Map<@NonNull String, @NonEmpty List<@NonNull @Readonly String>> {}
\end{Verbatim}

\noindent
so that \code{MyStringMap} may be used in place of the full,
unpalatable supertype.  However, this does somewhat limit reusability
since a method declared to take a \code{MyStringMap} cannot take a
\code{Map} of the appropriate type.
% also see Section~\ref{typedef} for problems with this approach.)

\paragraph{Throws clauses}
Annotations in the \code{throws} clauses of method declarations allow
programmers to enhance exception types. For instance, programs that
use the \code{@Critical} annotation from the above examples could be
statically checked to ensure that \code{catch} blocks that can catch a
\code{@Critical} exceptions are not empty.

% Actually, the JLS3 grammar for CatchClause refers to FormalParameter,
% which is not itself defined, so maybe special syntax is needed,
% depending on how that error is fixed.

There is no need for special syntax to permit annotations on the type of a
caught exception; it is already permitted, as in

\preverbnegspace
\begin{Verbatim}
  catch (@NonCritical Exception e) { ... }
\end{Verbatim}

\noindent
In this example case, a tool could warn if any \<@Critical> exception can
reach the \<catch> clause.

% However, special syntax is provided for disjunctive and conjunctive types that arise
% from a multi-\<catch> clause.
% It is possible to annotate the individual alternatives, as in
% 
% \preverbnegspace
% \begin{Verbatim}
%   catch ( @A E1 | @A E2 | @A E3 e ) { ... }
% \end{Verbatim}
% 
% \noindent
% but it is desirable to annotate the disjunctive type as a whole, since a
% programmer often wants to annotate each disjuct identically.  This code
% desugars to the above:
% % It would be a shame to force a programmer to repeat the annotations, since
% % the multi-\<catch> syntax is intended to reduce code size.
% 
% \preverbnegspace
% \begin{Verbatim}
%   catch ( @A | E1 | E2 | E3 e ) { ... }
% \end{Verbatim}
% 
% It is possible to specify both a type annotation on the disjunction and
% also type annotations on (some of) the disjuncts.  Each disjuct is treated
% as containing all the annotations on the disjunct, plus all the annotations
% on itself.  Conjuncts are treated analogously.


\subsection{Related work\label{related-work}}

Section~\ref{type-qualifier-examples} gave many examples of how type
qualifiers have been used in the past.  Also see the related work section
of~\cite{PapiACPE2008}.

C\#'s attributes~\cite[chap.~24]{ECMA334-4th} play the same role as Java's
annotations:  they attach metadata to specific parts of a program, and are
carried through to the compiled bytecode representation, where they can be
accessed via reflection.  The syntax is different:  C\# uses
\code{[AnnotationName]} or \code{[AnnotationName:  data]} where Java uses
\code{@AnnotationName} or
\code{@AnnotationName(data)}; C\# uses \code{AttributeUsageAttribute} where
Java uses \code{Target}; and so forth.
However, C\# permits metadata
on generic arguments, and C\# permits multiple metadata instances of the
same type to appear at a given location.


Like Java, C\# does not permit
metadata on elements within a method body.
(The ``[a]C\#'' language~\cite{CazzolaCC2005:JOT}, whose name is pronounced
``annotated C sharp'', is an extension to C\# that permits annotation of
statements and code blocks.)

Harmon and Klefstad~\cite{HarmonK2007} propose a standard for worst-case
execution time annotations.


Pechtchanski's dissertation~\cite{Pechtchanski2003} uses annotations in the
aid of dynamic program optimization.  Pechtchanski implemented an extension
to the Jikes compiler that supports stylized comments, and uses these
annotations on classes, fields, methods, formals, local variable
declarations, object creation (\code{new}) expressions, method invocations
(calls), and program points (empty statements).  The annotations are
propagated by the compiler to the class file.


Mathias Ricken's LAPT-javac
(\myurl{http://www.cs.rice.edu/~mgricken/research/laptjavac/}) is a version
of javac (version 1.5.0\_06) that
encodes annotations on local variables in the class file, in new
\code{Runtime\-\{Inv},\code{V\}isible\-Local\-Variable\-Annotations} attributes.
The class file format of LAPT-javac differs from that proposed in this
document.
Ricken's xajavac (Extended Annotation Enabled
javac) permits subtyping of annotations (\myurl{http://www.cs.rice.edu/~mgricken/research/xajavac/}).


The Java Modeling Language, JML~\cite{LeavensBR2006:JML}, is a behavioral
modeling language for writing specifications for Java code.  It uses
stylized comments as annotations, some of which apply to types.


Ownership types~\cite{ClarkePN98,Boyapati2004:PhD,Clark2001,ClarkD2002,PotaninNCB2006,NobleVP98,DietlM2005,LeinoM2004,YuP2006}
permit programmers to control aliasing and access among objects.  Ownership
types can be expressed with type annotations
% with type qualifiers??
and have been applied to program
verification~\cite{LeinoM2004,Muller2002,MullerPHL2006}, thread
synchronization~\cite{BoyapatiLR2002,JacobsPLS2005}, memory
management~\cite{AndreaCGNVZ2006,BoyapatiSBR2003}, and representation
independence~\cite{BanerjeeN2002}.


JavaCOP~\cite{AndreaeNMM2006} is a framework for implementing pluggable
type systems in Java.  Whereas JSR 308 uses standard
interfaces such as the Tree API and the JSR 269 annotation processing
framework, JavaCOP defines its own incompatible variants.
A JavaCOP type checker must be programmed in a combination of Java and
JavaCOP's own declarative pattern-matching and rule-based language.
JavaCOP's authors have defined parts of over a dozen
type-checkers in their language.  Their paper does not report that they have
run any of these type-checkers on a real program; this is due to
limitations that make JavaCOP impractical (so far) for real use.


JACK makes annotations on array brackets refer to the array, not the
elements of the array~\cite{MalePPD2008}.
% "Java bytecode verification for @NonNull types" says:
% @NonNull Integer @NonNull [] a1;
% When annotating arrays, the leftmost annotation associates with the element
% type, whilst that just before the braces associates with the array
% reference type.



%% This appendix has been dispersed, mostly to the normative (non-appendix)
%% part of the document.
% \section{Discussion of Java language syntax extensions\label{syntax-discussion}}
% 
% In Java SE 7, annotations can be written only on method formal parameters
% and the declarations of packages, classes, methods, fields, and local variables.
% Additional annotations are necessary in order to fully specify Java classes and
% methods.






% \subsection{Target meta-annotation for type annotations}
% \label{target-meta-annotation}
% 
% 
% 
% 
% The remainder of this section provides background on the \code{Target}
% meta-annotation.
% 
% Java's \code{Target} meta-annotation provides machine-readable
% documentation of the intended use of an annotation.  For example, the
% \code{@Override} annotation is intended for use only on method
% declarations, so its definition is written as
% 
% \preverbnegspace
% \begin{Verbatim}
%   @Target(ElementType.METHOD)
%   public @interface Override { }
% \end{Verbatim}
% 
% 
% Java SE 7 requires the compiler to issue an error if a programmer places an
% annotation in a location not permitted by its Target meta-annotation.
% % Thus, it is already the case that examination of the
% % source alone does not indicate whether the annotation will be rejected by
% % the compiler.
% 
% \preverbnegspace
% \begin{Verbatim}
%   @Override                                     // compile-time error
%   class Window extends Container {
%     @Override String warningString;             // compile-time error
%     @Override Container getContainer() { ... }  // OK; correct use of @Override
% \end{Verbatim}


\section{Tool modifications\label{mods-discussion}}

As a consequence of JSR 308's changes to the Java syntax and class file
format (Sections~\ref{syntax} and~\ref{class-file}), several tools need to
be updated.

These tools include:
\begin{itemize}
\item
  The Java compiler accepts type annotations and adds them to the program's
  AST\@.  It writes them to the classfile, including accounting for
  synthetic methods (such as bridge methods) and optimizations.  Its
  \code{-Xprint} functionality prints type annotations.

  When producing bytecode for an earlier version of the
  virtual machine, via the \code{-target} command-line option, a Java 8
  compiler is permitted to place type
  annotations in declaration attributes.

\item
  The Java Model AST of JSR 198 (Extension API for Integrated Development
  Environments)~\cite{JSR198} represents all new locations for annotations.

  Oracle's Tree API, which exposes the AST (including annotations) to
  authors of javac annotation processors (compile-time plug-ins),
  represents type annotations, per Section~\ref{syntax}.  The same goes for
  other Java compilers, such as that of Eclipse.

  The JSR 269 (annotation processing) model
  represents type annotations that are visible down to class member
  declarations (the top of Figure~\ref{tbl:target_types}), but not within a
  method, such as on individual statements and expressions.

\item
  No modifications to the virtual machine are necessary.
  (Some changes to reflection do change virtual machine APIs in a minor
  way, but the representation of execution of bytecodes is unaffected.)
  
  The \code{javap} disassembler recognizes the new class
  file format and outputs annotations.
  
  The pack200/unpack200 tool preserves the new attributes through a
  compress-decompress cycle.
  % http://download.oracle.com/javase/7/docs/api/java/util/jar/Pack200.html

\item
  Javadoc  outputs type annotations in class and method signatures.

\end{itemize}


% \subsection{Compiler\label{compiler-mods}}
% 
% The Java compiler must accept type annotations
% and add them to the program's AST\@.  The relevant AST node
% classes must also be modified to store these annotations.
% When generating \code{.class} files, the compiler must emit the attributes
% described in Section~\ref{class-file}.
% % (The JLS needs to be updated to so state.)
% 
% % Section~\ref{compiler}
% This specification requires compilers to place certain annotations
% in the class file.  This is consistent with the principle that annotations
% should not affect behavior:  in the absence of an annotation processor, the
% compiler produces the same bytecodes for annotated code as it would have
% for the same code without annotations.  (The class file may differ, since
% the annotations are stored in it, but the bytecode part does not differ.)
% 
% The compiler sometimes creates new methods or other constructs
% that did not appear in the
% source code.  The compiler must annotate this compiler-generated code
% in an appropriate way.  For example, a bridge method is an implementation
% strategy used when the erased signature of the actual method being invoked
% differs from that of the compile-time method
% declaration~\cite[\S15.12.4.5]{GoslingJSB2005}.  The bridge method does
% nothing but call an existing method.  The compiler must copy annotations
% from the method being invoked to the bridge method.
% As another example, an anonymous class desugars into a class declaration
% and instantiation.  A programmer may write both type and declaration
% annotations on an anonymous class in Java source code.  In the classfile,
% the type annotations are copied to the class declaration and to its
% instantiation (\<new> expression).  The declaration annotations are copied
% only to the class declaration.
% 
% %% Cut this -- too compiler-specific.
% % and
% % constructors for anonymous classes~\cite[\S15.9.5.1]{GoslingJSB2005}.
% 
% %% Remarks about javac are not appropriate in a specification.
% % (As of Java SE 6, javac does not copy/transfer any annotations from
% % original methods to the compiler-generated methods; that is probably a bug
% % in javac.  It is, however, perhaps debatable whether \emph{all} annotations
% % should be copied.)
% 
% Javac's \code{-Xprint} functionality reads a \code{.class} file and prints
% the interface (class declarations with signatures of all fields and
% methods).  (The \code{-Xprint} functionality is similar to javap, but
% cannot provide
% any information about bytecodes or method bodies, because it is implemented
% internally as an annotation processor.)
% This must be updated to print the type annotations as well.
% Also see Section~\ref{class-file-tools}.



% \subsubsection{Backward and forward compatibility\label{mods-compatibility}}
% 
% Java compilers can often produce bytecode for an earlier version of the
% virtual machine, via the \code{-target} command-line option.  For example,
% a programmer could execute a compilation command such as \code{javac
% -source 8 -target 5 MyFile.java}.  A Java 8 compiler produces a class file
% with the same attributes for type annotations as when the target is a
% version 8 JVM\@.  However, the compiler is permitted to also place type
% annotations in declaration attributes.  For instance, the annotation on the
% top level of a return type would also be placed on the method (in the
% method attribute in the class file).  This enables class file analysis
% tools that are written for Java SE 5 to view a subset of the type
% qualifiers (lacking generics, array levels, method receivers, etc.), albeit
% attached to declarations.
% 
% A user can use a Java SE 7 compiler to compile a Java class that contains
% type annotations, so long as the type annotations only appear in places
% that are legal in Java SE 7.  Furthermore, the compiler must be provided
% with a definition of the annotation that is meta-annotated not with
% \code{@Target(ElementType.TYPE\_USE)} (since \code{ElementType.TYPE\_USE}
% does not exist in Java SE 7), but with no meta-annotation or with a
% meta-annotation that permits annotations on any declaration.
% 
% 
% \subsubsection{Optimizations\label{mods-optimization}}
% 
% 
% The compiler is required to preserve annotations in the class file.  More
% precisely, if an expression has an annotated type (and the annotation has
% class file or runtime retention), and that expression is represented
% in the compiled class file, then the annotation must be present, in the
% compiled class file, on the type of the compiled representation of the
% expression.  For expressions within a field initializer or a static
% initializer block, the compiled representation might be in an \code{<init>}
% or \code{<clinit>} method, or in a helper method called by them.
% 
% If the compiler optimizes away an expression, then it may also
% optimize away the annotation.  (Exception:  when a type cast
% is optimized away without optimizing away its argument, the
% annotation remains on the argument; see Section~\ref{target_type}.)
% 
% 
% 
% This may change the compiler implementation of certain optimizations, such
% as common subexpression elimination, but this restriction on the compiler
% implementation is unobjectionable for three reasons.
% \begin{enumerate}
% \item
% Java-to-bytecode compilers rarely perform sophisticated
% optimizations, since the bytecode-to-native (JIT) compiler is the major
% determinant in Java program performance.  Thus, the restriction will not
% affect most compilers.
% % (We need a list of problematic optimizations, and a catalog of widely used
% % Java-to-bytecode compilers indicating whether each one performs the
% % optimizations.)
% \item
% The compiler workarounds are simple.  Suppose that two expressions
% that are candidates for common subexpression elimination
% have different type annotations.
% % or statements
% A compiler could:  not perform the optimization when the annotations
% differ; create a single expression whose type has both of the annotations
% (e.g., merging \code{(@Positive Integer) 42} and \code{(@Even Integer) 42}
% into \code{(@Positive @Even Integer) 42});
% or create an unannotated expression and copy its value into two variables
% with differently-annotated types.
% % 
% \item
% It seems unlikely that two identical, non-trivial expressions would
% have differently-annotated types.  Thus, any compiler restrictions will
% have little or no effect on most compiled programs.
% % (We should try to predict or measure this number.)
% \end{enumerate}




% \subsubsection{Annotations in comments\label{annotations-in-comments}}
% 
% To ease the transition from standard Java SE 7 code to code with type
% annotations, the reference implementation recognizes the type
% annotations when surrounded by comment markers:
% 
% \preverbnegspace
% \begin{Verbatim}
%   List</*@Readonly*/ Object> myList;
% \end{Verbatim}
% 
% \noindent
% This permits use of both standard Java SE 7 tools and the new annotations even
% before Java SE 8 is released.  However, it is not part of the proposal;
% that is, it is not required that every Java compiler parses comments.
% Oracle's OpenJDK implementation does not support a switch that
% makes it recognize the new annotations when embedded in comments.
% The Spec\#~\cite{BarnettLS2004} extension to C\# can be made compilable by
% a standard C\# compiler in a similar way, by enclosing its annotations in
% special \verb|/*^|\ldots\verb|^*/| comment markers.
% The \code{/*@} comment syntax is a standard part of the
% Splint~\cite{Evans96}, ESC/Java~\cite{FlanaganLLNSS02}, and
% JML~\cite{LeavensBR2006:JML} tools (that is, not
% with the goal of backward compatibility).
% % Javadoc has /**@ comments.



% \subsection{ASTs and annotation processing\label{asts-and-annotation-processing}}
% 
% The Java Model AST of JSR 198 (Extension API for Integrated Development
% Environments)~\cite{JSR198} gives access to the entire source code of a
% method.  This AST (abstract syntax tree) must be updated to represent all
% new locations for annotations.
% 
% Oracle's Tree API, which exposes the AST (including annotations) to authors of
% javac annotation processors (compile-time plug-ins), must be updated to
% reflect the modifications made to the internal AST node classes described in
% Section~\ref{syntax}.
% The same goes for other Java compilers, such as that of Eclipse.
% 
% 
% \label{jsr269-changes}
% 
% Like reflection, the JSR 269 (annotation processing) model
% does not represent constructs below the
% method level, such as individual statements and expressions.  Therefore, it
% needs to be updated only with respect to declaration-related annotations
% (the top of Figure~\ref{tbl:target_types}.
% % ; also see Section~\ref{out-of-scope-annotation-processing-api}).
% The JSR 269 model, \code{javax.lang.model.*},
% already has some classes representing annotations; see
% \myurl{http://download.oracle.com/javase/7/docs/api/javax/lang/model/element/package-summary.html}.
% The annotation processing API in \code{javax.annotation.processing} must
% also be revised.
% 
% 
% 
% 
% \subsection{Virtual machine and class file analysis tools\label{class-file-tools}}
% 
% % Since load-time plug-ins
% % % (described in Section~\ref{load-time}) can use
% % the \code{premain} method,
% No modifications to the virtual machine are necessary.
% (The changes to reflection (Section~\ref{reflection}) do change virtual
% machine APIs in a minor way, but the representation of execution of
% bytecodes is unaffected.)
% 
% The \code{javap} disassembler must recognize the new class
% file format and must output annotations.
% 
% The pack200/unpack200 tool must preserve the new attributes through a
% compress-decompress cycle.
% % http://download.oracle.com/javase/7/docs/api/java/util/jar/Pack200.html
% 
% 
% The compiler and other tools that read class files are trivially compatible
% with class files produced by a Java SE 7 compiler.  However, the tools
% would not be able to read the impoverished version of type qualifiers that
% is expressible in Java SE 7 (see Section~\ref{compiler-mods}).  It is
% desirable for class file tools to be able to read at least that subset of
% type qualifiers.  Therefore, APIs for reading annotations from a class file
% must be dependent on the class file version (as a number of APIs already
% are).  If the class file version indicates Java SE 7, and none of the
% extended annotations defined by JSR 308 appear in the class file, then the
% API may return (all) annotations from declarations when queried for the
% annotations on the top-level type associated with the declaration (for
% example, the top-level return type, for a method declaration).
% 
% 
% \subsection{Other tools\label{other-tools}}
% 
% Javadoc must output annotations at the new locations when those are part
% of the public API, such as in a
% method signature.
% %% (Does the doclet API itself need to change??
% 
% % This is true but irrelevant: "(if meta-annotated with \code{@Documented})"
% % http://download.oracle.com/javase/7/docs/api/java/lang/annotation/Documented.html
% % @Documented annotations "are part of the public API of the annotated
% % elements", and "are documented by javadoc and similar tools by default".
% % (No option exists at present in the standard doclet to override that default.)
% 
% Similar modifications need to be made to tools outside the Oracle JDK,
% such as IDEs (Eclipse, IDEA, JBuilder, jEdit, NetBeans), other tools that
% manipulate Java code (grammars for CUP, javacc), and tools that
% manipulate class files (ASM, BCEL\@).  These changes need to be made by the
% authors of the respective tools.
% % Eugene Kuleshov has volunteered to make the ASM updates.
% % (\myurl{http://asm.objectweb.org/})
% % (One example in ASM:  \code{util.TraceClassVisitor} prints annotations and
% % must be extended to print the new annotations.)
% 
% 
% A separate document, ``Custom type qualifiers via annotations on Java
% types'' (\myurl{http://types.cs.washington.edu/jsr308/java-type-qualifiers.pdf}),
% explores implementation strategies for annotation processors that act as type-checking
% compiler plug-ins.  It is not germane to this proposal, both because this proposal
% does not concern itself with annotation semantics and because writing such
% plug-ins does not require any changes beyond those described in this document.
% 
% 
% A separate document, ``Annotation File Specification''
% (\myurl{http://types.cs.washington.edu/annotation-file-utilities/annotation-file-format.pdf}), describes a
% textual format for annotations that is independent of \code{.java} or
% \code{.class} files.  This textual format can represent annotations for
% libraries that cannot or should not be modified.  We have built
% tools for manipulating annotations, including extracting
% annotations from and inserting annotations in \code{.java} and
% \code{.class} files.
% % Such tools will be useful in transitioning to use of the new annotations.
% That file format is not part of this proposal for
% extending Java's annotations; it is better viewed as an implementation
% detail of our tools.



% \subsection{Java Language Specification}
% 
% The Java Language Specification needs to be updated.  A separate document
% will one day contain a list of all the changes.  (The list of changes would
% be too long to include in this document, and would not succinctly convey
% the implications of the changes.)  Finding all the places that mention
% annotations is a good start.
% 
% One nonobvious place is in places like section~9.6.1.5, where the
% description of \code{SuppressWarnings} states
% ``If a program declaration is annotated \ldots''
% and the use of ``declaration'' must be changed to ``element'' or
% ``construct'' or some similar word.
% 
% 
% \subsection{Java Virtual Machine Specification}
% 
% The Java Virtual Machine Specification (JVMS) needs to be updated to
% describe the \code{Runtime\{Inv,V\}isibleType\-Annotations} attributes.
% This section, or a separate document, will one day contain a list of all
% the changes.  Since the third edition of JVMS~\cite{LindholmY:JVMS3} has
% not yet been published, the changes must work from the document titled
% ``revisions to `The class file Format'\,'', which this document calls
% ``JVMS3'' and is available at
% \myurl{http://java.sun.com/docs/books/jvms/second_edition/ClassFileFormat-Java5.pdf}.


\section*{Acknowledgments}

The co-spec-lead, Alex Buckley, made significant contributions throughout
the process, to both the design and this document.  The JSR 308 Expert
Group also have helpful feedback and suggestions.

Matt Papi, Mahmood Ali, and Werner Dietl designed and implemented the JSR 308 compiler as
modifications to Oracle's OpenJDK javac compiler, and contributed to the JSR
308 design.

The members of the JSR 308 mailing list
(\myurl{http://groups.google.com/group/jsr308-discuss}) provided
valuable comments and suggestions.
%% This was accurate as of the end of 2006.
%% I have not kept track of additional feedback.
% We thank
% %
%  Joshua Bloch,
%  Gilad Bracha,
%  Alex Buckley,
%  Wayne Carr,
%  Bruce Chapman,
%  Joe Darcy,
%  Jeff Foster,
%  Neal Gafter,
%  David Greenfieldboyce,
%  Evan Ireland,
%  Sacha Labourey,
%  Doug Lea,
%  Todd Millstein,
%  R. Matthew McCutchen,
%  Ted Neward,
%  Jens Palsberg,
%  Bill Pugh,
%  Jaime Quinonez,
%  Matthew Tschantz,
% and
%  Eugene Vigdorchik
% %
% for their comments and suggestions.
%
Additional feedback is welcome.

% http://jcp.org/en/press/news/2007award_winners
JSR 308 received
the Most Innovative Java SE/EE JSR of the Year award in 2007,
at the 5th annual JCP Program Awards.
% http://jcp.org/en/press/news/2008JavaOnePR
JSR 308's spec leads (Michael
Ernst and Alex Buckley) were nominated as Most Outstanding Spec Lead for
Java SE/EE in 2008, at the 6th annual JCP Program Awards.
% http://java.sun.com/javaone/2009/rockstars.jsp
Michael Ernst won a Java Rock Star award for a presentation on the Checker
Framework, which builds on the Type Annotations syntax, at JavaOne 2009.





\bibliographystyle{alpha}
\bibliography{bibstring-unabbrev,types,ernst,invariants,generals,alias,concurrency}

\end{document}

% LocalWords:  fontsize pt NonNull getNeighbors const Readonly WonderfulList A1
% LocalWords:  VM toolchain visibleTypeAnnotations JLS checkcast ops op pc anno
% LocalWords:  LocalVariableTable AbstractProcessor Messager premain javac YY
% LocalWords:  YYYY localizable unaliased typestate readwrite mpapi annotatable
% LocalWords:  ASM Bracha instanceof java Jens Palsberg multibyte UTF JIT un ta
% LocalWords:  getAnnotations getParameterAnnotations getReceiverAnnotation TCK
% LocalWords:  NonEmpty getReceiverAnnotations Neward Quinonez tion param JSRs
% LocalWords:  visibleParameterAnnotations sourced Sacha Labourey Vigdorchik lt
% LocalWords:  unibyte MyStringMap Hevea commandchars DIV gt edgesNonNull useE
% LocalWords:  InternedAnnotation LinkedList eNonNull objectequality JavaBean
% LocalWords:  WebSphere wildcards rodocs roarray util TraceClassVisitor Xprint
% LocalWords:  metadata AnnotationName AttributeUsageAttribute Pechtchanski's
% LocalWords:  Pechtchanski getSize Javac's ElementType NonNegative TODO javap
% LocalWords:  PrintStream println sec Klefstad int Ricken's LAPT TypeParameter
% LocalWords:  Inv isibleLocalVariableAnnotations JCP doclet javadoc JBuilder
% LocalWords:  jEdit NetBeans javacc BCEL nonobvious li href isibleType EE arr
% LocalWords:  SuppressWarnings dArray Kuleshov ClassFileFormat pdf num Dietl
% LocalWords:  DefaultAnnotation strictfp saveenumi ReadOnly ArrayAnnots roa O2
% LocalWords:  ArrayAnnot MyAnnotation ReadonlyDims Vitek LNCS Potanin myString
% LocalWords:  myObject GuardedBy accessLock Goetz JavaCOP signedness Nonnull
% LocalWords:  NotNull JavaCOP's TypeArguments BasicType ForVarControl nullness
% LocalWords:  FormalParameterDecls ForVarControlRest FormalParameterDeclsRest
% LocalWords:  FormalParameters VoidMethodDeclaratorRest BracketsOpt myVar 2x5
% LocalWords:  QualifiedTypeList MethodDeclaratorRest IdentifierSuffix MyClass
% LocalWords:  InterfaceMethodDeclaratorRest VoidInterfaceMethodDeclaratorRest
% LocalWords:  ConstructorDeclaratorRest VariableDeclaratorRest nonnullelements
% LocalWords:  VariableInitializer ConstantDeclaratorRest VariableDeclaratorId
% LocalWords:  UnannType MethodOrFieldDecl Unann MethodOrFieldRest langtools
% LocalWords:  InterfaceMethodOrFieldDecl InterfaceMethodOrFieldRest realtime
% LocalWords:  FindBugs CheckForNull Nullable desugar classfile var Alast BNF
% LocalWords:  isibleTypeAnnotations isible Papi's varargs Afirst ELTS toolset
% LocalWords:  MultipleAnnotations subannotation Subannotations MyFile enum pre
% LocalWords:  ThreadSafe Aldrich AContainer Javadocs bytecode JVMs bytecodes
% LocalWords:  wildcard superinterface supertype API readonly XML AOP Foo JML
% LocalWords:  JML's AtomicReferenceFieldUpdater atomicity Gafter Subclassing
% LocalWords:  subclassing subtyping OpenJDK Jikes xajavac Papi Gilad Millstein
% LocalWords:  Greenfieldboyce McCutchen Tschantz webpage IGJ ESC filesToSearch
% LocalWords:  DefaultQualifier grep nullable MyRepeatedAnnotationContainer lst
% LocalWords:  AnnotatedElement getRepeatedAnnotation annotationClass ASTNode
% LocalWords:  OuterClass InnerClass ReadonlyReceiver lastAccessedTime astStack
% LocalWords:  getReturnAnnotation getReturnAnnotations getAnnotation enums ldc
% LocalWords:  SingleElementAnnotation ValueMember AstStackIncrement dividedBy
% LocalWords:  expressionStack ExpressionStackIncrement genericsStack YAML wiki
% LocalWords:  GenericsStackIncrement identifierStack getDeclaredAnnotations
% LocalWords:  ReceiverNonNull superclasses TypeArgument runtime newMap ernst
% LocalWords:  Nonnegative typedefs unabbrev reusability typedef getstatic init
% LocalWords:  QualifiedType toArray localvar ReadWrite 2x5 0x86 0x00 0x02 0x10
% LocalWords:  ClassFile clinit StackMapTable StaticNested boolean var1 0x12
% LocalWords:  ReferenceType VariableModifier UnannReferenceType OrReceiver 2x5
% LocalWords:  FormalParameter Decls FormalParameterOrReceiverDecls nestings
% LocalWords:  NestedStatic decompiler myLock Reified arr2 syntaxes TargetType
% LocalWords:  SillyAnnotation 0x00 0x01 0x02 0x03 0x04 0x05 0x06 0x07 0x08
% LocalWords:  0x09 0x0A 0x0B 0x0C 0x0D 0x0E 0x0F 0x10 0x11 0x12 0x13 0x14 0x15
% LocalWords:  0x16 0x17 0x18 0x19 0x1A 0x1B 0x1C 0x1D 0x1E 0x1F 0x20 0x21 0x22
% LocalWords:  0x23 0x24 0x25 308's casted useE1 useE2 pack200 unpack200 0x14
% LocalWords:  Mahmood theReceiver 2x5 foo 0x00 0x01 0x02 0x03 0x04 0x05 0x16
% LocalWords:  0x06 0x07 0x08 0x09 0x0A 0x0B 0x0C 0x0D 0x0E 0x0F 0x10 2x5 0x18
% LocalWords:  0x11 0x12 0x13 0x14 0x15 0x16 0x17 0x18 0x19 0x1A 0x1B 0x00 0x1A
% LocalWords:  0x1C 0x1D 0x1E 0x1F 0x20 0x21 0x22 0x23 0x24 0x25 308's 7in 0x1C
%  LocalWords:  0x01 0x02 0x03 0x10 0x11 0x12 0x13 0x14 0x15 0x16 0x17 2x5 0x1E
%  LocalWords:  0x18 0x19 0x1A 0x1B 0x1C 0x1D 0x1E 0x1F 0x80 0x81 0x82 2x5 0x80
%  LocalWords:  0x83 0x84 0x85 0x86 0x87 0x88 0x89 0x8A 0x8B 0x8C 0x8D 0x82
%  LocalWords:  0x8E 0x8F 308's 0x00 0x01 0x02 0x03 innerMethod 0x86 0x00 0x84
% LocalWords:  0x88 0x8A 0x8C 0x8E 0x90 0x92 308's multi
